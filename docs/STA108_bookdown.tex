\documentclass[12pt,]{book}
\usepackage{lmodern}
\usepackage{amssymb,amsmath}
\usepackage{ifxetex,ifluatex}
\usepackage{fixltx2e} % provides \textsubscript
\ifnum 0\ifxetex 1\fi\ifluatex 1\fi=0 % if pdftex
  \usepackage[T1]{fontenc}
  \usepackage[utf8]{inputenc}
\else % if luatex or xelatex
  \ifxetex
    \usepackage{mathspec}
  \else
    \usepackage{fontspec}
  \fi
  \defaultfontfeatures{Ligatures=TeX,Scale=MatchLowercase}
\fi
% use upquote if available, for straight quotes in verbatim environments
\IfFileExists{upquote.sty}{\usepackage{upquote}}{}
% use microtype if available
\IfFileExists{microtype.sty}{%
\usepackage{microtype}
\UseMicrotypeSet[protrusion]{basicmath} % disable protrusion for tt fonts
}{}
\usepackage[margin=1in]{geometry}
\usepackage{hyperref}
\hypersetup{unicode=true,
            pdftitle={Applied Statistical Methods: Regression Analysis},
            pdfauthor={Shizhe Chen},
            pdfborder={0 0 0},
            breaklinks=true}
\urlstyle{same}  % don't use monospace font for urls
\usepackage{natbib}
\bibliographystyle{apalike}
\usepackage{color}
\usepackage{fancyvrb}
\newcommand{\VerbBar}{|}
\newcommand{\VERB}{\Verb[commandchars=\\\{\}]}
\DefineVerbatimEnvironment{Highlighting}{Verbatim}{commandchars=\\\{\}}
% Add ',fontsize=\small' for more characters per line
\usepackage{framed}
\definecolor{shadecolor}{RGB}{248,248,248}
\newenvironment{Shaded}{\begin{snugshade}}{\end{snugshade}}
\newcommand{\KeywordTok}[1]{\textcolor[rgb]{0.13,0.29,0.53}{\textbf{#1}}}
\newcommand{\DataTypeTok}[1]{\textcolor[rgb]{0.13,0.29,0.53}{#1}}
\newcommand{\DecValTok}[1]{\textcolor[rgb]{0.00,0.00,0.81}{#1}}
\newcommand{\BaseNTok}[1]{\textcolor[rgb]{0.00,0.00,0.81}{#1}}
\newcommand{\FloatTok}[1]{\textcolor[rgb]{0.00,0.00,0.81}{#1}}
\newcommand{\ConstantTok}[1]{\textcolor[rgb]{0.00,0.00,0.00}{#1}}
\newcommand{\CharTok}[1]{\textcolor[rgb]{0.31,0.60,0.02}{#1}}
\newcommand{\SpecialCharTok}[1]{\textcolor[rgb]{0.00,0.00,0.00}{#1}}
\newcommand{\StringTok}[1]{\textcolor[rgb]{0.31,0.60,0.02}{#1}}
\newcommand{\VerbatimStringTok}[1]{\textcolor[rgb]{0.31,0.60,0.02}{#1}}
\newcommand{\SpecialStringTok}[1]{\textcolor[rgb]{0.31,0.60,0.02}{#1}}
\newcommand{\ImportTok}[1]{#1}
\newcommand{\CommentTok}[1]{\textcolor[rgb]{0.56,0.35,0.01}{\textit{#1}}}
\newcommand{\DocumentationTok}[1]{\textcolor[rgb]{0.56,0.35,0.01}{\textbf{\textit{#1}}}}
\newcommand{\AnnotationTok}[1]{\textcolor[rgb]{0.56,0.35,0.01}{\textbf{\textit{#1}}}}
\newcommand{\CommentVarTok}[1]{\textcolor[rgb]{0.56,0.35,0.01}{\textbf{\textit{#1}}}}
\newcommand{\OtherTok}[1]{\textcolor[rgb]{0.56,0.35,0.01}{#1}}
\newcommand{\FunctionTok}[1]{\textcolor[rgb]{0.00,0.00,0.00}{#1}}
\newcommand{\VariableTok}[1]{\textcolor[rgb]{0.00,0.00,0.00}{#1}}
\newcommand{\ControlFlowTok}[1]{\textcolor[rgb]{0.13,0.29,0.53}{\textbf{#1}}}
\newcommand{\OperatorTok}[1]{\textcolor[rgb]{0.81,0.36,0.00}{\textbf{#1}}}
\newcommand{\BuiltInTok}[1]{#1}
\newcommand{\ExtensionTok}[1]{#1}
\newcommand{\PreprocessorTok}[1]{\textcolor[rgb]{0.56,0.35,0.01}{\textit{#1}}}
\newcommand{\AttributeTok}[1]{\textcolor[rgb]{0.77,0.63,0.00}{#1}}
\newcommand{\RegionMarkerTok}[1]{#1}
\newcommand{\InformationTok}[1]{\textcolor[rgb]{0.56,0.35,0.01}{\textbf{\textit{#1}}}}
\newcommand{\WarningTok}[1]{\textcolor[rgb]{0.56,0.35,0.01}{\textbf{\textit{#1}}}}
\newcommand{\AlertTok}[1]{\textcolor[rgb]{0.94,0.16,0.16}{#1}}
\newcommand{\ErrorTok}[1]{\textcolor[rgb]{0.64,0.00,0.00}{\textbf{#1}}}
\newcommand{\NormalTok}[1]{#1}
\usepackage{longtable,booktabs}
\usepackage{graphicx,grffile}
\makeatletter
\def\maxwidth{\ifdim\Gin@nat@width>\linewidth\linewidth\else\Gin@nat@width\fi}
\def\maxheight{\ifdim\Gin@nat@height>\textheight\textheight\else\Gin@nat@height\fi}
\makeatother
% Scale images if necessary, so that they will not overflow the page
% margins by default, and it is still possible to overwrite the defaults
% using explicit options in \includegraphics[width, height, ...]{}
\setkeys{Gin}{width=\maxwidth,height=\maxheight,keepaspectratio}
\IfFileExists{parskip.sty}{%
\usepackage{parskip}
}{% else
\setlength{\parindent}{0pt}
\setlength{\parskip}{6pt plus 2pt minus 1pt}
}
\setlength{\emergencystretch}{3em}  % prevent overfull lines
\providecommand{\tightlist}{%
  \setlength{\itemsep}{0pt}\setlength{\parskip}{0pt}}
\setcounter{secnumdepth}{5}
% Redefines (sub)paragraphs to behave more like sections
\ifx\paragraph\undefined\else
\let\oldparagraph\paragraph
\renewcommand{\paragraph}[1]{\oldparagraph{#1}\mbox{}}
\fi
\ifx\subparagraph\undefined\else
\let\oldsubparagraph\subparagraph
\renewcommand{\subparagraph}[1]{\oldsubparagraph{#1}\mbox{}}
\fi

%%% Use protect on footnotes to avoid problems with footnotes in titles
\let\rmarkdownfootnote\footnote%
\def\footnote{\protect\rmarkdownfootnote}

%%% Change title format to be more compact
\usepackage{titling}

% Create subtitle command for use in maketitle
\providecommand{\subtitle}[1]{
  \posttitle{
    \begin{center}\large#1\end{center}
    }
}

\setlength{\droptitle}{-2em}

  \title{Applied Statistical Methods: Regression Analysis}
    \pretitle{\vspace{\droptitle}\centering\huge}
  \posttitle{\par}
    \author{Shizhe Chen}
    \preauthor{\centering\large\emph}
  \postauthor{\par}
      \predate{\centering\large\emph}
  \postdate{\par}
    \date{2020-04-10}

\usepackage{booktabs}
\usepackage{amsthm}
\makeatletter
\def\thm@space@setup{%
  \thm@preskip=8pt plus 2pt minus 4pt
  \thm@postskip=\thm@preskip
}
\makeatother

\begin{document}
\maketitle

{
\setcounter{tocdepth}{1}
\tableofcontents
}
\chapter*{Preface}\label{pre}
\addcontentsline{toc}{chapter}{Preface}

This Gitbook contains code for STA 108. Lecture notes can be found in
folder `notes' or on the course website on Canvas. The Githbook is a
work-in-progress, and materials in this Gitbook are updated constantly.

\chapter{\texorpdfstring{Linear regression with
\texttt{R}}{Linear regression with R}}\label{ch:lmR}

Reading materials: Slides 3 - 11 in STA108\_LinearRegression\_S20.pdf.

Fitting a linear model is simple in \texttt{R}. The bare minimum
requires you to know only two functions \texttt{lm()} and
\texttt{summary()}. We will apply linear regression on three data set
\texttt{advertising}, \texttt{flu\ shot}, and \texttt{Project\ STAR}.

\section{Advertising data}\label{advertising-data}

This data is taken from
\href{http://faculty.marshall.usc.edu/gareth-james/ISL/}{An Introduction
to Statistical Learning}, by James et al. A brief description from
Section 2.1 in ISL is provided below. You may read more about the data
set in the free e-book.

\begin{quote}
The Advertising data set consists of the sales of that product in 200
different markets, along with advertising budgets for the product in
each of those markets for three different media: TV, radio, and
newspaper\ldots{} It is not possible for our client to directly increase
sales of the product. On the other hand, they can control the
advertising expenditure in each of the three media. Therefore, if we
determine that there is an association between advertising and sales,
then we can instruct our client to adjust advertising budgets, thereby
indirectly increasing sales. In other words, our goal is to develop an
accurate model that can be used to predict sales on the basis of the
three media budgets.
\end{quote}

It is important to note that we will avoid making any causal statements
(i.e., increasing TV advertising budget will increase the sales by
\ldots{}) in our analysis. Causal inference on observational data is
another major field in statistics. We will limit our discussion on
association.

\begin{Shaded}
\begin{Highlighting}[]
\NormalTok{dat.advertising=}\KeywordTok{read.csv}\NormalTok{(}\StringTok{'./data/advertising.csv'}\NormalTok{);}
\CommentTok{# Use the code in Appendix C to visualize this data set}
\KeywordTok{library}\NormalTok{(ggplot2)}
\KeywordTok{library}\NormalTok{(GGally)}
\KeywordTok{ggpairs}\NormalTok{(dat.advertising)}
\end{Highlighting}
\end{Shaded}

\begin{Shaded}
\begin{Highlighting}[]
\CommentTok{# What is X in this data frame?}
\NormalTok{dat.advertising=dat.advertising[,}\OperatorTok{-}\DecValTok{1}\NormalTok{]}
\end{Highlighting}
\end{Shaded}

We will consider a simple linear regression model in this chapter. The
only covariate/predictor/independent variable that we use here is the TV
advertising budget. Therefore, the model is \[
    y_i = x_i \beta_1  +\beta_0 +  \epsilon_i, i=1,\ldots, 200,
    \] where \(y_i\) is the \texttt{sales} (1 unit = 1000 dollars) for
the \(i\)th entry, \(x_i\) is the TV advertising budget for the \(i\)th
entry, \(\beta_0\) is the intercept term, and \(\beta_1\) is the
regression slope. In addition, we assume that the errors
\(\{\epsilon_i\}_{i=1}^{200}\) satisfy that
\(\epsilon_1,\ldots, \epsilon_200\) are independently and identically
distributed (i.i.d.), \(\mathbb{E}[\epsilon_i]= 0\) for
\(i=1,2,\ldots, 200\) and \(\mathrm{var}(\epsilon_i)=\sigma^2\) for
\(i=1,2,\ldots, 200\). Recall that we consider fixed design (i.e.,
\(x_i\) is not random) in this course for simplicity.

\begin{Shaded}
\begin{Highlighting}[]
\CommentTok{# Fit a simple linear regression}
\NormalTok{fit.advertising =}\StringTok{ }\KeywordTok{lm}\NormalTok{(sales}\OperatorTok{~}\NormalTok{TV}\OperatorTok{+}\DecValTok{1}\NormalTok{,}\DataTypeTok{data=}\NormalTok{dat.advertising); }
\CommentTok{# Summarize the fitted results}
\KeywordTok{summary}\NormalTok{(fit.advertising) }
\end{Highlighting}
\end{Shaded}

We see that \(\hat{\beta}_1\) equals 0.05 and \(\hat{\beta}_0\) equals
7.03. We can draw the line in the scatter plot.

How would you interpret the fitted slope and intercept?

\begin{Shaded}
\begin{Highlighting}[]
\CommentTok{# Draw the fitted line using ggplot2 }
\KeywordTok{ggplot}\NormalTok{(}\DataTypeTok{data =}\NormalTok{ dat.advertising) }\OperatorTok{+}\StringTok{ }
\StringTok{  }\KeywordTok{geom_point}\NormalTok{(}\DataTypeTok{mapping =} \KeywordTok{aes}\NormalTok{(}\DataTypeTok{x =}\NormalTok{ TV, }\DataTypeTok{y =}\NormalTok{ sales)) }\OperatorTok{+}
\StringTok{  }\KeywordTok{geom_line}\NormalTok{(}\DataTypeTok{data =} \KeywordTok{fortify}\NormalTok{(fit.advertising ), }\KeywordTok{aes}\NormalTok{(}\DataTypeTok{x =}\NormalTok{ TV, }\DataTypeTok{y =}\NormalTok{ .fitted),}\DataTypeTok{color=}\StringTok{'red'}\NormalTok{)}
\end{Highlighting}
\end{Shaded}

\begin{Shaded}
\begin{Highlighting}[]
\CommentTok{# In fact, you can use the following code without fitting the lm()}

\KeywordTok{ggplot}\NormalTok{(}\DataTypeTok{data =}\NormalTok{ dat.advertising,a}\KeywordTok{es}\NormalTok{(}\DataTypeTok{x =}\NormalTok{ TV, }\DataTypeTok{y =}\NormalTok{ sales)) }\OperatorTok{+}\StringTok{ }
\StringTok{  }\KeywordTok{geom_point}\NormalTok{() }\OperatorTok{+}
\StringTok{  }\KeywordTok{geom_smooth}\NormalTok{(}\DataTypeTok{method =} \StringTok{"lm"}\NormalTok{)}
\end{Highlighting}
\end{Shaded}

\begin{Shaded}
\begin{Highlighting}[]
\CommentTok{# The band is a confidence band }
\end{Highlighting}
\end{Shaded}

\section{Flu shot}\label{flu-shot}

The data come from the following paper: McDonald, C., Hiu, S., AND
Tierney, W. (1992). Effects of computer reminders for influenza
vaccination on morbidity during influenza epidemics. MD Computing 9,
304-312. (\href{https://www.ncbi.nlm.nih.gov/pubmed/1522792}{link}).

Briefly, in this study, physicians were randomly selected to receive a
letter encouraging them to inoculate patients at risk for flu. The
treatment of interest is the actual flu shot, and the outcome is an
indicator for flu-related hospital visits from the patients. In this
data set, you can also find four important background covariates
including gender, age, a chronic obstructive pulmonary disease (COPD)
indicator, and a heart disease indicator. For further details, read the
original publication or Hirano, Imbens, Rubin and Zhou (2000)
(\href{https://www.ncbi.nlm.nih.gov/pubmed/12933526}{link}), or Chapter
25 in Imbens and Rubin (2015)
(\href{https://www.cambridge.org/core/books/causal-inference-for-statistics-social-and-biomedical-sciences/71126BE90C58F1A431FE9B2DD07938AB}{link}).

\begin{Shaded}
\begin{Highlighting}[]
\NormalTok{dat.flu =}\StringTok{ }\KeywordTok{read.table}\NormalTok{(}\StringTok{"./data/flu240.txt"}\NormalTok{, }\DataTypeTok{header =} \OtherTok{TRUE}\NormalTok{)}

\CommentTok{# Take a look at size of the data before proceed:}
\KeywordTok{dim}\NormalTok{(dat.flu)}
\KeywordTok{colnames}\NormalTok{(dat.flu)}
\CommentTok{# Basic visulization:}
\KeywordTok{ggpairs}\NormalTok{(dat.flu) }\CommentTok{# This might take a while..}
\end{Highlighting}
\end{Shaded}

\begin{Shaded}
\begin{Highlighting}[]
\CommentTok{# Do you think the plots are informative?}

\CommentTok{# One strategy: jittering}

\CommentTok{# Without jittering:}
\KeywordTok{ggplot}\NormalTok{(dat.flu, }\KeywordTok{aes}\NormalTok{(}\DataTypeTok{x=}\NormalTok{treatment.assigned, }\DataTypeTok{y=}\NormalTok{outcome)) }\OperatorTok{+}\StringTok{ }\KeywordTok{geom_point}\NormalTok{()}
\end{Highlighting}
\end{Shaded}

\begin{Shaded}
\begin{Highlighting}[]
\CommentTok{# With jittering:}
\KeywordTok{ggplot}\NormalTok{(dat.flu, }\KeywordTok{aes}\NormalTok{(}\DataTypeTok{x=}\NormalTok{treatment.assigned, }\DataTypeTok{y=}\NormalTok{outcome)) }\OperatorTok{+}\StringTok{ }\KeywordTok{geom_jitter}\NormalTok{(}\DataTypeTok{width =} \FloatTok{0.25}\NormalTok{)}
\end{Highlighting}
\end{Shaded}

Again, we will consider a simple linear regression model for now. The
only covariate/predictor/independent variable that we use here is the
treatment assignment. Therefore, the model is \[
    y_i = x_i \beta_1  +\beta_0 +  \epsilon_i, i=1,\ldots, 2891,
    \] where \(y_i\) is the \texttt{outcome} (0 or 1) for the \(i\)th
patient, \(x_i\) is the treatment assignment (0 or 1) for the \(i\)th
patient's physician, \(\beta_0\) is the intercept term, and \(\beta_1\)
is the regression slope. In addition, we assume that the errors
\(\{\epsilon_i\}_{i=1}^{2891}\) satisfy that
\(\epsilon_1,\ldots, \epsilon_2891\) are independently and identically
distributed (i.i.d.), \(\mathbb{E}[\epsilon_i]= 0\) for
\(i=1,2,\ldots, 2891\) and \(\mathrm{var}(\epsilon_i)=\sigma^2\) for
\(i=1,2,\ldots, 2891\). The following code is almost identical to those
in the previous section.

How would you interpret the fitted slope and intercept?

Are there any violations to the assumptions on
\(\{\epsilon_i\}_{i=1}^2891\)?

\begin{Shaded}
\begin{Highlighting}[]
\CommentTok{# Fit a simple linear regression}
\NormalTok{fit.flu=}\StringTok{ }\KeywordTok{lm}\NormalTok{(outcome}\OperatorTok{~}\NormalTok{treatment.assigned}\OperatorTok{+}\DecValTok{1}\NormalTok{,}\DataTypeTok{data=}\NormalTok{dat.flu); }
\CommentTok{# Summarize the fitted results}
\KeywordTok{summary}\NormalTok{(fit.flu) }

\CommentTok{# Draw the fitted line using ggplot2 }
\KeywordTok{ggplot}\NormalTok{(dat.flu, }\KeywordTok{aes}\NormalTok{(}\DataTypeTok{x=}\NormalTok{treatment.assigned, }\DataTypeTok{y=}\NormalTok{outcome)) }\OperatorTok{+}\StringTok{ }
\StringTok{  }\KeywordTok{geom_jitter}\NormalTok{(}\DataTypeTok{width =} \FloatTok{0.25}\NormalTok{)}\OperatorTok{+}
\StringTok{  }\KeywordTok{geom_smooth}\NormalTok{(}\DataTypeTok{method =} \StringTok{"lm"}\NormalTok{)}
\end{Highlighting}
\end{Shaded}

\section{Project STAR}\label{project-star}

Tennesses Student/Teacher Achievement Ratio study (Project STAR) was
conducted in the late 1980s to evaluate the effect of class size on test
scores. This dataset has been used as a classic examples in many
textbooks and research papers. Brieftly, the study randomly assigned
students to small classes, regular classes, and regular classes with a
teacher's aide. In order to randomize properly, schools were enrolled
only if they had enough studybody to have at least one class of each
type. Once the schools were enrolled, students were randomly assigned to
the three types of classes, and one teacher was randomly assigned to one
class. You can read more about the study on the Harvard dataverse
(\href{https://dataverse.harvard.edu/dataverse/star}{link}) or Chapter 9
in Imbens and Rubin (2015)
(\href{https://www.cambridge.org/core/books/causal-inference-for-statistics-social-and-biomedical-sciences/71126BE90C58F1A431FE9B2DD07938AB}{link}).

\begin{Shaded}
\begin{Highlighting}[]
\KeywordTok{library}\NormalTok{(AER)}
\KeywordTok{data}\NormalTok{(}\StringTok{"STAR"}\NormalTok{)}
\NormalTok{dat.STAR=STAR; }\CommentTok{# Just to be consistent}
\CommentTok{# Take a look at size of the data before proceed:}
\KeywordTok{dim}\NormalTok{(dat.STAR)}
\KeywordTok{colnames}\NormalTok{(dat.STAR)}
\CommentTok{# With 47 variables, you might want to read the help file to see what these variables are.}


\CommentTok{# Basic visulization:}
\CommentTok{# ggpairs(dat.flu) # This might take a really long time ...}

\CommentTok{# We will consider the math scores and the class assignments in the second grade}

\CommentTok{# With jittering:}
\KeywordTok{ggplot}\NormalTok{(dat.STAR, }\KeywordTok{aes}\NormalTok{(}\DataTypeTok{x=}\NormalTok{star2, }\DataTypeTok{y=}\NormalTok{math2)) }\OperatorTok{+}\StringTok{ }
\StringTok{  }\KeywordTok{geom_jitter}\NormalTok{(}\DataTypeTok{width =} \FloatTok{0.25}\NormalTok{)}
\end{Highlighting}
\end{Shaded}

Again, we will consider a simple linear regression model for now. We
will study the association between math scores and the class assignments
in the second grade. Therefore, the model is \[
    y_i =\beta_0 + x_{i,1} \beta_1  +x_{i,2} \beta_2  +  \epsilon_i, i=1,\ldots, 2891,
    \] where \(y_i\) is the math score for the \(i\)th student,
\(x_{i,1}\) is an indicator (0 or 1) whether the \(i\)th student is in
the small class, and \(x_{i,1}\) is an indicator (0 or 1) whether the
\(i\)th student is in the regular class with aide, \(\beta_0\) is the
intercept term, and \(\beta_1\) and \(\beta_2\) are the regression
coefficients for \(x_{i,1}\) and \(x_{i,2}\) , respectively. In
addition, we assume that the errors \(\{\epsilon_i\}_{i=1}^{2891}\)
satisfy that \(\epsilon_1,\ldots, \epsilon_2891\) are independently and
identically distributed (i.i.d.), \(\mathbb{E}[\epsilon_i]= 0\) for
\(i=1,2,\ldots, 2891\) and \(\mathrm{var}(\epsilon_i)=\sigma^2\) for
\(i=1,2,\ldots, 2891\). The following code is almost identical to those
in the previous section.

\begin{Shaded}
\begin{Highlighting}[]
\CommentTok{# Fit a simple linear regression}
\NormalTok{fit.STAR=}\StringTok{ }\KeywordTok{lm}\NormalTok{(math2}\OperatorTok{~}\KeywordTok{as.factor}\NormalTok{(star2)}\OperatorTok{+}\DecValTok{1}\NormalTok{,}\DataTypeTok{data=}\NormalTok{dat.STAR); }
\CommentTok{# Summarize the fitted results}
\KeywordTok{summary}\NormalTok{(fit.STAR) }

\CommentTok{# We can no longer draw a fitted line here...}
\end{Highlighting}
\end{Shaded}

\section{Note}\label{note}

In the homework assignments and the midterm project, you will be asked
to reproduce the analysis on the real and sythenetic data using your own
functions. The hope is that, by reinventing the wheels, you will have a
thorough understanding of linear regression.

\chapter{Estimation}\label{ch:est}

Reading materials: Slides 12 - 22 in STA108\_LinearRegression\_S20.pdf

\begin{Shaded}
\begin{Highlighting}[]
\CommentTok{# Consider the advertising dataset}
\NormalTok{dat.advertising=}\KeywordTok{read.csv}\NormalTok{(}\StringTok{'./data/advertising.csv'}\NormalTok{);}
\CommentTok{# obtain the residuals:}
\NormalTok{x=dat.advertising}\OperatorTok{$}\NormalTok{TV;}
\NormalTok{y=dat.advertising}\OperatorTok{$}\NormalTok{sales;}
\NormalTok{beta1=}\DecValTok{1}\NormalTok{;beta0=}\FloatTok{0.2}\NormalTok{; }\CommentTok{# Just two arbitrary numbers }
\NormalTok{res=y}\OperatorTok{-}\NormalTok{x}\OperatorTok{*}\NormalTok{beta1}\OperatorTok{-}\NormalTok{beta0;}

\CommentTok{# What if we want to feed in different numbers? }

\CommentTok{# Write a function that can calculate x*beta1 + beta0}

\NormalTok{linear.model<-}\ControlFlowTok{function}\NormalTok{(beta,covariate)\{}
  \CommentTok{# beta: a 2-vector, where the first entry is the intercept}
\NormalTok{  yout=covariate}\OperatorTok{*}\NormalTok{beta[}\DecValTok{2}\NormalTok{]}\OperatorTok{+}\NormalTok{beta[}\DecValTok{1}\NormalTok{]}
  \KeywordTok{return}\NormalTok{(yout);}
  \CommentTok{# Note: this function only works with simple linear regression model}
  \CommentTok{# How would you generalize it?}
\NormalTok{\}}

\NormalTok{res.new=dat.advertising}\OperatorTok{$}\NormalTok{sales}\OperatorTok{-}\KeywordTok{linear.model}\NormalTok{(}\DataTypeTok{beta=}\KeywordTok{c}\NormalTok{(}\FloatTok{0.2}\NormalTok{,}\DecValTok{1}\NormalTok{),}\DataTypeTok{covariate=}\NormalTok{dat.advertising}\OperatorTok{$}\NormalTok{TV);}

\KeywordTok{identical}\NormalTok{(res.new,res)}
\end{Highlighting}
\end{Shaded}

\begin{Shaded}
\begin{Highlighting}[]
\CommentTok{# With the above function, we can reproduce the plot in Slide 14}
\NormalTok{beta=}\KeywordTok{c}\NormalTok{(}\DecValTok{4}\NormalTok{,}\FloatTok{0.03}\NormalTok{);}
\KeywordTok{plot}\NormalTok{(}\DataTypeTok{x=}\NormalTok{dat.advertising}\OperatorTok{$}\NormalTok{TV, }\DataTypeTok{y=}\NormalTok{dat.advertising}\OperatorTok{$}\NormalTok{sales, }\DataTypeTok{pch=}\DecValTok{16}\NormalTok{,}\DataTypeTok{cex=}\FloatTok{1.3}\NormalTok{,}\DataTypeTok{col=}\StringTok{'blue'}\NormalTok{,}\DataTypeTok{xlab=}\StringTok{'x'}\NormalTok{,}\DataTypeTok{ylab=}\StringTok{'y'}\NormalTok{)}
\KeywordTok{abline}\NormalTok{(}\DataTypeTok{a=}\NormalTok{beta[}\DecValTok{1}\NormalTok{],}\DataTypeTok{b=}\NormalTok{beta[}\DecValTok{2}\NormalTok{],}\DataTypeTok{col=}\StringTok{'red'}\NormalTok{,}\DataTypeTok{lwd=}\DecValTok{3}\NormalTok{)}

\NormalTok{yout=}\KeywordTok{linear.model}\NormalTok{(}\DataTypeTok{beta=}\NormalTok{beta,}\DataTypeTok{covariate=}\NormalTok{dat.advertising}\OperatorTok{$}\NormalTok{TV);}
\ControlFlowTok{for}\NormalTok{(i }\ControlFlowTok{in} \DecValTok{1}\OperatorTok{:}\KeywordTok{dim}\NormalTok{(dat.advertising)[}\DecValTok{1}\NormalTok{])\{}
  \KeywordTok{segments}\NormalTok{(}\DataTypeTok{x0=}\NormalTok{dat.advertising}\OperatorTok{$}\NormalTok{TV[i], }\DataTypeTok{y0=}\NormalTok{yout[i],  }\DataTypeTok{y1 =}\NormalTok{ dat.advertising}\OperatorTok{$}\NormalTok{sales[i],}\DataTypeTok{col=}\StringTok{'blue'}\NormalTok{) }
\NormalTok{\}}
\end{Highlighting}
\end{Shaded}

\begin{Shaded}
\begin{Highlighting}[]
\CommentTok{# We can see that the line is less than ideal}
\CommentTok{# We will move on to find the minimizer of the squared error loss }

\NormalTok{sum.of.squares<-}\ControlFlowTok{function}\NormalTok{(beta,covariate,outcome)\{}
\NormalTok{  yout=}\KeywordTok{linear.model}\NormalTok{(}\DataTypeTok{beta=}\NormalTok{beta,}\DataTypeTok{covariate=}\NormalTok{covariate);}
\NormalTok{  res=outcome}\OperatorTok{-}\NormalTok{yout;}
\NormalTok{  sos=}\StringTok{ }\KeywordTok{sum}\NormalTok{(res}\OperatorTok{^}\DecValTok{2}\NormalTok{);}
  \KeywordTok{return}\NormalTok{(sos)}
\NormalTok{\}}

\CommentTok{# There are many ways to minimize the sum of squares}

\CommentTok{# 1, Use generic optimizer in R}
\NormalTok{fit.optim=}\KeywordTok{optim}\NormalTok{(}\KeywordTok{c}\NormalTok{(}\DecValTok{0}\NormalTok{,}\DecValTok{0}\NormalTok{),sum.of.squares,}\DataTypeTok{covariate=}\NormalTok{dat.advertising}\OperatorTok{$}\NormalTok{TV,}\DataTypeTok{outcome=}\NormalTok{dat.advertising}\OperatorTok{$}\NormalTok{sales)}
\NormalTok{beta.hat.optim=fit.optim}\OperatorTok{$}\NormalTok{par;}
\CommentTok{# 2. Use analytic form of the optimizer}

\NormalTok{fit.linear.model<-}\ControlFlowTok{function}\NormalTok{(covariate,outcome)\{}
  \CommentTok{# I will write down the function for (multiple) linear regression here}
\NormalTok{  X=}\KeywordTok{cbind}\NormalTok{(}\DecValTok{1}\NormalTok{,covariate);}
\NormalTok{  beta.fit=}\KeywordTok{solve}\NormalTok{( }\KeywordTok{t}\NormalTok{(X)}\OperatorTok\NormalTok{X )}\OperatorTok\KeywordTok{t}\NormalTok{(X)}\OperatorTok\NormalTok{outcome;}
  \KeywordTok{return}\NormalTok{(beta.fit)}
\NormalTok{\}}

\NormalTok{beta.hat=}\KeywordTok{fit.linear.model}\NormalTok{(}\DataTypeTok{covariate=}\NormalTok{dat.advertising}\OperatorTok{$}\NormalTok{TV,}\DataTypeTok{outcome=}\NormalTok{dat.advertising}\OperatorTok{$}\NormalTok{sales)}

\CommentTok{# 3. Write your own version of Newton-Raphson method, or gradient descent }
\CommentTok{# Not required or taught in this class }

\CommentTok{# We might want to check if our solution matches that from lm()}

\NormalTok{fit.advertising=}\KeywordTok{lm}\NormalTok{(sales}\OperatorTok{~}\NormalTok{TV}\OperatorTok{+}\DecValTok{1}\NormalTok{,}\DataTypeTok{data=}\NormalTok{dat.advertising); }

\NormalTok{fit.advertising}\OperatorTok{$}\NormalTok{coef}
\NormalTok{beta.hat}
\NormalTok{beta.hat.optim}
\end{Highlighting}
\end{Shaded}

\begin{Shaded}
\begin{Highlighting}[]
\CommentTok{# Calculate the decomposition of sum of squares }
\NormalTok{residual.sum.of.squares=}\KeywordTok{sum.of.squares}\NormalTok{(}\DataTypeTok{beta=}\NormalTok{beta.hat,}\DataTypeTok{covariate=}\NormalTok{dat.advertising}\OperatorTok{$}\NormalTok{TV,}\DataTypeTok{outcome=}\NormalTok{dat.advertising}\OperatorTok{$}\NormalTok{sales)}
\NormalTok{explained.sum.of.squares=}\KeywordTok{sum}\NormalTok{((}\KeywordTok{linear.model}\NormalTok{(}\DataTypeTok{beta=}\NormalTok{beta.hat,}\DataTypeTok{covariate=}\NormalTok{dat.advertising}\OperatorTok{$}\NormalTok{TV)}\OperatorTok{-}\KeywordTok{mean}\NormalTok{(dat.advertising}\OperatorTok{$}\NormalTok{sales))}\OperatorTok{^}\DecValTok{2}\NormalTok{)}
\NormalTok{total.sum.of.squares=}\KeywordTok{sum}\NormalTok{((dat.advertising}\OperatorTok{$}\NormalTok{sales}\OperatorTok{-}\KeywordTok{mean}\NormalTok{(dat.advertising}\OperatorTok{$}\NormalTok{sales))}\OperatorTok{^}\DecValTok{2}\NormalTok{)}

\CommentTok{# Check if the equality holds..}
\NormalTok{total.sum.of.squares}
\NormalTok{explained.sum.of.squares}\OperatorTok{+}\NormalTok{residual.sum.of.squares}

\CommentTok{# But if we use identical()...}
\KeywordTok{identical}\NormalTok{(total.sum.of.squares,explained.sum.of.squares}\OperatorTok{+}\NormalTok{residual.sum.of.squares) }


\CommentTok{# Calculate the coefficient of determination}
\NormalTok{R.sq=}\StringTok{ }\NormalTok{explained.sum.of.squares}\OperatorTok{/}\NormalTok{total.sum.of.squares;}

\CommentTok{# Calculate the Pearson sample correlation }
\NormalTok{cor.hat=}\KeywordTok{cor}\NormalTok{(dat.advertising}\OperatorTok{$}\NormalTok{TV,dat.advertising}\OperatorTok{$}\NormalTok{sales)}

\CommentTok{# Verify the identity:}
\NormalTok{R.sq}
\NormalTok{cor.hat}\OperatorTok{^}\DecValTok{2}

\CommentTok{# Q: what if these equalities only hold on this one data set? What would you do to reassure yourself?}
\end{Highlighting}
\end{Shaded}

\chapter{Sampling distribution}\label{ch:sampling}

\section{Understanding sampling distribution via
simulation}\label{understanding-sampling-distribution-via-simulation}

Reading materials: Slides 23 - 32 in STA108\_LinearRegression\_S20.pdf.

In this section, we assume a true data generating model in order to draw
samples from the true popluation distribution. In particular, we assume
that \[ y_i = x_i \beta_1 + \beta_0 + \epsilon_i, \ i=1,\ldots, 50\]
where \(\epsilon_i \sim {\mathcal{U}}(-2.5, 2,5)\). We set
\(\beta_1=0.15\) and \(\beta_0=20\).

\begin{Shaded}
\begin{Highlighting}[]
\NormalTok{### We will simulate synthetic data to understand the concept of sampling distribution}

\NormalTok{### To match our setup, we will generate the covarates and keep them fixed in the remainder of the simulation. }
\KeywordTok{set.seed}\NormalTok{(}\DecValTok{1}\NormalTok{)}
\NormalTok{n=}\DecValTok{50}\NormalTok{;}
\NormalTok{x=}\KeywordTok{as.matrix}\NormalTok{(}\KeywordTok{rnorm}\NormalTok{(n,}\DataTypeTok{mean=}\DecValTok{10}\NormalTok{,}\DataTypeTok{sd=}\DecValTok{2}\NormalTok{),}\DataTypeTok{ncol=}\NormalTok{n);}
\NormalTok{beta.true=}\KeywordTok{c}\NormalTok{(}\DecValTok{20}\NormalTok{,}\FloatTok{0.15}\NormalTok{)}
\NormalTok{Ey=}\StringTok{ }\NormalTok{x}\OperatorTok{*}\NormalTok{beta.true[}\DecValTok{2}\NormalTok{]}\OperatorTok{+}\NormalTok{beta.true[}\DecValTok{1}\NormalTok{];}

\NormalTok{error.terms=}\StringTok{ }\NormalTok{(}\KeywordTok{runif}\NormalTok{(n)}\OperatorTok{-}\FloatTok{0.5}\NormalTok{)}\OperatorTok{*}\DecValTok{5}\NormalTok{;}
\NormalTok{y=Ey}\OperatorTok{+}\NormalTok{error.terms;}
\NormalTok{beta.hat=}\KeywordTok{fit.linear.model}\NormalTok{(}\DataTypeTok{covariate=}\NormalTok{x,}\DataTypeTok{outcome=}\NormalTok{y);}


\KeywordTok{plot}\NormalTok{(}\DataTypeTok{x=}\NormalTok{x,}\DataTypeTok{y=}\NormalTok{y,}\DataTypeTok{pch=}\DecValTok{16}\NormalTok{,}\DataTypeTok{cex=}\DecValTok{2}\NormalTok{)}
\KeywordTok{abline}\NormalTok{(}\DataTypeTok{b=}\NormalTok{beta.hat[}\DecValTok{2}\NormalTok{],}\DataTypeTok{a=}\NormalTok{beta.hat[}\DecValTok{1}\NormalTok{],}\DataTypeTok{col=}\StringTok{'red'}\NormalTok{,}\DataTypeTok{lwd=}\DecValTok{2}\NormalTok{)}
\end{Highlighting}
\end{Shaded}

\begin{Shaded}
\begin{Highlighting}[]
\NormalTok{### We can run the simulation by putting the above lines into a large for-loop}
\NormalTok{### We will wrap them up into one function}

\NormalTok{simulate.one.instance<-}\ControlFlowTok{function}\NormalTok{(x,beta.true)\{}
\NormalTok{  n=}\KeywordTok{length}\NormalTok{(x);}
\NormalTok{  Ey=}\StringTok{ }\NormalTok{x}\OperatorTok{*}\NormalTok{beta.true[}\DecValTok{2}\NormalTok{]}\OperatorTok{+}\NormalTok{beta.true[}\DecValTok{1}\NormalTok{];}
\NormalTok{  error.terms=}\StringTok{ }\NormalTok{(}\KeywordTok{runif}\NormalTok{(n)}\OperatorTok{-}\FloatTok{0.5}\NormalTok{)}\OperatorTok{*}\DecValTok{5}\NormalTok{;}
\NormalTok{  y=Ey}\OperatorTok{+}\NormalTok{error.terms;}
\NormalTok{  beta.hat=}\KeywordTok{fit.linear.model}\NormalTok{(}\DataTypeTok{covariate=}\NormalTok{x,}\DataTypeTok{outcome=}\NormalTok{y);}
  \KeywordTok{return}\NormalTok{(beta.hat)}
\NormalTok{\}}

\NormalTok{### Set the number of replicates to be 10000}
\NormalTok{N.sim=}\FloatTok{1e4}\NormalTok{;}
\KeywordTok{set.seed}\NormalTok{(}\DecValTok{1}\NormalTok{)}
\NormalTok{beta.sim=}\KeywordTok{replicate}\NormalTok{(N.sim,}\KeywordTok{simulate.one.instance}\NormalTok{(x,beta.true));}


\CommentTok{# Draw the histogram for the estimated intercept }
\KeywordTok{hist}\NormalTok{(beta.sim[}\DecValTok{1}\NormalTok{,}\DecValTok{1}\NormalTok{,],}\DataTypeTok{xlab=}\StringTok{'Fitted intercept'}\NormalTok{,}\DataTypeTok{main=}\StringTok{''}\NormalTok{)}
\KeywordTok{abline}\NormalTok{(}\DataTypeTok{v=}\NormalTok{beta.true[}\DecValTok{1}\NormalTok{],}\DataTypeTok{lwd=}\DecValTok{3}\NormalTok{,}\DataTypeTok{col=}\StringTok{'red'}\NormalTok{)}
\end{Highlighting}
\end{Shaded}

Recall from the lecture notes that the variance of \(\hat{\beta}_1\) is
\[{\rm var}\big(\hat{\beta}_1\big)=\frac{1}{\sum_{i=1}^n (x_i-\bar{x})^2} \sigma^2.\]

\begin{Shaded}
\begin{Highlighting}[]
\NormalTok{## Using the simulation results, we can verify the statement on Slide 26}

\NormalTok{### Expectation }
\KeywordTok{mean}\NormalTok{(beta.sim[}\DecValTok{1}\NormalTok{,}\DecValTok{1}\NormalTok{,])}\OperatorTok{-}\NormalTok{beta.true[}\DecValTok{1}\NormalTok{] }\CommentTok{# intercept }
\KeywordTok{mean}\NormalTok{(beta.sim[}\DecValTok{2}\NormalTok{,}\DecValTok{1}\NormalTok{,])}\OperatorTok{-}\NormalTok{beta.true[}\DecValTok{2}\NormalTok{] }\CommentTok{# slope }

\NormalTok{### Variance }
\NormalTok{sigma.sq =}\StringTok{ }\DecValTok{5}\OperatorTok{^}\DecValTok{2}\OperatorTok{/}\DecValTok{12}\NormalTok{; }\CommentTok{# variance of the error: (runif(n)-0.5)*5;}

\CommentTok{# slope:}
\NormalTok{sigma.sq}\OperatorTok{/}\KeywordTok{sum}\NormalTok{(  (x}\OperatorTok{-}\KeywordTok{mean}\NormalTok{(x))}\OperatorTok{^}\DecValTok{2}\NormalTok{  ) }\CommentTok{# theoretical value }
\KeywordTok{var}\NormalTok{(beta.sim[}\DecValTok{2}\NormalTok{,}\DecValTok{1}\NormalTok{,]) }\CommentTok{# variance from the simulation }


\CommentTok{# intercept:}
\NormalTok{sigma.sq}\OperatorTok{*}\KeywordTok{sum}\NormalTok{(x}\OperatorTok{^}\DecValTok{2}\NormalTok{)}\OperatorTok{/}\KeywordTok{sum}\NormalTok{(  (x}\OperatorTok{-}\KeywordTok{mean}\NormalTok{(x))}\OperatorTok{^}\DecValTok{2}\NormalTok{  )}\OperatorTok{/}\NormalTok{n }\CommentTok{# theoretical value }
\KeywordTok{var}\NormalTok{(beta.sim[}\DecValTok{1}\NormalTok{,}\DecValTok{1}\NormalTok{,]) }\CommentTok{# variance from the simulation }
\end{Highlighting}
\end{Shaded}

Q: How would you verify the Gauss-Markove theorem in this simulation?

\begin{Shaded}
\begin{Highlighting}[]
\CommentTok{# A very simple simulation to compare the LSE with another linear unbiased estimator: }
\NormalTok{simulate.one.instance<-}\ControlFlowTok{function}\NormalTok{(x,beta.true)\{}
\NormalTok{  n=}\KeywordTok{length}\NormalTok{(x);}
\NormalTok{  Ey=}\StringTok{ }\NormalTok{x}\OperatorTok{*}\NormalTok{beta.true[}\DecValTok{2}\NormalTok{]}\OperatorTok{+}\NormalTok{beta.true[}\DecValTok{1}\NormalTok{];}
\NormalTok{  error.terms=}\StringTok{ }\NormalTok{(}\KeywordTok{runif}\NormalTok{(n)}\OperatorTok{-}\FloatTok{0.5}\NormalTok{)}\OperatorTok{*}\DecValTok{5}\NormalTok{;}
\NormalTok{  y=Ey}\OperatorTok{+}\NormalTok{error.terms;}
\NormalTok{  beta.hat=}\KeywordTok{fit.linear.model}\NormalTok{(}\DataTypeTok{covariate=}\NormalTok{x,}\DataTypeTok{outcome=}\NormalTok{y);}
\NormalTok{  slope.hat=beta.hat[}\DecValTok{2}\NormalTok{];}
\NormalTok{  slope.dot=}\StringTok{ }\NormalTok{(y[n]}\OperatorTok{-}\NormalTok{y[}\DecValTok{1}\NormalTok{])}\OperatorTok{/}\NormalTok{(x[n]}\OperatorTok{-}\NormalTok{x[}\DecValTok{1}\NormalTok{])}
  \KeywordTok{return}\NormalTok{(}\KeywordTok{c}\NormalTok{(slope.hat,slope.dot))}
\NormalTok{\}}

\NormalTok{### Set the number of replicates to be 10000}
\NormalTok{N.sim=}\FloatTok{1e4}\NormalTok{;}
\KeywordTok{set.seed}\NormalTok{(}\DecValTok{1}\NormalTok{)}
\NormalTok{slope.sim=}\KeywordTok{replicate}\NormalTok{(N.sim,}\KeywordTok{simulate.one.instance}\NormalTok{(x,beta.true));}

\KeywordTok{apply}\NormalTok{(slope.sim,}\DecValTok{1}\NormalTok{,mean) }\CommentTok{# both are unbiased}
\KeywordTok{apply}\NormalTok{(slope.sim,}\DecValTok{1}\NormalTok{,var) }\CommentTok{# variance of LSE is smaller}
\end{Highlighting}
\end{Shaded}

\begin{Shaded}
\begin{Highlighting}[]
\NormalTok{## Estimate the variance of errors using the residuals }
\KeywordTok{set.seed}\NormalTok{(}\DecValTok{1}\NormalTok{)}
\NormalTok{error.terms=}\StringTok{ }\NormalTok{(}\KeywordTok{runif}\NormalTok{(n)}\OperatorTok{-}\FloatTok{0.5}\NormalTok{)}\OperatorTok{*}\DecValTok{5}\NormalTok{;}
\NormalTok{y=Ey}\OperatorTok{+}\NormalTok{error.terms;}
\NormalTok{beta.hat=}\KeywordTok{fit.linear.model}\NormalTok{(}\DataTypeTok{covariate=}\NormalTok{x,}\DataTypeTok{outcome=}\NormalTok{y);}

\NormalTok{residual.sum.of.squares=}\KeywordTok{sum.of.squares}\NormalTok{(}\DataTypeTok{beta=}\NormalTok{beta.hat,}\DataTypeTok{covariate=}\NormalTok{x,}\DataTypeTok{outcome=}\NormalTok{y)}

\NormalTok{sigma.sq.hat=residual.sum.of.squares}\OperatorTok{/}\NormalTok{(n}\OperatorTok{-}\DecValTok{2}\NormalTok{) }\CommentTok{# estimates from one instance}

\NormalTok{## Run a simulation to verify the claim}

\NormalTok{## }\AlertTok{NOTE}\NormalTok{: we will use the same function name}
\NormalTok{simulate.one.instance<-}\ControlFlowTok{function}\NormalTok{(x,beta.true)\{}
\NormalTok{  n=}\KeywordTok{length}\NormalTok{(x);}
\NormalTok{  Ey=}\StringTok{ }\NormalTok{x}\OperatorTok{*}\NormalTok{beta.true[}\DecValTok{2}\NormalTok{]}\OperatorTok{+}\NormalTok{beta.true[}\DecValTok{1}\NormalTok{];}
\NormalTok{  error.terms=}\StringTok{ }\NormalTok{(}\KeywordTok{runif}\NormalTok{(n)}\OperatorTok{-}\FloatTok{0.5}\NormalTok{)}\OperatorTok{*}\DecValTok{5}\NormalTok{;}
\NormalTok{  y=Ey}\OperatorTok{+}\NormalTok{error.terms;}
\NormalTok{  beta.hat=}\KeywordTok{fit.linear.model}\NormalTok{(}\DataTypeTok{covariate=}\NormalTok{x,}\DataTypeTok{outcome=}\NormalTok{y);}
\NormalTok{  residual.sum.of.squares=}\KeywordTok{sum.of.squares}\NormalTok{(}\DataTypeTok{beta=}\NormalTok{beta.hat,}\DataTypeTok{covariate=}\NormalTok{x,}\DataTypeTok{outcome=}\NormalTok{y)}

\NormalTok{sigma.sq.hat=residual.sum.of.squares}\OperatorTok{/}\NormalTok{(n}\OperatorTok{-}\DecValTok{2}\NormalTok{) }\CommentTok{# estimates from one instance}

  \KeywordTok{return}\NormalTok{(sigma.sq.hat)}
\NormalTok{\}}

\NormalTok{### Set the number of replicates to be 10000}
\NormalTok{N.sim=}\FloatTok{1e4}\NormalTok{;}
\KeywordTok{set.seed}\NormalTok{(}\DecValTok{1}\NormalTok{)}
\NormalTok{sigma.sq.hat.sim=}\KeywordTok{replicate}\NormalTok{(N.sim,}\KeywordTok{simulate.one.instance}\NormalTok{(x,beta.true));}

\KeywordTok{mean}\NormalTok{(sigma.sq.hat.sim)}
\NormalTok{sigma.sq}
\end{Highlighting}
\end{Shaded}

\begin{Shaded}
\begin{Highlighting}[]
\NormalTok{## To wrap this up}
\NormalTok{## For any new fits, we can estiamte the variance and standad errors of the estimators }
\KeywordTok{set.seed}\NormalTok{(}\DecValTok{1}\NormalTok{)}
\NormalTok{error.terms=}\StringTok{ }\NormalTok{(}\KeywordTok{runif}\NormalTok{(n)}\OperatorTok{-}\FloatTok{0.5}\NormalTok{)}\OperatorTok{*}\DecValTok{5}\NormalTok{;}
\NormalTok{y=Ey}\OperatorTok{+}\NormalTok{error.terms;}
\NormalTok{beta.hat=}\KeywordTok{fit.linear.model}\NormalTok{(}\DataTypeTok{covariate=}\NormalTok{x,}\DataTypeTok{outcome=}\NormalTok{y);}

\NormalTok{estimate.sigma.sq<-}\ControlFlowTok{function}\NormalTok{(beta,covariate,outcome)\{}
\NormalTok{   residual.sum.of.squares=}\KeywordTok{sum.of.squares}\NormalTok{(}\DataTypeTok{beta=}\NormalTok{beta,}\DataTypeTok{covariate=}\NormalTok{covariate,}\DataTypeTok{outcome=}\NormalTok{outcome)}
\NormalTok{   n=}\KeywordTok{length}\NormalTok{(outcome)}
\NormalTok{   sigma.sq.hat=residual.sum.of.squares}\OperatorTok{/}\NormalTok{(n}\OperatorTok{-}\DecValTok{2}\NormalTok{) }
   \KeywordTok{return}\NormalTok{(sigma.sq.hat)}
\NormalTok{\}}

\NormalTok{estimate.coef.var<-}\ControlFlowTok{function}\NormalTok{(beta,covariate,outcome)\{}
\NormalTok{  sigma.sq.hat=}\KeywordTok{estimate.sigma.sq}\NormalTok{(beta,covariate,outcome)}
\NormalTok{  var.hat.beta=beta;}
\NormalTok{  var.hat.beta[}\DecValTok{2}\NormalTok{]=sigma.sq.hat}\OperatorTok{/}\KeywordTok{sum}\NormalTok{(  (covariate}\OperatorTok{-}\KeywordTok{mean}\NormalTok{(covariate))}\OperatorTok{^}\DecValTok{2}\NormalTok{  ) }
\NormalTok{  n=}\KeywordTok{length}\NormalTok{(outcome)}
\NormalTok{  var.hat.beta[}\DecValTok{1}\NormalTok{]=sigma.sq.hat}\OperatorTok{*}\KeywordTok{sum}\NormalTok{(covariate}\OperatorTok{^}\DecValTok{2}\NormalTok{)}\OperatorTok{/}\KeywordTok{sum}\NormalTok{((covariate}\OperatorTok{-}\KeywordTok{mean}\NormalTok{(covariate))}\OperatorTok{^}\DecValTok{2}\NormalTok{  )}\OperatorTok{/}\NormalTok{n}
  \KeywordTok{return}\NormalTok{( var.hat.beta)}
\NormalTok{\}}

\NormalTok{estimate.coef.sd<-}\ControlFlowTok{function}\NormalTok{(beta,covariate,outcome)\{}
\NormalTok{  var.hat.beta=}\KeywordTok{estimate.coef.var}\NormalTok{(beta,covariate,outcome)}
\NormalTok{ sd.hat.beta=}\KeywordTok{sqrt}\NormalTok{(var.hat.beta);}
  \KeywordTok{return}\NormalTok{(sd.hat.beta)}
\NormalTok{\}}
\KeywordTok{estimate.coef.sd}\NormalTok{(}\DataTypeTok{beta=}\NormalTok{beta.hat,}\DataTypeTok{covariate=}\NormalTok{x,}\DataTypeTok{outcome=}\NormalTok{y)}

\NormalTok{## Compare your numbers with the output from lm()}
\KeywordTok{summary}\NormalTok{(}\KeywordTok{lm}\NormalTok{(y}\OperatorTok{~}\NormalTok{x}\OperatorTok{+}\DecValTok{1}\NormalTok{))}
\end{Highlighting}
\end{Shaded}

\section{Shapes of sampling
distributions}\label{shapes-of-sampling-distributions}

Reading materials: Slide 33 in STA108\_LinearRegression\_S20.pdf.

\begin{Shaded}
\begin{Highlighting}[]
\CommentTok{# There can be multiple distributions with the same mean and variance. Consider three distributions discussed in the lecture notes.}

\NormalTok{xgrid =}\StringTok{ }\KeywordTok{seq}\NormalTok{(}\DataTypeTok{from =} \OperatorTok{-}\DecValTok{5}\NormalTok{, }\DataTypeTok{to =} \DecValTok{5}\NormalTok{, }\DataTypeTok{by=}\FloatTok{0.01}\NormalTok{);}
\NormalTok{norm_density=}\KeywordTok{dnorm}\NormalTok{(xgrid,}\DataTypeTok{mean=}\DecValTok{0}\NormalTok{,}\DataTypeTok{sd=}\KeywordTok{sqrt}\NormalTok{(}\DecValTok{2}\NormalTok{));}
\NormalTok{unif_density=}\KeywordTok{dunif}\NormalTok{(xgrid,}\DataTypeTok{min=}\OperatorTok{-}\KeywordTok{sqrt}\NormalTok{(}\DecValTok{6}\NormalTok{),}\DataTypeTok{max=}\KeywordTok{sqrt}\NormalTok{(}\DecValTok{6}\NormalTok{));}

\NormalTok{xbin=}\KeywordTok{c}\NormalTok{(}\OperatorTok{-}\KeywordTok{sqrt}\NormalTok{(}\DecValTok{2}\NormalTok{),}\KeywordTok{sqrt}\NormalTok{(}\DecValTok{2}\NormalTok{));}
\NormalTok{ybin=}\KeywordTok{c}\NormalTok{(}\FloatTok{0.5}\NormalTok{,}\FloatTok{0.5}\NormalTok{);}


\KeywordTok{plot}\NormalTok{(norm_density}\OperatorTok{~}\NormalTok{xgrid,}\DataTypeTok{type=}\StringTok{"l"}\NormalTok{,}\DataTypeTok{col=}\StringTok{'red'}\NormalTok{,}\DataTypeTok{lwd=}\DecValTok{3}\NormalTok{,}\DataTypeTok{ylim=}\KeywordTok{c}\NormalTok{(}\DecValTok{0}\NormalTok{,}\FloatTok{0.52}\NormalTok{),}\DataTypeTok{xlab=}\StringTok{"z"}\NormalTok{,}\DataTypeTok{ylab=}\StringTok{"density"}\NormalTok{)}
\KeywordTok{lines}\NormalTok{(unif_density}\OperatorTok{~}\NormalTok{xgrid,}\DataTypeTok{col=}\StringTok{'black'}\NormalTok{,}\DataTypeTok{lwd=}\DecValTok{3}\NormalTok{)}

\KeywordTok{segments}\NormalTok{(xbin[}\DecValTok{1}\NormalTok{],}\DecValTok{0}\NormalTok{,xbin[}\DecValTok{1}\NormalTok{],ybin[}\DecValTok{1}\NormalTok{],}\DataTypeTok{col=}\StringTok{'green'}\NormalTok{,}\DataTypeTok{lwd=}\DecValTok{10}\NormalTok{)}
\KeywordTok{segments}\NormalTok{(xbin[}\DecValTok{2}\NormalTok{],}\DecValTok{0}\NormalTok{,xbin[}\DecValTok{2}\NormalTok{],ybin[}\DecValTok{2}\NormalTok{],}\DataTypeTok{col=}\StringTok{'green'}\NormalTok{,}\DataTypeTok{lwd=}\DecValTok{10}\NormalTok{)}
\KeywordTok{legend}\NormalTok{(}\DataTypeTok{x=}\FloatTok{2.5}\NormalTok{,}\DataTypeTok{y=}\FloatTok{0.5}\NormalTok{,}\DataTypeTok{legend=}\KeywordTok{c}\NormalTok{(}\StringTok{"Uniform"}\NormalTok{,}\StringTok{"Normal"}\NormalTok{,}\StringTok{"Binary"}\NormalTok{),}\DataTypeTok{lwd=}\DecValTok{2}\NormalTok{,}\DataTypeTok{col=}\KeywordTok{c}\NormalTok{(}\StringTok{'black'}\NormalTok{,}\StringTok{'red'}\NormalTok{,}\StringTok{'green'}\NormalTok{))}
\end{Highlighting}
\end{Shaded}

\subsection{Asymptotic distribution}\label{asymptotic-distribution}

Reading materials: Slides 34 - 48 in STA108\_LinearRegression\_S20.pdf.

The L-F central limit theorem claims that the asymptotic distribution
for \(\frac{\hat{\beta}_1 - \beta_1}{\hat{\sigma} / S_{xx}^{1/2} }\) is
\(\mathcal{N}(0,1)\).

\begin{Shaded}
\begin{Highlighting}[]
\NormalTok{## Edit the simulation code }
\NormalTok{### This time with large n (n=500)}
\NormalTok{### Three different versions of errors}
\NormalTok{### Unequal variances  }
\NormalTok{simulate.one.instance<-}\ControlFlowTok{function}\NormalTok{(x,beta.true,error.type)\{}
\NormalTok{  n=}\KeywordTok{length}\NormalTok{(x);}
\NormalTok{  Ey=}\StringTok{ }\NormalTok{x}\OperatorTok{*}\NormalTok{beta.true[}\DecValTok{2}\NormalTok{]}\OperatorTok{+}\NormalTok{beta.true[}\DecValTok{1}\NormalTok{];}
\NormalTok{  vars=}\FloatTok{0.1}\OperatorTok{+}\KeywordTok{runif}\NormalTok{(n)}\OperatorTok{*}\DecValTok{2}\NormalTok{; }\CommentTok{# unequal variance }
  \ControlFlowTok{if}\NormalTok{(error.type}\OperatorTok{==}\StringTok{'uniform'}\NormalTok{)\{}
\NormalTok{    error.terms=}\StringTok{ }\NormalTok{(}\KeywordTok{runif}\NormalTok{(n)}\OperatorTok{-}\FloatTok{0.5}\NormalTok{)}\OperatorTok{*}\KeywordTok{sqrt}\NormalTok{(vars);}
    
\NormalTok{  \}}\ControlFlowTok{else} \ControlFlowTok{if}\NormalTok{(error.type}\OperatorTok{==}\StringTok{'Bernoulli'}\NormalTok{)\{}
\NormalTok{    error.terms=(}\KeywordTok{rbinom}\NormalTok{(n,}\DataTypeTok{size=}\DecValTok{1}\NormalTok{,}\DataTypeTok{prob=}\FloatTok{0.5}\NormalTok{)}\OperatorTok{-}\FloatTok{0.5}\NormalTok{)}\OperatorTok{*}\KeywordTok{sqrt}\NormalTok{(vars);}
\NormalTok{  \}}\ControlFlowTok{else}\NormalTok{\{}
\NormalTok{    error.terms=}\StringTok{ }\KeywordTok{rnorm}\NormalTok{(n)}\OperatorTok{*}\KeywordTok{sqrt}\NormalTok{(vars);}
\NormalTok{  \}}
\NormalTok{  y=Ey}\OperatorTok{+}\NormalTok{error.terms;}
\NormalTok{  beta.hat=}\KeywordTok{fit.linear.model}\NormalTok{(}\DataTypeTok{covariate=}\NormalTok{x,}\DataTypeTok{outcome=}\NormalTok{y);}
  
\NormalTok{  residual.sum.of.squares=}\KeywordTok{sum.of.squares}\NormalTok{(}\DataTypeTok{beta=}\NormalTok{beta.hat,}\DataTypeTok{covariate=}\NormalTok{x,}\DataTypeTok{outcome=}\NormalTok{y)}

\NormalTok{sigma.sq.hat=residual.sum.of.squares}\OperatorTok{/}\NormalTok{(n}\OperatorTok{-}\DecValTok{2}\NormalTok{) }\CommentTok{# estimates from one instance}

\NormalTok{  beta.hat.std =(beta.hat[}\DecValTok{2}\NormalTok{]}\OperatorTok{-}\NormalTok{beta.true[}\DecValTok{2}\NormalTok{])}\OperatorTok{/}\KeywordTok{sqrt}\NormalTok{(sigma.sq.hat)}\OperatorTok{*}\KeywordTok{sqrt}\NormalTok{( }\KeywordTok{sum}\NormalTok{( (x}\OperatorTok{-}\KeywordTok{mean}\NormalTok{(x))}\OperatorTok{^}\DecValTok{2}\NormalTok{ ) );  }\CommentTok{# standardized intercept}
  
  \KeywordTok{return}\NormalTok{(beta.hat.std)}
\NormalTok{\}}

\NormalTok{### Set the number of replicates to be 10000}
\NormalTok{N.sim=}\FloatTok{1e4}\NormalTok{;}
\KeywordTok{set.seed}\NormalTok{(}\DecValTok{1}\NormalTok{)}

\NormalTok{n=}\DecValTok{500}\NormalTok{;}
\NormalTok{x=}\KeywordTok{as.matrix}\NormalTok{(}\KeywordTok{rnorm}\NormalTok{(n,}\DataTypeTok{mean=}\DecValTok{10}\NormalTok{,}\DataTypeTok{sd=}\DecValTok{2}\NormalTok{),}\DataTypeTok{ncol=}\NormalTok{n);}
\NormalTok{beta.hat.std.uniform=}\KeywordTok{replicate}\NormalTok{(N.sim,}\KeywordTok{simulate.one.instance}\NormalTok{(x,beta.true,}\DataTypeTok{error.type=}\StringTok{'uniform'}\NormalTok{));}
\NormalTok{beta.hat.std.Bernoulli=}\KeywordTok{replicate}\NormalTok{(N.sim,}\KeywordTok{simulate.one.instance}\NormalTok{(x,beta.true,}\DataTypeTok{error.type=}\StringTok{'Bernoulli'}\NormalTok{));}
\NormalTok{beta.hat.std.normal=}\KeywordTok{replicate}\NormalTok{(N.sim,}\KeywordTok{simulate.one.instance}\NormalTok{(x,beta.true,}\DataTypeTok{error.type=}\StringTok{'normal'}\NormalTok{));}
\end{Highlighting}
\end{Shaded}

\begin{Shaded}
\begin{Highlighting}[]
\CommentTok{# Draw the density plot}

\NormalTok{density.uniform=}\KeywordTok{density}\NormalTok{( beta.hat.std.uniform);}
\NormalTok{density.Bernoulli=}\KeywordTok{density}\NormalTok{( beta.hat.std.Bernoulli);}
\NormalTok{density.normal=}\KeywordTok{density}\NormalTok{( beta.hat.std.normal);}

\CommentTok{# Theoretical density:}
\NormalTok{xgrid =}\StringTok{ }\KeywordTok{seq}\NormalTok{(}\DataTypeTok{from=}\OperatorTok{-}\DecValTok{4}\NormalTok{,}\DataTypeTok{to=}\DecValTok{4}\NormalTok{,}\DataTypeTok{length.out=}\DecValTok{100}\NormalTok{);}
\NormalTok{normal.pdf=}\StringTok{ }\KeywordTok{dnorm}\NormalTok{(xgrid)}

\KeywordTok{plot}\NormalTok{(density.uniform,}\DataTypeTok{xlab=}\StringTok{"Standardized intercept"}\NormalTok{,}\DataTypeTok{col=}\StringTok{'red'}\NormalTok{,}\DataTypeTok{lwd=}\DecValTok{3}\NormalTok{,}\DataTypeTok{ylim=}\KeywordTok{c}\NormalTok{(}\DecValTok{0}\NormalTok{,}\FloatTok{0.5}\NormalTok{),}\DataTypeTok{main=}\StringTok{''}\NormalTok{,}\DataTypeTok{lty=}\DecValTok{2}\NormalTok{)}
\KeywordTok{lines}\NormalTok{(density.Bernoulli,}\DataTypeTok{col=}\StringTok{'green'}\NormalTok{,}\DataTypeTok{lwd=}\DecValTok{3}\NormalTok{,}\DataTypeTok{lty=}\DecValTok{2}\NormalTok{)}
\KeywordTok{lines}\NormalTok{(density.normal,}\DataTypeTok{col=}\StringTok{'blue'}\NormalTok{,}\DataTypeTok{lwd=}\DecValTok{3}\NormalTok{,}\DataTypeTok{lty=}\DecValTok{2}\NormalTok{)}
\KeywordTok{lines}\NormalTok{(normal.pdf}\OperatorTok{~}\NormalTok{xgrid,}\DataTypeTok{lwd=}\DecValTok{2}\NormalTok{,}\DataTypeTok{lty=}\DecValTok{3}\NormalTok{)}
\KeywordTok{legend}\NormalTok{(}\DataTypeTok{x=}\DecValTok{2}\NormalTok{,}\DataTypeTok{y=}\FloatTok{0.46}\NormalTok{,}\DataTypeTok{legend=}\KeywordTok{c}\NormalTok{(}\StringTok{"Uniform errors"}\NormalTok{,}\StringTok{"Bernoulli errors"}\NormalTok{,}\StringTok{"Normal errors"}\NormalTok{, }\StringTok{"N(0,1)"}\NormalTok{),}\DataTypeTok{lwd=}\DecValTok{3}\NormalTok{,}\DataTypeTok{col=}\KeywordTok{c}\NormalTok{(}\StringTok{'red'}\NormalTok{,}\StringTok{'green'}\NormalTok{,}\StringTok{'blue'}\NormalTok{,}\StringTok{'black'}\NormalTok{),}\DataTypeTok{lty=}\KeywordTok{c}\NormalTok{(}\DecValTok{2}\NormalTok{,}\DecValTok{2}\NormalTok{,}\DecValTok{2}\NormalTok{,}\DecValTok{3}\NormalTok{))}
\end{Highlighting}
\end{Shaded}

\begin{Shaded}
\begin{Highlighting}[]
\NormalTok{## The L-F CLT allows us to write down the asymptotic distribution in closed-form}
\NormalTok{## The bootstrap method is an alternative when analytic solution is not available }
\NormalTok{## Note: Bootstrap is not that useful for linear regression, but has proven to be a powerful tool for other methods. }

\KeywordTok{set.seed}\NormalTok{(}\DecValTok{1}\NormalTok{)}
\CommentTok{# Generate one set of data: }
\NormalTok{n=}\DecValTok{500}\NormalTok{;}
\NormalTok{x=}\KeywordTok{as.matrix}\NormalTok{(}\KeywordTok{rnorm}\NormalTok{(n,}\DataTypeTok{mean=}\DecValTok{10}\NormalTok{,}\DataTypeTok{sd=}\DecValTok{2}\NormalTok{),}\DataTypeTok{ncol=}\NormalTok{n);}
\NormalTok{Ey=}\StringTok{ }\NormalTok{x}\OperatorTok{*}\NormalTok{beta.true[}\DecValTok{2}\NormalTok{]}\OperatorTok{+}\NormalTok{beta.true[}\DecValTok{1}\NormalTok{];}
\NormalTok{vars=}\FloatTok{0.1}\OperatorTok{+}\KeywordTok{runif}\NormalTok{(n)}\OperatorTok{*}\DecValTok{2}\NormalTok{; }\CommentTok{# unequal variance }
\NormalTok{error.terms=}\StringTok{ }\NormalTok{(}\KeywordTok{runif}\NormalTok{(n)}\OperatorTok{-}\FloatTok{0.5}\NormalTok{)}\OperatorTok{*}\KeywordTok{sqrt}\NormalTok{(vars);}
\NormalTok{y=Ey}\OperatorTok{+}\NormalTok{error.terms;}
\NormalTok{beta.hat=}\KeywordTok{fit.linear.model}\NormalTok{(}\DataTypeTok{covariate=}\NormalTok{x,}\DataTypeTok{outcome=}\NormalTok{y);}


\CommentTok{# Now write a function for bootstrap }
\NormalTok{boot.one.instance<-}\ControlFlowTok{function}\NormalTok{(covariate,outcome)\{}
\NormalTok{  n=}\KeywordTok{length}\NormalTok{(outcome);}
\NormalTok{  sample_indices =}\StringTok{ }\KeywordTok{sample}\NormalTok{(}\DecValTok{1}\OperatorTok{:}\NormalTok{n,n,}\DataTypeTok{replace=}\OtherTok{TRUE}\NormalTok{) }\CommentTok{# sampling with replacement}
\NormalTok{  covariate.boot=}\StringTok{ }\NormalTok{covariate[sample_indices]; outcome.boot=}\StringTok{ }\NormalTok{outcome[sample_indices];}
   
\NormalTok{  beta.hat=}\KeywordTok{fit.linear.model}\NormalTok{(}\DataTypeTok{covariate=}\NormalTok{covariate.boot,}\DataTypeTok{outcome=}\NormalTok{outcome.boot);}
  \KeywordTok{return}\NormalTok{(beta.hat[}\DecValTok{2}\NormalTok{]) }
\NormalTok{\}}

\KeywordTok{set.seed}\NormalTok{(}\DecValTok{1}\NormalTok{)}

\NormalTok{B=}\FloatTok{1e5}\NormalTok{;}
\NormalTok{beta.hat.boot=}\KeywordTok{replicate}\NormalTok{(B,}\KeywordTok{boot.one.instance}\NormalTok{(}\DataTypeTok{covariate=}\NormalTok{x,}\DataTypeTok{outcome=}\NormalTok{y));}
\end{Highlighting}
\end{Shaded}

\begin{Shaded}
\begin{Highlighting}[]
\NormalTok{## Compare the bootstrap distribution of the standardized slope against that from the CLT }


\NormalTok{density.boot=}\KeywordTok{density}\NormalTok{((beta.hat.boot}\OperatorTok{-}\KeywordTok{mean}\NormalTok{(beta.hat.boot))}\OperatorTok{/}\KeywordTok{sd}\NormalTok{(beta.hat.boot));}
\CommentTok{# Theoretical density (from CLT)}
\NormalTok{xgrid =}\StringTok{ }\KeywordTok{seq}\NormalTok{(}\DataTypeTok{from=}\OperatorTok{-}\DecValTok{4}\NormalTok{,}\DataTypeTok{to=}\DecValTok{4}\NormalTok{,}\DataTypeTok{length.out=}\DecValTok{100}\NormalTok{);}
\NormalTok{normal.pdf=}\StringTok{ }\KeywordTok{dnorm}\NormalTok{(xgrid)}

\KeywordTok{plot}\NormalTok{(density.boot,}\DataTypeTok{xlab=}\StringTok{"Standardized intercept"}\NormalTok{,}\DataTypeTok{col=}\StringTok{'red'}\NormalTok{,}\DataTypeTok{lwd=}\DecValTok{3}\NormalTok{,}\DataTypeTok{ylim=}\KeywordTok{c}\NormalTok{(}\DecValTok{0}\NormalTok{,}\FloatTok{0.5}\NormalTok{),}\DataTypeTok{main=}\StringTok{''}\NormalTok{,}\DataTypeTok{lty=}\DecValTok{2}\NormalTok{)}
\KeywordTok{lines}\NormalTok{(normal.pdf}\OperatorTok{~}\NormalTok{xgrid,}\DataTypeTok{lwd=}\DecValTok{2}\NormalTok{,}\DataTypeTok{lty=}\DecValTok{3}\NormalTok{)}
\KeywordTok{legend}\NormalTok{(}\DataTypeTok{x=}\DecValTok{2}\NormalTok{,}\DataTypeTok{y=}\FloatTok{0.46}\NormalTok{,}\DataTypeTok{legend=}\KeywordTok{c}\NormalTok{(}\StringTok{"Bootstrap"}\NormalTok{, }\StringTok{"N(0,1)"}\NormalTok{),}\DataTypeTok{lwd=}\DecValTok{3}\NormalTok{,}\DataTypeTok{col=}\KeywordTok{c}\NormalTok{(}\StringTok{'red'}\NormalTok{,}\StringTok{'green'}\NormalTok{,}\StringTok{'blue'}\NormalTok{,}\StringTok{'black'}\NormalTok{),}\DataTypeTok{lty=}\KeywordTok{c}\NormalTok{(}\DecValTok{2}\NormalTok{,}\DecValTok{3}\NormalTok{))}
\end{Highlighting}
\end{Shaded}

\begin{Shaded}
\begin{Highlighting}[]
\NormalTok{## Compare the distributions using the raw bootstrap distribution, and the distribution using the moments only}


\NormalTok{density.boot=}\KeywordTok{density}\NormalTok{(beta.hat.boot);}
\CommentTok{# Theoretical density (from CLT)}
\NormalTok{xgrid =}\StringTok{ }\KeywordTok{seq}\NormalTok{(}\DataTypeTok{from=}\KeywordTok{min}\NormalTok{(beta.hat.boot),}\DataTypeTok{to=}\KeywordTok{max}\NormalTok{(beta.hat.boot),}\DataTypeTok{length.out=}\DecValTok{100}\NormalTok{);}
\NormalTok{normal.pdf=}\StringTok{ }\KeywordTok{dnorm}\NormalTok{(xgrid,}\DataTypeTok{mean=}\KeywordTok{mean}\NormalTok{(beta.hat.boot),}\DataTypeTok{sd=}\KeywordTok{sd}\NormalTok{(beta.hat.boot))}

\KeywordTok{plot}\NormalTok{(density.boot,}\DataTypeTok{xlab=}\StringTok{"Standardized intercept"}\NormalTok{,}\DataTypeTok{col=}\StringTok{'red'}\NormalTok{,}\DataTypeTok{lwd=}\DecValTok{3}\NormalTok{,}\DataTypeTok{main=}\StringTok{''}\NormalTok{,}\DataTypeTok{lty=}\DecValTok{2}\NormalTok{)}
\KeywordTok{lines}\NormalTok{(normal.pdf}\OperatorTok{~}\NormalTok{xgrid,}\DataTypeTok{lwd=}\DecValTok{2}\NormalTok{,}\DataTypeTok{lty=}\DecValTok{3}\NormalTok{)}
\KeywordTok{legend}\NormalTok{(}\DataTypeTok{x=}\FloatTok{0.15}\NormalTok{,}\DataTypeTok{y=}\DecValTok{50}\NormalTok{,}\DataTypeTok{legend=}\KeywordTok{c}\NormalTok{(}\StringTok{"Empirical"}\NormalTok{, }\StringTok{"Moment-based"}\NormalTok{),}\DataTypeTok{lwd=}\DecValTok{3}\NormalTok{,}\DataTypeTok{col=}\KeywordTok{c}\NormalTok{(}\StringTok{'red'}\NormalTok{,}\StringTok{'green'}\NormalTok{,}\StringTok{'blue'}\NormalTok{,}\StringTok{'black'}\NormalTok{),}\DataTypeTok{lty=}\KeywordTok{c}\NormalTok{(}\DecValTok{2}\NormalTok{,}\DecValTok{3}\NormalTok{))}
\end{Highlighting}
\end{Shaded}

\section{Small sample size}\label{small-sample-size}

Reading materials: Slides 49 - 57 in STA108\_LinearRegression\_S20.pdf.

When the sample size is small, the asymptotic distribution is not close
to the true sampling distribution. Neither the central limit theorem or
the bootstrap distribution are good approximation of the asymptotic
distribution.

\begin{Shaded}
\begin{Highlighting}[]
\CommentTok{# Simulation with small sample size}
\CommentTok{# We will demonstrate the case with 5 samples}
\CommentTok{# You will find the code in this trunk almost identical with previous code}

\CommentTok{# Generate one set of data:}
\KeywordTok{set.seed}\NormalTok{(}\DecValTok{1}\NormalTok{)}
\NormalTok{n=}\DecValTok{10}\NormalTok{;}
\NormalTok{x=}\KeywordTok{as.matrix}\NormalTok{(}\KeywordTok{rnorm}\NormalTok{(n,}\DataTypeTok{mean=}\DecValTok{10}\NormalTok{,}\DataTypeTok{sd=}\DecValTok{2}\NormalTok{),}\DataTypeTok{ncol=}\NormalTok{n);}
\NormalTok{beta.true=}\KeywordTok{c}\NormalTok{(}\DecValTok{20}\NormalTok{,}\FloatTok{0.15}\NormalTok{)}
\NormalTok{Ey=}\StringTok{ }\NormalTok{x}\OperatorTok{*}\NormalTok{beta.true[}\DecValTok{2}\NormalTok{]}\OperatorTok{+}\NormalTok{beta.true[}\DecValTok{1}\NormalTok{];}

\NormalTok{error.terms=}\StringTok{ }\KeywordTok{rnorm}\NormalTok{(n)}\OperatorTok{*}\DecValTok{5}\NormalTok{; }\CommentTok{# Normal errors!}
\NormalTok{y=Ey}\OperatorTok{+}\NormalTok{error.terms;}
\NormalTok{beta.hat=}\KeywordTok{fit.linear.model}\NormalTok{(}\DataTypeTok{covariate=}\NormalTok{x,}\DataTypeTok{outcome=}\NormalTok{y);}
\KeywordTok{plot}\NormalTok{(}\DataTypeTok{x=}\NormalTok{x,}\DataTypeTok{y=}\NormalTok{y,}\DataTypeTok{pch=}\DecValTok{16}\NormalTok{,}\DataTypeTok{cex=}\DecValTok{2}\NormalTok{)}
\KeywordTok{abline}\NormalTok{(}\DataTypeTok{b=}\NormalTok{beta.hat[}\DecValTok{2}\NormalTok{],}\DataTypeTok{a=}\NormalTok{beta.hat[}\DecValTok{1}\NormalTok{],}\DataTypeTok{col=}\StringTok{'red'}\NormalTok{,}\DataTypeTok{lwd=}\DecValTok{2}\NormalTok{)}
\end{Highlighting}
\end{Shaded}

\begin{Shaded}
\begin{Highlighting}[]
\NormalTok{## Simulate the true sampling distribution:}
\NormalTok{simulate.one.instance<-}\ControlFlowTok{function}\NormalTok{(x,beta.true)\{}
\NormalTok{  n=}\KeywordTok{length}\NormalTok{(x);}
\NormalTok{  Ey=}\StringTok{ }\NormalTok{x}\OperatorTok{*}\NormalTok{beta.true[}\DecValTok{2}\NormalTok{]}\OperatorTok{+}\NormalTok{beta.true[}\DecValTok{1}\NormalTok{];}
\NormalTok{  error.terms=}\StringTok{ }\NormalTok{(}\KeywordTok{runif}\NormalTok{(n)}\OperatorTok{-}\FloatTok{0.5}\NormalTok{)}\OperatorTok{*}\DecValTok{5}\NormalTok{;}
\NormalTok{  y=Ey}\OperatorTok{+}\NormalTok{error.terms;}
\NormalTok{  beta.hat=}\KeywordTok{fit.linear.model}\NormalTok{(}\DataTypeTok{covariate=}\NormalTok{x,}\DataTypeTok{outcome=}\NormalTok{y);}
  \KeywordTok{return}\NormalTok{(beta.hat)}
\NormalTok{\}}

\NormalTok{### Set the number of replicates to be 10000}
\NormalTok{N.sim=}\FloatTok{1e4}\NormalTok{;}
\NormalTok{true.beta.hat=}\KeywordTok{replicate}\NormalTok{(N.sim,}\KeywordTok{simulate.one.instance}\NormalTok{(x,beta.true));}

\NormalTok{### Obtain the mean and standard deviation using CLT:}
\NormalTok{residual.sum.of.squares=}\KeywordTok{sum.of.squares}\NormalTok{(}\DataTypeTok{beta=}\NormalTok{beta.hat,}\DataTypeTok{covariate=}\NormalTok{x,}\DataTypeTok{outcome=}\NormalTok{y)}
\NormalTok{sigma.sq.hat=residual.sum.of.squares}\OperatorTok{/}\NormalTok{(n}\OperatorTok{-}\DecValTok{2}\NormalTok{) }\CommentTok{# estimates from one instance}

  
\NormalTok{### Obtain the bootstrap distribution }
\NormalTok{B=}\FloatTok{1e5}\NormalTok{;}
\NormalTok{boot.beta.hat=}\KeywordTok{replicate}\NormalTok{(B,}\KeywordTok{boot.one.instance}\NormalTok{(}\DataTypeTok{covariate=}\NormalTok{x,}\DataTypeTok{outcome=}\NormalTok{y));}
\end{Highlighting}
\end{Shaded}

\begin{Shaded}
\begin{Highlighting}[]
\NormalTok{density.true=}\KeywordTok{density}\NormalTok{( (true.beta.hat[}\DecValTok{2}\NormalTok{,}\DecValTok{1}\NormalTok{,]}\OperatorTok{-}\KeywordTok{mean}\NormalTok{(true.beta.hat[}\DecValTok{2}\NormalTok{,}\DecValTok{1}\NormalTok{,]))}\OperatorTok{/}\KeywordTok{sd}\NormalTok{(true.beta.hat[}\DecValTok{2}\NormalTok{,}\DecValTok{1}\NormalTok{,]) );}

\NormalTok{density.boot=}\KeywordTok{density}\NormalTok{((boot.beta.hat}\OperatorTok{-}\KeywordTok{mean}\NormalTok{(boot.beta.hat))}\OperatorTok{/}\KeywordTok{sd}\NormalTok{(boot.beta.hat));}
\CommentTok{# Theoretical density (from CLT)}
\NormalTok{xgrid =}\StringTok{ }\KeywordTok{seq}\NormalTok{(}\DataTypeTok{from=}\OperatorTok{-}\DecValTok{4}\NormalTok{,}\DataTypeTok{to=}\DecValTok{4}\NormalTok{,}\DataTypeTok{length.out=}\DecValTok{100}\NormalTok{);}
\NormalTok{normal.pdf=}\StringTok{ }\KeywordTok{dnorm}\NormalTok{(xgrid)}


\KeywordTok{plot}\NormalTok{(density.true,}\DataTypeTok{xlab=}\StringTok{"Estimated intercept"}\NormalTok{,}\DataTypeTok{lwd=}\DecValTok{3}\NormalTok{,}\DataTypeTok{ylim=}\KeywordTok{c}\NormalTok{(}\DecValTok{0}\NormalTok{,}\FloatTok{0.5}\NormalTok{),}\DataTypeTok{xlim=}\KeywordTok{c}\NormalTok{(}\OperatorTok{-}\DecValTok{4}\NormalTok{,}\DecValTok{4}\NormalTok{),}\DataTypeTok{main=}\StringTok{''}\NormalTok{,}\DataTypeTok{lty=}\DecValTok{3}\NormalTok{)}
\KeywordTok{lines}\NormalTok{(density.boot,}\DataTypeTok{col=}\StringTok{'green'}\NormalTok{,}\DataTypeTok{lwd=}\DecValTok{3}\NormalTok{,}\DataTypeTok{lty=}\DecValTok{2}\NormalTok{)}
\KeywordTok{lines}\NormalTok{(normal.pdf}\OperatorTok{~}\NormalTok{xgrid,}\DataTypeTok{col=}\StringTok{'red'}\NormalTok{,,}\DataTypeTok{lwd=}\DecValTok{2}\NormalTok{,}\DataTypeTok{lty=}\DecValTok{2}\NormalTok{)}
\KeywordTok{legend}\NormalTok{(}\DataTypeTok{x=}\DecValTok{2}\NormalTok{,}\DataTypeTok{y=}\FloatTok{0.5}\NormalTok{,}\DataTypeTok{legend=}\KeywordTok{c}\NormalTok{(}\StringTok{"True"}\NormalTok{,}\StringTok{"Bootstrap"}\NormalTok{,}\StringTok{"CLT"}\NormalTok{),}\DataTypeTok{lwd=}\DecValTok{3}\NormalTok{,}\DataTypeTok{col=}\KeywordTok{c}\NormalTok{(}\StringTok{'black'}\NormalTok{,}\StringTok{'green'}\NormalTok{,}\StringTok{'red'}\NormalTok{),}\DataTypeTok{lty=}\KeywordTok{c}\NormalTok{(}\DecValTok{3}\NormalTok{,}\DecValTok{2}\NormalTok{,}\DecValTok{2}\NormalTok{))}
\end{Highlighting}
\end{Shaded}

\begin{Shaded}
\begin{Highlighting}[]
\NormalTok{## With very few samples, we have to exploit the parametric assumption}
\NormalTok{## In this case, we take advantage of the normality assumption on the errors}


\NormalTok{t.pdf=}\StringTok{ }\KeywordTok{dt}\NormalTok{(xgrid,}\DataTypeTok{df=}\NormalTok{n}\OperatorTok{-}\DecValTok{2}\NormalTok{)}

\KeywordTok{plot}\NormalTok{(density.true,}\DataTypeTok{xlab=}\StringTok{"Estimated intercept"}\NormalTok{,}\DataTypeTok{lwd=}\DecValTok{3}\NormalTok{,}\DataTypeTok{ylim=}\KeywordTok{c}\NormalTok{(}\DecValTok{0}\NormalTok{,}\FloatTok{0.5}\NormalTok{),}\DataTypeTok{xlim=}\KeywordTok{c}\NormalTok{(}\OperatorTok{-}\DecValTok{4}\NormalTok{,}\DecValTok{4}\NormalTok{),}\DataTypeTok{main=}\StringTok{''}\NormalTok{,}\DataTypeTok{lty=}\DecValTok{3}\NormalTok{)}
\KeywordTok{lines}\NormalTok{(density.boot,}\DataTypeTok{col=}\StringTok{'green'}\NormalTok{,}\DataTypeTok{lwd=}\DecValTok{3}\NormalTok{,}\DataTypeTok{lty=}\DecValTok{2}\NormalTok{)}
\KeywordTok{lines}\NormalTok{(normal.pdf}\OperatorTok{~}\NormalTok{xgrid,}\DataTypeTok{col=}\StringTok{'red'}\NormalTok{,,}\DataTypeTok{lwd=}\DecValTok{2}\NormalTok{,}\DataTypeTok{lty=}\DecValTok{2}\NormalTok{)}

\KeywordTok{lines}\NormalTok{(t.pdf}\OperatorTok{~}\NormalTok{xgrid,}\DataTypeTok{col=}\StringTok{'blue'}\NormalTok{,,}\DataTypeTok{lwd=}\DecValTok{2}\NormalTok{,}\DataTypeTok{lty=}\DecValTok{2}\NormalTok{)}
\KeywordTok{legend}\NormalTok{(}\DataTypeTok{x=}\DecValTok{2}\NormalTok{,}\DataTypeTok{y=}\FloatTok{0.5}\NormalTok{,}\DataTypeTok{legend=}\KeywordTok{c}\NormalTok{(}\StringTok{"True"}\NormalTok{,}\StringTok{"Bootstrap"}\NormalTok{,}\StringTok{"CLT"}\NormalTok{,}\StringTok{'Student t'}\NormalTok{),}\DataTypeTok{lwd=}\DecValTok{3}\NormalTok{,}\DataTypeTok{col=}\KeywordTok{c}\NormalTok{(}\StringTok{'black'}\NormalTok{,}\StringTok{'green'}\NormalTok{,}\StringTok{'red'}\NormalTok{,}\StringTok{'blue'}\NormalTok{),}\DataTypeTok{lty=}\KeywordTok{c}\NormalTok{(}\DecValTok{3}\NormalTok{,}\DecValTok{2}\NormalTok{,}\DecValTok{2}\NormalTok{))}
\end{Highlighting}
\end{Shaded}

\begin{Shaded}
\begin{Highlighting}[]
\NormalTok{## The Student t distribution grows more similar to the standard normal as its degrees of freedom increases }

\NormalTok{t_dist=}\KeywordTok{dt}\NormalTok{(xgrid,}\DataTypeTok{df=}\DecValTok{1}\NormalTok{);}
\KeywordTok{plot}\NormalTok{(t_dist}\OperatorTok{~}\NormalTok{xgrid,}\DataTypeTok{xlab=}\StringTok{"beta 1 hat"}\NormalTok{,}\DataTypeTok{col=}\KeywordTok{rgb}\NormalTok{(}\DecValTok{1}\NormalTok{,}\DecValTok{0}\NormalTok{,}\DecValTok{0}\NormalTok{,}\FloatTok{0.1}\NormalTok{),}\DataTypeTok{lwd=}\DecValTok{3}\NormalTok{,}\DataTypeTok{type=}\StringTok{"l"}\NormalTok{,}\DataTypeTok{ylim=}\KeywordTok{c}\NormalTok{(}\DecValTok{0}\NormalTok{,}\FloatTok{0.5}\NormalTok{),}\DataTypeTok{xlim=}\KeywordTok{c}\NormalTok{(}\OperatorTok{-}\DecValTok{5}\NormalTok{,}\DecValTok{5}\NormalTok{),}\DataTypeTok{main=}\StringTok{''}\NormalTok{)}

\ControlFlowTok{for}\NormalTok{(i }\ControlFlowTok{in} \DecValTok{2}\OperatorTok{:}\DecValTok{10}\NormalTok{)\{}
\NormalTok{t_dist=}\KeywordTok{dt}\NormalTok{(xgrid,}\DataTypeTok{df=}\NormalTok{i);}

\KeywordTok{lines}\NormalTok{(t_dist}\OperatorTok{~}\NormalTok{xgrid,}\DataTypeTok{col=}\KeywordTok{rgb}\NormalTok{(}\DecValTok{1}\NormalTok{,}\DecValTok{0}\NormalTok{,}\DecValTok{0}\NormalTok{,i}\OperatorTok{/}\DecValTok{10}\NormalTok{),}\DataTypeTok{lwd=}\DecValTok{3}\NormalTok{)}

\NormalTok{\}}

\KeywordTok{legend}\NormalTok{(}\DataTypeTok{x=}\DecValTok{2}\NormalTok{,}\DataTypeTok{y=}\FloatTok{0.46}\NormalTok{,}\DataTypeTok{legend=}\KeywordTok{c}\NormalTok{(}\StringTok{"Student t"}\NormalTok{, }\StringTok{"N(0,1)"}\NormalTok{),}\DataTypeTok{lwd=}\DecValTok{3}\NormalTok{,}\DataTypeTok{col=}\KeywordTok{c}\NormalTok{(}\StringTok{'red'}\NormalTok{,}\StringTok{'green'}\NormalTok{))}

\KeywordTok{lines}\NormalTok{(normal.pdf}\OperatorTok{~}\NormalTok{xgrid,}\DataTypeTok{col=}\KeywordTok{rgb}\NormalTok{(}\DecValTok{0}\NormalTok{,}\DecValTok{1}\NormalTok{,}\DecValTok{0}\NormalTok{,}\FloatTok{0.4}\NormalTok{),}\DataTypeTok{lwd=}\DecValTok{5}\NormalTok{)}
\end{Highlighting}
\end{Shaded}

\chapter{Statistical inference: Confidence Intervals}\label{ch:CI}

Reading materials: Slides 57 - 72 in STA108\_LinearRegression\_S20.pdf.

\section{Confidence interval}\label{confidence-interval}

\begin{Shaded}
\begin{Highlighting}[]
\NormalTok{## We will look at synthetic data here, because we have control over the truth }
\KeywordTok{set.seed}\NormalTok{(}\DecValTok{1}\NormalTok{)}
\NormalTok{n=}\DecValTok{50}\NormalTok{;}
\NormalTok{x=}\KeywordTok{as.matrix}\NormalTok{(}\KeywordTok{rnorm}\NormalTok{(n,}\DataTypeTok{mean=}\DecValTok{10}\NormalTok{,}\DataTypeTok{sd=}\DecValTok{2}\NormalTok{),}\DataTypeTok{ncol=}\NormalTok{n);}
\NormalTok{beta.true=}\KeywordTok{c}\NormalTok{(}\DecValTok{20}\NormalTok{,}\FloatTok{0.15}\NormalTok{)}
\NormalTok{Ey=}\StringTok{ }\NormalTok{x}\OperatorTok{*}\NormalTok{beta.true[}\DecValTok{2}\NormalTok{]}\OperatorTok{+}\NormalTok{beta.true[}\DecValTok{1}\NormalTok{];}
\NormalTok{error.terms=}\StringTok{ }\KeywordTok{rnorm}\NormalTok{(n)}\OperatorTok{*}\DecValTok{5}\NormalTok{;}
\NormalTok{y=Ey}\OperatorTok{+}\NormalTok{error.terms;}
\end{Highlighting}
\end{Shaded}

\begin{Shaded}
\begin{Highlighting}[]
\CommentTok{# For a given confidence level alpha, construct a 100(1-alpha)% confidence interval }

\NormalTok{alpha=}\FloatTok{0.023}\NormalTok{;}
\CommentTok{# There is a function confint() in R}
\NormalTok{fit.lm=}\KeywordTok{lm}\NormalTok{(y}\OperatorTok{~}\NormalTok{x}\OperatorTok{+}\DecValTok{1}\NormalTok{);}
\KeywordTok{confint}\NormalTok{(fit.lm,}\DataTypeTok{level=}\DecValTok{1}\OperatorTok{-}\NormalTok{alpha)}
\end{Highlighting}
\end{Shaded}

\begin{Shaded}
\begin{Highlighting}[]
\CommentTok{# We will implement our own version}

\NormalTok{## We actually have most of the part ready}
\NormalTok{beta.hat=}\KeywordTok{fit.linear.model}\NormalTok{(}\DataTypeTok{covariate=}\NormalTok{x,}\DataTypeTok{outcome=}\NormalTok{y);}
\NormalTok{beta.sd=}\KeywordTok{estimate.coef.sd}\NormalTok{(}\DataTypeTok{beta=}\NormalTok{beta.hat,}\DataTypeTok{covariate=}\NormalTok{x,}\DataTypeTok{outcome=}\NormalTok{y);}

\NormalTok{## The only missing piece is the quantile}
\NormalTok{conf.int.quantile<-}\ControlFlowTok{function}\NormalTok{(alpha,type,...)\{}
  \ControlFlowTok{if}\NormalTok{(type}\OperatorTok{==}\StringTok{"t"}\NormalTok{)\{}
\NormalTok{    out=}\KeywordTok{qt}\NormalTok{(}\KeywordTok{c}\NormalTok{(}\DecValTok{1}\OperatorTok{-}\NormalTok{alpha}\OperatorTok{/}\DecValTok{2}\NormalTok{,alpha}\OperatorTok{/}\DecValTok{2}\NormalTok{), ... ) }
\NormalTok{  \}}\ControlFlowTok{else} \ControlFlowTok{if}\NormalTok{ (type}\OperatorTok{==}\StringTok{"normal"}\NormalTok{)\{}
\NormalTok{    out=}\KeywordTok{qnorm}\NormalTok{(}\KeywordTok{c}\NormalTok{(}\DecValTok{1}\OperatorTok{-}\NormalTok{alpha}\OperatorTok{/}\DecValTok{2}\NormalTok{,alpha}\OperatorTok{/}\DecValTok{2}\NormalTok{), ... ) }
\NormalTok{  \}}
  \KeywordTok{return}\NormalTok{(out)}
\NormalTok{\}}
\NormalTok{quants<-}\KeywordTok{conf.int.quantile}\NormalTok{(alpha,}\DataTypeTok{type=}\StringTok{'t'}\NormalTok{,}\DataTypeTok{df=}\NormalTok{n}\OperatorTok{-}\DecValTok{2}\NormalTok{)}

\NormalTok{beta.hat}\OperatorTok\KeywordTok{c}\NormalTok{(}\DecValTok{1}\NormalTok{,}\DecValTok{1}\NormalTok{)}\OperatorTok{-}\NormalTok{beta.sd}\OperatorTok\NormalTok{quants}

\NormalTok{## Compare with the output from confint }
\KeywordTok{confint}\NormalTok{(fit.lm,}\DataTypeTok{level=}\DecValTok{1}\OperatorTok{-}\NormalTok{alpha)}

\NormalTok{## How about using bootstrap to construct the CIs?}
\NormalTok{boot.fit<-}\ControlFlowTok{function}\NormalTok{(covariate,outcome)\{}
\NormalTok{  n=}\KeywordTok{length}\NormalTok{(outcome);}
\NormalTok{  sample_indices =}\StringTok{ }\KeywordTok{sample}\NormalTok{(}\DecValTok{1}\OperatorTok{:}\NormalTok{n,n,}\DataTypeTok{replace=}\OtherTok{TRUE}\NormalTok{) }\CommentTok{# sampling with replacement}
\NormalTok{  covariate.boot=}\StringTok{ }\NormalTok{covariate[sample_indices]; outcome.boot=}\StringTok{ }\NormalTok{outcome[sample_indices];}
   
\NormalTok{  beta.hat=}\KeywordTok{fit.linear.model}\NormalTok{(}\DataTypeTok{covariate=}\NormalTok{covariate.boot,}\DataTypeTok{outcome=}\NormalTok{outcome.boot);}
  \KeywordTok{return}\NormalTok{(}\KeywordTok{t}\NormalTok{(beta.hat ))}
\NormalTok{\}}
\NormalTok{B=}\FloatTok{1e5}\NormalTok{;}
\NormalTok{beta.hat.boot=}\KeywordTok{replicate}\NormalTok{(B,}\KeywordTok{boot.fit}\NormalTok{(}\DataTypeTok{covariate=}\NormalTok{x,}\DataTypeTok{outcome=}\NormalTok{y));}

\KeywordTok{dim}\NormalTok{(beta.hat.boot)}
\KeywordTok{apply}\NormalTok{(beta.hat.boot[}\DecValTok{1}\NormalTok{,,],}\DecValTok{1}\NormalTok{,quantile,}\DataTypeTok{probs=}\KeywordTok{c}\NormalTok{(alpha}\OperatorTok{/}\DecValTok{2}\NormalTok{,}\DecValTok{1}\OperatorTok{-}\NormalTok{alpha}\OperatorTok{/}\DecValTok{2}\NormalTok{))}

\CommentTok{# We can wrap this up into a function}
\NormalTok{conf.int<-}\ControlFlowTok{function}\NormalTok{(alpha,type,covariate,outcome,}\DataTypeTok{B=}\FloatTok{1e5}\NormalTok{)\{}
  
\NormalTok{  beta.hat=}\KeywordTok{fit.linear.model}\NormalTok{(covariate,outcome);}
\NormalTok{  beta.sd=}\KeywordTok{estimate.coef.sd}\NormalTok{(}\DataTypeTok{beta=}\NormalTok{beta.hat,covariate,outcome);}
  \ControlFlowTok{if}\NormalTok{(type}\OperatorTok{==}\StringTok{'bootstrap'}\NormalTok{)\{}
\NormalTok{    beta.hat.boot=}\KeywordTok{replicate}\NormalTok{(B,}\KeywordTok{boot.fit}\NormalTok{(covariate,outcome));}
\NormalTok{    out=}\KeywordTok{t}\NormalTok{(}\KeywordTok{apply}\NormalTok{(beta.hat.boot[}\DecValTok{1}\NormalTok{,,],}\DecValTok{1}\NormalTok{,quantile,}\DataTypeTok{probs=}\KeywordTok{c}\NormalTok{(alpha}\OperatorTok{/}\DecValTok{2}\NormalTok{,}\DecValTok{1}\OperatorTok{-}\NormalTok{alpha}\OperatorTok{/}\DecValTok{2}\NormalTok{)));}
\NormalTok{  \}}\ControlFlowTok{else} \ControlFlowTok{if}\NormalTok{(type}\OperatorTok{==}\StringTok{'t'}\NormalTok{)\{}
\NormalTok{    quants<-}\KeywordTok{conf.int.quantile}\NormalTok{(alpha,}\DataTypeTok{type=}\StringTok{'t'}\NormalTok{,}\DataTypeTok{df=}\NormalTok{n}\OperatorTok{-}\DecValTok{2}\NormalTok{)}
\NormalTok{    out=beta.hat}\OperatorTok\KeywordTok{c}\NormalTok{(}\DecValTok{1}\NormalTok{,}\DecValTok{1}\NormalTok{)}\OperatorTok{-}\NormalTok{beta.sd}\OperatorTok\NormalTok{quants;}
\NormalTok{  \}}\ControlFlowTok{else}\NormalTok{\{}
\NormalTok{    quants<-}\KeywordTok{conf.int.quantile}\NormalTok{(alpha,}\DataTypeTok{type=}\StringTok{'normal'}\NormalTok{)}
\NormalTok{    out=beta.hat}\OperatorTok\KeywordTok{c}\NormalTok{(}\DecValTok{1}\NormalTok{,}\DecValTok{1}\NormalTok{)}\OperatorTok{-}\NormalTok{beta.sd}\OperatorTok\NormalTok{quants;}
\NormalTok{  \}}
  
  \KeywordTok{colnames}\NormalTok{(out)=}\KeywordTok{c}\NormalTok{( }\KeywordTok{paste}\NormalTok{(}\KeywordTok{round}\NormalTok{(alpha}\OperatorTok{*}\DecValTok{50}\NormalTok{,}\DataTypeTok{digits=}\DecValTok{3}\NormalTok{),}\StringTok{'%'}\NormalTok{), }\KeywordTok{paste}\NormalTok{(}\DecValTok{100}\OperatorTok{-}\KeywordTok{round}\NormalTok{(alpha}\OperatorTok{*}\DecValTok{50}\NormalTok{,}\DataTypeTok{digits=}\DecValTok{3}\NormalTok{),}\StringTok{'%'}\NormalTok{)  )}
  \KeywordTok{return}\NormalTok{(out)}
\NormalTok{\}}
\KeywordTok{conf.int}\NormalTok{(}\DataTypeTok{alpha=}\NormalTok{alpha,}\DataTypeTok{type=}\StringTok{'t'}\NormalTok{,}\DataTypeTok{covariate=}\NormalTok{x,}\DataTypeTok{outcome=}\NormalTok{y)}
\KeywordTok{confint}\NormalTok{(fit.lm,}\DataTypeTok{level=}\DecValTok{1}\OperatorTok{-}\NormalTok{alpha)}
\end{Highlighting}
\end{Shaded}

\begin{Shaded}
\begin{Highlighting}[]
\NormalTok{## Verify the coverage of the confidence invervals }
\NormalTok{simulate.one.instance<-}\ControlFlowTok{function}\NormalTok{(x,beta.true,alpha)\{}
\NormalTok{  n=}\KeywordTok{length}\NormalTok{(x);}
\NormalTok{  Ey=}\StringTok{ }\NormalTok{x}\OperatorTok{*}\NormalTok{beta.true[}\DecValTok{2}\NormalTok{]}\OperatorTok{+}\NormalTok{beta.true[}\DecValTok{1}\NormalTok{];}
\NormalTok{  error.terms=}\StringTok{ }\KeywordTok{rnorm}\NormalTok{(n)}\OperatorTok{*}\DecValTok{5}\NormalTok{;}
\NormalTok{  y=Ey}\OperatorTok{+}\NormalTok{error.terms;}
\NormalTok{  CIs=}\KeywordTok{conf.int}\NormalTok{(}\DataTypeTok{alpha=}\NormalTok{alpha,}\DataTypeTok{type=}\StringTok{'t'}\NormalTok{,}\DataTypeTok{covariate=}\NormalTok{x,}\DataTypeTok{outcome=}\NormalTok{y);}
  \KeywordTok{return}\NormalTok{(CIs)}
\NormalTok{\}}

\NormalTok{N.sim=}\FloatTok{1e4}\NormalTok{;}
\KeywordTok{set.seed}\NormalTok{(}\DecValTok{1}\NormalTok{)}
\NormalTok{n=}\DecValTok{50}\NormalTok{;}
\NormalTok{x=}\KeywordTok{as.matrix}\NormalTok{(}\KeywordTok{rnorm}\NormalTok{(n,}\DataTypeTok{mean=}\DecValTok{10}\NormalTok{,}\DataTypeTok{sd=}\DecValTok{2}\NormalTok{),}\DataTypeTok{ncol=}\NormalTok{n);}
\NormalTok{beta.true=}\KeywordTok{c}\NormalTok{(}\DecValTok{20}\NormalTok{,}\FloatTok{0.15}\NormalTok{)}
\NormalTok{sim.CIs=}\KeywordTok{replicate}\NormalTok{(N.sim,}\KeywordTok{simulate.one.instance}\NormalTok{(x,beta.true,alpha));}
\end{Highlighting}
\end{Shaded}

\begin{Shaded}
\begin{Highlighting}[]
\NormalTok{## Visualize and verify the coverage for the slope }
\NormalTok{coverage=}\KeywordTok{sum}\NormalTok{(sim.CIs[}\DecValTok{2}\NormalTok{,}\DecValTok{1}\NormalTok{,]}\OperatorTok{<}\NormalTok{beta.true[}\DecValTok{2}\NormalTok{] }\OperatorTok{&}\StringTok{ }\NormalTok{sim.CIs[}\DecValTok{2}\NormalTok{,}\DecValTok{2}\NormalTok{,]}\OperatorTok{>}\NormalTok{beta.true[}\DecValTok{2}\NormalTok{])}\OperatorTok{/}\NormalTok{N.sim;}
\DecValTok{1}\OperatorTok{-}\NormalTok{alpha}

\KeywordTok{plot}\NormalTok{(}\DecValTok{0}\NormalTok{,}\DecValTok{0}\NormalTok{,}\DataTypeTok{col=}\StringTok{"white"}\NormalTok{,}\DataTypeTok{xlim=}\KeywordTok{c}\NormalTok{(}\KeywordTok{min}\NormalTok{(sim.CIs[}\DecValTok{2}\NormalTok{,}\DecValTok{1}\NormalTok{,]),}\KeywordTok{max}\NormalTok{(sim.CIs[}\DecValTok{2}\NormalTok{,}\DecValTok{2}\NormalTok{,])),}\DataTypeTok{ylim=}\KeywordTok{c}\NormalTok{(}\DecValTok{0}\NormalTok{,}\DecValTok{20}\NormalTok{),}\DataTypeTok{xlab=}\StringTok{"Slope"}\NormalTok{,}\DataTypeTok{ylab=}\StringTok{"Index"}\NormalTok{,}\DataTypeTok{main=}\KeywordTok{paste}\NormalTok{(}\StringTok{'Average Coverage:'}\NormalTok{, }\KeywordTok{signif}\NormalTok{(coverage,}\DecValTok{3}\NormalTok{)))}
\KeywordTok{abline}\NormalTok{(}\DataTypeTok{v=}\NormalTok{beta.true[}\DecValTok{2}\NormalTok{],}\DataTypeTok{lwd=}\DecValTok{3}\NormalTok{,}\DataTypeTok{col=}\StringTok{'red'}\NormalTok{)}
\ControlFlowTok{for}\NormalTok{(i }\ControlFlowTok{in} \DecValTok{1}\OperatorTok{:}\DecValTok{19}\NormalTok{)\{}
  \KeywordTok{segments}\NormalTok{(sim.CIs[}\DecValTok{2}\NormalTok{,}\DecValTok{1}\NormalTok{,i],i,sim.CIs[}\DecValTok{2}\NormalTok{,}\DecValTok{2}\NormalTok{,i],i,}\DataTypeTok{lwd=}\DecValTok{3}\NormalTok{)}
\NormalTok{\}}
\end{Highlighting}
\end{Shaded}

\subsection{Determine cutoffs}\label{determine-cutoffs}

We can definite multiple confidence intervals of the same confidence
levels for on the same sampling distribution.

\begin{Shaded}
\begin{Highlighting}[]
\NormalTok{alpha=}\FloatTok{0.05}\NormalTok{;}

\NormalTok{cutoffs=}\KeywordTok{c}\NormalTok{(}\FloatTok{0.01}\NormalTok{,}\FloatTok{0.025}\NormalTok{,}\FloatTok{0.049}\NormalTok{);}

\NormalTok{colorlist=}\KeywordTok{c}\NormalTok{(}\StringTok{'red'}\NormalTok{,}\StringTok{'green'}\NormalTok{,}\StringTok{'blue'}\NormalTok{);}
\NormalTok{leg_text=}\KeywordTok{c}\NormalTok{(}\StringTok{"1"}\NormalTok{,}\StringTok{"2"}\NormalTok{,}\StringTok{"3"}\NormalTok{);}

\NormalTok{normal.pdf=}\StringTok{ }\KeywordTok{dnorm}\NormalTok{(xgrid)}
\KeywordTok{plot}\NormalTok{(normal.pdf}\OperatorTok{~}\NormalTok{xgrid,}\DataTypeTok{xlab=}\StringTok{"beta 1 hat"}\NormalTok{,}\DataTypeTok{ylab=}\StringTok{"Density"}\NormalTok{,}\DataTypeTok{col=}\KeywordTok{rgb}\NormalTok{(}\DecValTok{0}\NormalTok{,}\DecValTok{0}\NormalTok{,}\DecValTok{0}\NormalTok{,}\DecValTok{1}\NormalTok{),}\DataTypeTok{lwd=}\DecValTok{3}\NormalTok{,}\DataTypeTok{type=}\StringTok{"l"}\NormalTok{,}\DataTypeTok{ylim=}\KeywordTok{c}\NormalTok{(}\DecValTok{0}\NormalTok{,}\FloatTok{0.5}\NormalTok{),}\DataTypeTok{xlim=}\KeywordTok{c}\NormalTok{(}\OperatorTok{-}\DecValTok{5}\NormalTok{,}\DecValTok{5}\NormalTok{),}\DataTypeTok{main=}\StringTok{''}\NormalTok{)}

\ControlFlowTok{for}\NormalTok{(i }\ControlFlowTok{in} \DecValTok{1}\OperatorTok{:}\KeywordTok{length}\NormalTok{(cutoffs))\{}
\NormalTok{lower_bound =}\StringTok{ }\KeywordTok{qnorm}\NormalTok{(cutoffs[i]);}
\NormalTok{upper_bound =}\StringTok{ }\KeywordTok{qnorm}\NormalTok{(}\DecValTok{1}\OperatorTok{-}\NormalTok{alpha}\OperatorTok{+}\NormalTok{cutoffs[i]);}
\KeywordTok{segments}\NormalTok{(lower_bound,}\DecValTok{0}\NormalTok{,lower_bound,}\FloatTok{0.5}\NormalTok{,}\DataTypeTok{col=}\NormalTok{colorlist[i],}\DataTypeTok{lwd=}\DecValTok{3}\NormalTok{)}
\KeywordTok{segments}\NormalTok{(upper_bound,}\DecValTok{0}\NormalTok{,upper_bound,}\FloatTok{0.5}\NormalTok{,}\DataTypeTok{col=}\NormalTok{colorlist[i],}\DataTypeTok{lwd=}\DecValTok{3}\NormalTok{)}
\KeywordTok{segments}\NormalTok{(lower_bound,i}\OperatorTok{*}\FloatTok{0.1}\NormalTok{,upper_bound,i}\OperatorTok{*}\FloatTok{0.1}\NormalTok{,}\DataTypeTok{col=}\NormalTok{colorlist[i],}\DataTypeTok{lwd=}\DecValTok{3}\NormalTok{)}
\NormalTok{leg_text[i] =}\KeywordTok{paste}\NormalTok{(cutoffs[i],}\StringTok{"to"}\NormalTok{, }\DecValTok{1}\OperatorTok{-}\NormalTok{alpha}\OperatorTok{+}\NormalTok{cutoffs[i], }\StringTok{"(length:"}\NormalTok{, }\KeywordTok{signif}\NormalTok{(upper_bound}\OperatorTok{-}\NormalTok{lower_bound,}\DecValTok{2}\NormalTok{), }\StringTok{")"}\NormalTok{);}
\NormalTok{\}}
\KeywordTok{legend}\NormalTok{(}\DataTypeTok{x=}\OperatorTok{-}\DecValTok{5}\NormalTok{,}\DataTypeTok{y=}\FloatTok{0.5}\NormalTok{,}\DataTypeTok{legend=}\NormalTok{leg_text,}\DataTypeTok{lwd=}\DecValTok{3}\NormalTok{,}\DataTypeTok{col=}\NormalTok{colorlist)}
\end{Highlighting}
\end{Shaded}

\section{Prediction interval}\label{prediction-interval}

\begin{Shaded}
\begin{Highlighting}[]
\KeywordTok{set.seed}\NormalTok{(}\DecValTok{1}\NormalTok{)}
\NormalTok{n=}\DecValTok{50}\NormalTok{;}
\NormalTok{x=}\KeywordTok{as.matrix}\NormalTok{(}\KeywordTok{rnorm}\NormalTok{(n,}\DataTypeTok{mean=}\DecValTok{10}\NormalTok{,}\DataTypeTok{sd=}\DecValTok{2}\NormalTok{),}\DataTypeTok{ncol=}\NormalTok{n);}
\NormalTok{beta.true=}\KeywordTok{c}\NormalTok{(}\DecValTok{20}\NormalTok{,}\FloatTok{0.15}\NormalTok{)}
\NormalTok{Ey=}\StringTok{ }\NormalTok{x}\OperatorTok{*}\NormalTok{beta.true[}\DecValTok{2}\NormalTok{]}\OperatorTok{+}\NormalTok{beta.true[}\DecValTok{1}\NormalTok{];}
\NormalTok{error.terms=}\StringTok{ }\KeywordTok{rnorm}\NormalTok{(n)}\OperatorTok{*}\DecValTok{5}\NormalTok{;}
\NormalTok{y=Ey}\OperatorTok{+}\NormalTok{error.terms;}
\NormalTok{fit.lm=}\KeywordTok{lm}\NormalTok{(y}\OperatorTok{~}\KeywordTok{as.vector}\NormalTok{(x)}\OperatorTok{+}\DecValTok{1}\NormalTok{);}

\NormalTok{alpha=}\FloatTok{0.05}\NormalTok{;}
\NormalTok{dat.new =}\StringTok{ }\KeywordTok{data.frame}\NormalTok{(}\DataTypeTok{x=}\KeywordTok{c}\NormalTok{(}\DecValTok{10}\NormalTok{,}\DecValTok{20}\NormalTok{));}

\CommentTok{# Prediction interval (assume normality)}
\KeywordTok{predict}\NormalTok{(fit.lm,}\DataTypeTok{newdata=}\NormalTok{dat.new,}\DataTypeTok{interval=}\StringTok{"prediction"}\NormalTok{,}\DataTypeTok{level=}\DecValTok{1}\OperatorTok{-}\NormalTok{alpha)}

\CommentTok{# and confidence interval}
\KeywordTok{predict}\NormalTok{(fit.lm,}\DataTypeTok{newdata=}\NormalTok{dat.new,}\DataTypeTok{interval=}\StringTok{"confidence"}\NormalTok{,}\DataTypeTok{level=}\DecValTok{1}\OperatorTok{-}\NormalTok{alpha)}
\end{Highlighting}
\end{Shaded}

\begin{Shaded}
\begin{Highlighting}[]
\CommentTok{# Implementation is similar to our conf.int function, and is thus skipped.}
\CommentTok{# You can verify the coverage by modifying the code above.}
\end{Highlighting}
\end{Shaded}

\section{Simultaneous confidence
intervals/bands/regions}\label{simultaneous-confidence-intervalsbandsregions}

\begin{Shaded}
\begin{Highlighting}[]
\NormalTok{## We will return to the advertising data }
\NormalTok{fit.lm=}\KeywordTok{lm}\NormalTok{(sales}\OperatorTok{~}\NormalTok{TV}\OperatorTok{+}\DecValTok{1}\NormalTok{,}\DataTypeTok{data=}\NormalTok{dat.advertising)}
\NormalTok{fit.lm.sum=}\KeywordTok{summary}\NormalTok{(fit.lm)}
\NormalTok{alpha=}\FloatTok{0.05}\NormalTok{;}
\NormalTok{xgrid_pred=}\KeywordTok{seq}\NormalTok{(}\DataTypeTok{from=}\DecValTok{5}\NormalTok{,}\DataTypeTok{to=}\DecValTok{50}\NormalTok{,}\DataTypeTok{by=}\FloatTok{2.5}\NormalTok{);}
\NormalTok{dat.new =}\StringTok{ }\KeywordTok{data.frame}\NormalTok{(}\DataTypeTok{TV=}\NormalTok{xgrid_pred);}
\NormalTok{pointwise=}\KeywordTok{predict}\NormalTok{(fit.lm,}\DataTypeTok{newdata=}\NormalTok{dat.new,}\DataTypeTok{interval=}\StringTok{"confidence"}\NormalTok{,}\DataTypeTok{level=}\DecValTok{1}\OperatorTok{-}\NormalTok{alpha)}

\CommentTok{# Bonferroni corrected:}
\NormalTok{Bonf=}\KeywordTok{predict}\NormalTok{(fit.lm,}\DataTypeTok{newdata=}\NormalTok{dat.new,}\DataTypeTok{interval=}\StringTok{"confidence"}\NormalTok{,}\DataTypeTok{level=}\DecValTok{1}\OperatorTok{-}\NormalTok{alpha}\OperatorTok{/}\KeywordTok{length}\NormalTok{(xgrid_pred))}


\CommentTok{# Working-Hotelling}
\NormalTok{fstat=}\StringTok{ }\KeywordTok{qf}\NormalTok{(}\DecValTok{1}\OperatorTok{-}\NormalTok{alpha,}\DecValTok{2}\NormalTok{,}\KeywordTok{length}\NormalTok{(dat.advertising}\OperatorTok{$}\NormalTok{TV)}\OperatorTok{-}\DecValTok{2}\NormalTok{);}
\NormalTok{xdense=}\KeywordTok{seq}\NormalTok{(}\DataTypeTok{from=}\DecValTok{0}\NormalTok{,}\DataTypeTok{to=}\DecValTok{50}\NormalTok{,}\DataTypeTok{by=}\FloatTok{0.1}\NormalTok{);}
\NormalTok{Sxx=}\StringTok{ }\KeywordTok{sum}\NormalTok{( (dat.advertising}\OperatorTok{$}\NormalTok{TV}\OperatorTok{-}\KeywordTok{mean}\NormalTok{(dat.advertising}\OperatorTok{$}\NormalTok{TV))}\OperatorTok{^}\DecValTok{2}\NormalTok{);}
\NormalTok{pivot=}\KeywordTok{sqrt}\NormalTok{(}\DecValTok{2}\OperatorTok{*}\NormalTok{fstat)}\OperatorTok{*}\NormalTok{fit.lm.sum}\OperatorTok{$}\NormalTok{sigma}\OperatorTok{*}\KeywordTok{sqrt}\NormalTok{(}\DecValTok{1}\OperatorTok{/}\KeywordTok{length}\NormalTok{(dat.advertising}\OperatorTok{$}\NormalTok{TV)}\OperatorTok{+}\NormalTok{(xdense}\OperatorTok{-}\KeywordTok{mean}\NormalTok{(dat.advertising}\OperatorTok{$}\NormalTok{TV))}\OperatorTok{^}\DecValTok{2}\OperatorTok{/}\NormalTok{Sxx);}
\NormalTok{ylb=fit.lm}\OperatorTok{$}\NormalTok{coefficients[}\DecValTok{2}\NormalTok{]}\OperatorTok{*}\NormalTok{xdense}\OperatorTok{+}\NormalTok{fit.lm}\OperatorTok{$}\NormalTok{coefficients[}\DecValTok{1}\NormalTok{]}\OperatorTok{-}\NormalTok{pivot;}
\NormalTok{yub=fit.lm}\OperatorTok{$}\NormalTok{coefficients[}\DecValTok{2}\NormalTok{]}\OperatorTok{*}\NormalTok{xdense}\OperatorTok{+}\NormalTok{fit.lm}\OperatorTok{$}\NormalTok{coefficients[}\DecValTok{1}\NormalTok{]}\OperatorTok{+}\NormalTok{pivot;}


\KeywordTok{plot}\NormalTok{(dat.advertising}\OperatorTok{$}\NormalTok{sales}\OperatorTok{~}\NormalTok{dat.advertising}\OperatorTok{$}\NormalTok{TV,}\DataTypeTok{pch=}\DecValTok{16}\NormalTok{,}\DataTypeTok{xlim=}\KeywordTok{c}\NormalTok{(}\DecValTok{0}\NormalTok{,}\DecValTok{50}\NormalTok{),}\DataTypeTok{ylim=}\KeywordTok{c}\NormalTok{(}\DecValTok{4}\NormalTok{,}\DecValTok{12}\NormalTok{),}\DataTypeTok{xlab=}\StringTok{'TV'}\NormalTok{,}\DataTypeTok{ylab=}\StringTok{'Sales'}\NormalTok{)}
\KeywordTok{abline}\NormalTok{(}\DataTypeTok{b=}\NormalTok{fit.lm}\OperatorTok{$}\NormalTok{coefficients[}\DecValTok{2}\NormalTok{],}\DataTypeTok{a=}\NormalTok{fit.lm}\OperatorTok{$}\NormalTok{coefficients[}\DecValTok{1}\NormalTok{],}\DataTypeTok{lwd=}\DecValTok{3}\NormalTok{)}
\ControlFlowTok{for}\NormalTok{(i }\ControlFlowTok{in} \DecValTok{1}\OperatorTok{:}\KeywordTok{length}\NormalTok{(xgrid_pred))\{}

\KeywordTok{segments}\NormalTok{(xgrid_pred[i]}\OperatorTok{-}\DecValTok{1}\NormalTok{,Bonf[i,}\DecValTok{2}\NormalTok{],xgrid_pred[i]}\OperatorTok{-}\DecValTok{1}\NormalTok{,Bonf[i,}\DecValTok{3}\NormalTok{],}\DataTypeTok{lwd=}\DecValTok{3}\NormalTok{,}\DataTypeTok{col=}\StringTok{'green'}\NormalTok{)}
\KeywordTok{segments}\NormalTok{(xgrid_pred[i]}\OperatorTok{-}\DecValTok{1}\NormalTok{,pointwise[i,}\DecValTok{2}\NormalTok{],xgrid_pred[i]}\OperatorTok{-}\DecValTok{1}\NormalTok{,pointwise[i,}\DecValTok{3}\NormalTok{],}\DataTypeTok{lwd=}\DecValTok{3}\NormalTok{,}\DataTypeTok{col=}\StringTok{'red'}\NormalTok{)}

\NormalTok{\}}
\KeywordTok{lines}\NormalTok{(ylb}\OperatorTok{~}\NormalTok{xdense,}\DataTypeTok{col=}\StringTok{'blue'}\NormalTok{,}\DataTypeTok{lwd=}\DecValTok{3}\NormalTok{)}

\KeywordTok{lines}\NormalTok{(yub}\OperatorTok{~}\NormalTok{xdense,}\DataTypeTok{col=}\StringTok{'blue'}\NormalTok{,}\DataTypeTok{lwd=}\DecValTok{3}\NormalTok{)}
\KeywordTok{legend}\NormalTok{(}\DataTypeTok{x=}\DecValTok{32}\NormalTok{,}\DataTypeTok{y=}\DecValTok{6}\NormalTok{,}\DataTypeTok{legend=}\KeywordTok{c}\NormalTok{(}\StringTok{"Pointwise"}\NormalTok{, }\StringTok{"Bonferroni"}\NormalTok{, }\StringTok{"W-H band"}\NormalTok{), }\DataTypeTok{lwd=}\DecValTok{3}\NormalTok{, }\DataTypeTok{col=}\KeywordTok{c}\NormalTok{(}\StringTok{"red"}\NormalTok{,}\StringTok{"green"}\NormalTok{,}\StringTok{"blue"}\NormalTok{))}
\end{Highlighting}
\end{Shaded}

\chapter{Statistical inference: Hypothesis testing}\label{ch:test}

Reading materials: Slides 73 - 91 in STA108\_LinearRegression\_S20.pdf.

We will continue working on synthetic data in this chapter.

\begin{Shaded}
\begin{Highlighting}[]
\KeywordTok{set.seed}\NormalTok{(}\DecValTok{1}\NormalTok{)}
\NormalTok{n=}\DecValTok{50}\NormalTok{;}
\NormalTok{x=}\KeywordTok{as.matrix}\NormalTok{(}\KeywordTok{rnorm}\NormalTok{(n,}\DataTypeTok{mean=}\DecValTok{10}\NormalTok{,}\DataTypeTok{sd=}\DecValTok{2}\NormalTok{),}\DataTypeTok{ncol=}\NormalTok{n);}
\NormalTok{beta.true=}\KeywordTok{c}\NormalTok{(}\DecValTok{20}\NormalTok{,}\FloatTok{0.15}\NormalTok{)}
\NormalTok{Ey=}\StringTok{ }\NormalTok{x}\OperatorTok{*}\NormalTok{beta.true[}\DecValTok{2}\NormalTok{]}\OperatorTok{+}\NormalTok{beta.true[}\DecValTok{1}\NormalTok{];}
\NormalTok{error.terms=}\StringTok{ }\KeywordTok{rnorm}\NormalTok{(n)}\OperatorTok{*}\DecValTok{2}\NormalTok{; }\CommentTok{# Normal errors, in order to verify the performance of the t-test }
\CommentTok{# You can change it to other distributions }
\NormalTok{y=Ey}\OperatorTok{+}\NormalTok{error.terms;}
\end{Highlighting}
\end{Shaded}

\section{Hypothesit testing}\label{hypothesit-testing}

\begin{Shaded}
\begin{Highlighting}[]
\CommentTok{# The lm() and summary() functions will automatically calculate the test statistics and p-values for you }
\NormalTok{fit.lm<-}\KeywordTok{lm}\NormalTok{(y}\OperatorTok{~}\NormalTok{x}\OperatorTok{+}\DecValTok{1}\NormalTok{);fit.lm.summary<-}\KeywordTok{summary}\NormalTok{(fit.lm)}
\NormalTok{fit.lm.summary}
\end{Highlighting}
\end{Shaded}

\begin{Shaded}
\begin{Highlighting}[]
\CommentTok{# We will create our own function for calculating the "t value"}
\KeywordTok{source}\NormalTok{(}\StringTok{'sources.r'}\NormalTok{) }\CommentTok{# load the functions we have written }

\NormalTok{beta.hat=}\KeywordTok{fit.linear.model}\NormalTok{(}\DataTypeTok{covariate=}\NormalTok{x,}\DataTypeTok{outcome=}\NormalTok{y)}
\NormalTok{beta.hat.sd=}\KeywordTok{estimate.coef.sd}\NormalTok{(}\DataTypeTok{beta=}\NormalTok{beta.hat,}\DataTypeTok{covariate=}\NormalTok{x,}\DataTypeTok{outcome=}\NormalTok{y)}

\CommentTok{# The test statistics for H0: beta1=0, and H0: beta0=0 are }
\NormalTok{beta.hat.t =}\StringTok{ }\NormalTok{(beta.hat}\OperatorTok{-}\DecValTok{0}\NormalTok{)}\OperatorTok{/}\NormalTok{beta.hat.sd;}

\NormalTok{calculate.t<-}\ControlFlowTok{function}\NormalTok{(covariate,outcome)\{}
\NormalTok{  beta.hat=}\KeywordTok{fit.linear.model}\NormalTok{(}\DataTypeTok{covariate=}\NormalTok{covariate,}\DataTypeTok{outcome=}\NormalTok{outcome)}
\NormalTok{  beta.hat.sd=}\KeywordTok{estimate.coef.sd}\NormalTok{(}\DataTypeTok{beta=}\NormalTok{beta.hat,}\DataTypeTok{covariate=}\NormalTok{covariate,}\DataTypeTok{outcome=}\NormalTok{outcome)}
\NormalTok{  beta.hat.t =}\StringTok{ }\NormalTok{(beta.hat}\OperatorTok{-}\DecValTok{0}\NormalTok{)}\OperatorTok{/}\NormalTok{beta.hat.sd;}
  \KeywordTok{return}\NormalTok{(beta.hat.t)}
\NormalTok{\}}

\CommentTok{# The p-values using the Student t distribution are }
\DecValTok{2}\OperatorTok{*}\KeywordTok{apply}\NormalTok{(}\KeywordTok{cbind}\NormalTok{(}\KeywordTok{pt}\NormalTok{(}\OperatorTok{-}\KeywordTok{abs}\NormalTok{(beta.hat.t),}\DataTypeTok{df=}\NormalTok{n}\OperatorTok{-}\DecValTok{2}\NormalTok{),(}\DecValTok{1}\OperatorTok{-}\KeywordTok{pt}\NormalTok{(}\KeywordTok{abs}\NormalTok{(beta.hat.t),}\DataTypeTok{df=}\NormalTok{n}\OperatorTok{-}\DecValTok{2}\NormalTok{))),}\DecValTok{1}\NormalTok{,min)}

\CommentTok{# We can also calculate the p-values using CLT}
\DecValTok{2}\OperatorTok{*}\KeywordTok{apply}\NormalTok{(}\KeywordTok{cbind}\NormalTok{(}\KeywordTok{pnorm}\NormalTok{(}\OperatorTok{-}\KeywordTok{abs}\NormalTok{(beta.hat.t)),(}\DecValTok{1}\OperatorTok{-}\KeywordTok{pnorm}\NormalTok{(}\KeywordTok{abs}\NormalTok{(beta.hat.t)))),}\DecValTok{1}\NormalTok{,min)}

\CommentTok{# Or to use the bootstrap }
\CommentTok{# This part is NOT required}
\CommentTok{# It is recommended to read more about bootstrap if you want to use it for hypothesis testing}
\CommentTok{# since the empirical bootstrap procedure is not the most efficient one for this purpose }

\NormalTok{beta.hat.boot=}\KeywordTok{replicate}\NormalTok{(}\FloatTok{1e5}\NormalTok{,}\KeywordTok{boot.fit}\NormalTok{(}\DataTypeTok{covariate=}\NormalTok{x,}\DataTypeTok{outcome=}\NormalTok{y));}
\NormalTok{pval=}\KeywordTok{numeric}\NormalTok{(}\KeywordTok{length}\NormalTok{(beta.hat));}
\ControlFlowTok{for}\NormalTok{(i }\ControlFlowTok{in} \DecValTok{1}\OperatorTok{:}\KeywordTok{length}\NormalTok{(beta.hat))\{}
\NormalTok{  boot.est=beta.hat.boot[}\DecValTok{1}\NormalTok{,i,]}
\NormalTok{  pval[i]=}\DecValTok{2}\OperatorTok{*}\KeywordTok{min}\NormalTok{( }\KeywordTok{mean}\NormalTok{(}\DecValTok{0}\OperatorTok{<}\NormalTok{boot.est), }\KeywordTok{mean}\NormalTok{(}\DecValTok{0}\OperatorTok{>}\NormalTok{boot.est) )}
\NormalTok{\}}

\NormalTok{## Wrap up the above code into one function}
\NormalTok{calculate.pvalue<-}\ControlFlowTok{function}\NormalTok{(covariate,outcome,type)\{}
\NormalTok{  beta.hat.t=}\KeywordTok{calculate.t}\NormalTok{(covariate,outcome);}
  \ControlFlowTok{if}\NormalTok{ (type}\OperatorTok{==}\StringTok{'t'}\NormalTok{)\{}
\NormalTok{    n=}\KeywordTok{length}\NormalTok{(outcome);}
\NormalTok{    pval=}\DecValTok{2}\OperatorTok{*}\KeywordTok{apply}\NormalTok{(}\KeywordTok{cbind}\NormalTok{(}\KeywordTok{pt}\NormalTok{(beta.hat.t,}\DataTypeTok{df=}\NormalTok{n}\OperatorTok{-}\DecValTok{2}\NormalTok{),(}\DecValTok{1}\OperatorTok{-}\KeywordTok{pt}\NormalTok{(beta.hat.t,}\DataTypeTok{df=}\NormalTok{n}\OperatorTok{-}\DecValTok{2}\NormalTok{))),}\DecValTok{1}\NormalTok{,min);}

\NormalTok{  \}}\ControlFlowTok{else} \ControlFlowTok{if}\NormalTok{ (type}\OperatorTok{==}\StringTok{'z'}\NormalTok{)\{}
\NormalTok{    pval=}\DecValTok{2}\OperatorTok{*}\KeywordTok{apply}\NormalTok{(}\KeywordTok{cbind}\NormalTok{(}\KeywordTok{pnorm}\NormalTok{(}\OperatorTok{-}\KeywordTok{abs}\NormalTok{(beta.hat.t)),(}\DecValTok{1}\OperatorTok{-}\KeywordTok{pnorm}\NormalTok{(}\KeywordTok{abs}\NormalTok{(beta.hat.t)))),}\DecValTok{1}\NormalTok{,min);}
    
\NormalTok{  \}}\ControlFlowTok{else} \ControlFlowTok{if}\NormalTok{ (type }\OperatorTok{==}\StringTok{ 'bootstrap'}\NormalTok{)\{}
    
\NormalTok{    beta.hat=}\KeywordTok{fit.linear.model}\NormalTok{(}\DataTypeTok{covariate=}\NormalTok{covariate,}\DataTypeTok{outcome=}\NormalTok{outcome)}
\NormalTok{    beta.hat.boot=}\KeywordTok{replicate}\NormalTok{(}\FloatTok{1e5}\NormalTok{,}\KeywordTok{boot.fit}\NormalTok{(}\DataTypeTok{covariate=}\NormalTok{covariate,}\DataTypeTok{outcome=}\NormalTok{outcome));}
\NormalTok{    pval=}\KeywordTok{numeric}\NormalTok{(}\KeywordTok{length}\NormalTok{(beta.hat));}
    \ControlFlowTok{for}\NormalTok{(i }\ControlFlowTok{in} \DecValTok{1}\OperatorTok{:}\KeywordTok{length}\NormalTok{(beta.hat))\{}
\NormalTok{      boot.est=beta.hat.boot[}\DecValTok{1}\NormalTok{,i,]}
\NormalTok{      pval[i]=}\DecValTok{2}\OperatorTok{*}\KeywordTok{min}\NormalTok{( }\KeywordTok{mean}\NormalTok{(}\DecValTok{0}\OperatorTok{<}\NormalTok{boot.est), }\KeywordTok{mean}\NormalTok{(}\DecValTok{0}\OperatorTok{>}\NormalTok{boot.est) )}
\NormalTok{    \}}
\NormalTok{  \}}
  \KeywordTok{return}\NormalTok{(pval)}
\NormalTok{\}}
\end{Highlighting}
\end{Shaded}

\begin{Shaded}
\begin{Highlighting}[]
\NormalTok{## Write a simulation to verify the equivalence between type I error rate and significance level, under the null H0: beta1=0 (slope is zero)}
\NormalTok{alpha=}\FloatTok{0.03}\NormalTok{; }\CommentTok{# significance level }
\NormalTok{simulate.one.instance<-}\ControlFlowTok{function}\NormalTok{(x,beta.null,alpha,type)\{}
\NormalTok{  n=}\KeywordTok{length}\NormalTok{(x);}
\NormalTok{  Ey=}\StringTok{ }\NormalTok{x}\OperatorTok{*}\NormalTok{beta.null[}\DecValTok{2}\NormalTok{]}\OperatorTok{+}\NormalTok{beta.null[}\DecValTok{1}\NormalTok{];}
\NormalTok{  error.terms=}\StringTok{ }\KeywordTok{rnorm}\NormalTok{(n)}\OperatorTok{*}\DecValTok{2}\NormalTok{;}
\NormalTok{  y=Ey}\OperatorTok{+}\NormalTok{error.terms;}
\NormalTok{  pval=}\KeywordTok{calculate.pvalue}\NormalTok{(}\DataTypeTok{covariate=}\NormalTok{x,}\DataTypeTok{outcome=}\NormalTok{y,}\DataTypeTok{type=}\NormalTok{type);}
\NormalTok{  rej.flag=}\StringTok{ }\NormalTok{pval[}\DecValTok{2}\NormalTok{]}\OperatorTok{<}\NormalTok{alpha}
  \KeywordTok{return}\NormalTok{(rej.flag)}
\NormalTok{\}}

\NormalTok{N.sim=}\FloatTok{1e4}\NormalTok{;}
\KeywordTok{set.seed}\NormalTok{(}\DecValTok{1}\NormalTok{)}
\NormalTok{n=}\DecValTok{50}\NormalTok{;}
\NormalTok{x=}\KeywordTok{as.matrix}\NormalTok{(}\KeywordTok{rnorm}\NormalTok{(n,}\DataTypeTok{mean=}\DecValTok{10}\NormalTok{,}\DataTypeTok{sd=}\DecValTok{2}\NormalTok{),}\DataTypeTok{ncol=}\NormalTok{n);}
\NormalTok{beta.null=}\KeywordTok{c}\NormalTok{(beta.true[}\DecValTok{1}\NormalTok{],}\DecValTok{0}\NormalTok{)}
\NormalTok{sim.typeI=}\KeywordTok{replicate}\NormalTok{(N.sim,}\KeywordTok{simulate.one.instance}\NormalTok{(x,beta.null,alpha,}\DataTypeTok{type=}\StringTok{'t'}\NormalTok{));}
\KeywordTok{mean}\NormalTok{(sim.typeI)}
\end{Highlighting}
\end{Shaded}

\begin{Shaded}
\begin{Highlighting}[]
\NormalTok{## Assess the power }
\NormalTok{N.sim=}\FloatTok{1e4}\NormalTok{;}
\KeywordTok{set.seed}\NormalTok{(}\DecValTok{1}\NormalTok{)}
\NormalTok{n=}\DecValTok{50}\NormalTok{;}
\NormalTok{x=}\KeywordTok{as.matrix}\NormalTok{(}\KeywordTok{rnorm}\NormalTok{(n,}\DataTypeTok{mean=}\DecValTok{10}\NormalTok{,}\DataTypeTok{sd=}\DecValTok{2}\NormalTok{),}\DataTypeTok{ncol=}\NormalTok{n);}
\NormalTok{powers<-}\KeywordTok{numeric}\NormalTok{(}\DecValTok{3}\NormalTok{)}
\ControlFlowTok{for}\NormalTok{(a }\ControlFlowTok{in} \DecValTok{1}\OperatorTok{:}\DecValTok{3}\NormalTok{)\{}
\NormalTok{  beta.alt=}\KeywordTok{c}\NormalTok{(beta.true[}\DecValTok{1}\NormalTok{],a}\OperatorTok{/}\DecValTok{10}\NormalTok{)}
\NormalTok{  sim.rej=}\KeywordTok{replicate}\NormalTok{(N.sim,}\KeywordTok{simulate.one.instance}\NormalTok{(x,beta.alt,alpha,}\DataTypeTok{type=}\StringTok{'t'}\NormalTok{));}
\NormalTok{  powers[a]<-}\KeywordTok{mean}\NormalTok{(sim.rej)}
  
\NormalTok{\}}
\NormalTok{powers}
\end{Highlighting}
\end{Shaded}

\begin{Shaded}
\begin{Highlighting}[]
\NormalTok{## Permutation test for testing H0: beta1 = 0 (slope is zero)}

\NormalTok{permutation.test<-}\ControlFlowTok{function}\NormalTok{(covariate,outcome)\{}
\NormalTok{  n=}\KeywordTok{length}\NormalTok{(outcome);}
\NormalTok{  sample_indices =}\StringTok{ }\KeywordTok{sample}\NormalTok{(}\DecValTok{1}\OperatorTok{:}\NormalTok{n,n,}\DataTypeTok{replace=}\OtherTok{FALSE}\NormalTok{) }\CommentTok{# sampling without replacement}
\NormalTok{  covariate.perm=}\StringTok{ }\NormalTok{covariate[sample_indices]; outcome.perm=}\StringTok{ }\NormalTok{outcome[sample_indices];}

\NormalTok{  beta.hat.t=}\KeywordTok{calculate.t}\NormalTok{(covariate.perm,outcome.perm)}

  \KeywordTok{return}\NormalTok{(beta.hat.t[}\DecValTok{2}\NormalTok{])}
\NormalTok{\}}


\KeywordTok{set.seed}\NormalTok{(}\DecValTok{1}\NormalTok{)}
\NormalTok{n=}\DecValTok{50}\NormalTok{;}
\NormalTok{x=}\KeywordTok{as.matrix}\NormalTok{(}\KeywordTok{rnorm}\NormalTok{(n,}\DataTypeTok{mean=}\DecValTok{10}\NormalTok{,}\DataTypeTok{sd=}\DecValTok{2}\NormalTok{),}\DataTypeTok{ncol=}\NormalTok{n);}
\NormalTok{beta.true=}\KeywordTok{c}\NormalTok{(}\DecValTok{20}\NormalTok{,}\FloatTok{0.15}\NormalTok{)}
\NormalTok{Ey=}\StringTok{ }\NormalTok{x}\OperatorTok{*}\NormalTok{beta.true[}\DecValTok{2}\NormalTok{]}\OperatorTok{+}\NormalTok{beta.true[}\DecValTok{1}\NormalTok{];}
\NormalTok{error.terms=}\StringTok{ }\KeywordTok{rnorm}\NormalTok{(n)}\OperatorTok{*}\DecValTok{2}\NormalTok{; }\CommentTok{# Normal errors, in order to verify the performance of the t-test }
\CommentTok{# You can change it to other distributions }
\NormalTok{y=Ey}\OperatorTok{+}\NormalTok{error.terms;}

\NormalTok{beta.hat.t.perm=}\KeywordTok{replicate}\NormalTok{(}\FloatTok{1e5}\NormalTok{,}\KeywordTok{permutation.test}\NormalTok{(}\DataTypeTok{covariate=}\NormalTok{x,}\DataTypeTok{outcome=}\NormalTok{y));}
\NormalTok{beta.hat.t=}\KeywordTok{calculate.t}\NormalTok{(x,y)}
\DecValTok{2}\OperatorTok{*}\KeywordTok{min}\NormalTok{( }\KeywordTok{mean}\NormalTok{(beta.hat.t[}\DecValTok{2}\NormalTok{]}\OperatorTok{<}\NormalTok{beta.hat.t.perm),}\KeywordTok{mean}\NormalTok{(beta.hat.t[}\DecValTok{2}\NormalTok{]}\OperatorTok{>}\NormalTok{beta.hat.t.perm))}
\end{Highlighting}
\end{Shaded}

\begin{Shaded}
\begin{Highlighting}[]
\NormalTok{## F-test can be done using the anova() function in R}
\NormalTok{## It will be left as an exercise for you to write your version of F test}
\KeywordTok{set.seed}\NormalTok{(}\DecValTok{1}\NormalTok{)}
\NormalTok{n=}\DecValTok{50}\NormalTok{;}
\NormalTok{x=}\KeywordTok{as.matrix}\NormalTok{(}\KeywordTok{rnorm}\NormalTok{(n,}\DataTypeTok{mean=}\DecValTok{10}\NormalTok{,}\DataTypeTok{sd=}\DecValTok{2}\NormalTok{),}\DataTypeTok{ncol=}\NormalTok{n);}
\NormalTok{beta.true=}\KeywordTok{c}\NormalTok{(}\DecValTok{20}\NormalTok{,}\FloatTok{0.15}\NormalTok{)}
\NormalTok{Ey=}\StringTok{ }\NormalTok{x}\OperatorTok{*}\NormalTok{beta.true[}\DecValTok{2}\NormalTok{]}\OperatorTok{+}\NormalTok{beta.true[}\DecValTok{1}\NormalTok{];}
\NormalTok{error.terms=}\StringTok{ }\KeywordTok{rnorm}\NormalTok{(n)}\OperatorTok{*}\DecValTok{2}\NormalTok{; }\CommentTok{# Normal errors, in order to verify the performance of the t-test }
\CommentTok{# You can change it to other distributions }
\NormalTok{y=Ey}\OperatorTok{+}\NormalTok{error.terms;}

\NormalTok{full_model=}\KeywordTok{lm}\NormalTok{(y}\OperatorTok{~}\NormalTok{x}\OperatorTok{+}\DecValTok{1}\NormalTok{);}
\NormalTok{reduced_model=}\KeywordTok{lm}\NormalTok{(y}\OperatorTok{~}\DecValTok{1}\NormalTok{);}
\KeywordTok{anova}\NormalTok{(reduced_model,full_model)}
\KeywordTok{summary}\NormalTok{(}\KeywordTok{lm}\NormalTok{(y}\OperatorTok{~}\NormalTok{x}\OperatorTok{+}\DecValTok{1}\NormalTok{))}
\end{Highlighting}
\end{Shaded}

\section{Multiple testing}\label{multiple-testing}

\begin{Shaded}
\begin{Highlighting}[]
\CommentTok{# To discuss multiple testing, we generate 1000 outcomes, among which 100 are truly associated with the covariate }

\KeywordTok{set.seed}\NormalTok{(}\DecValTok{1}\NormalTok{)}
\NormalTok{n=}\DecValTok{50}\NormalTok{;m=}\DecValTok{1000}\NormalTok{;}
\NormalTok{x=}\KeywordTok{as.matrix}\NormalTok{(}\KeywordTok{rnorm}\NormalTok{(n,}\DataTypeTok{mean=}\DecValTok{10}\NormalTok{,}\DataTypeTok{sd=}\DecValTok{2}\NormalTok{),}\DataTypeTok{ncol=}\NormalTok{n);}
\NormalTok{Y=}\KeywordTok{matrix}\NormalTok{(}\DecValTok{0}\NormalTok{,}\DataTypeTok{nrow=}\NormalTok{n,}\DataTypeTok{ncol=}\NormalTok{m)}
\NormalTok{beta.true=}\KeywordTok{c}\NormalTok{(}\DecValTok{20}\NormalTok{,}\FloatTok{0.6}\NormalTok{)}
\ControlFlowTok{for}\NormalTok{(i }\ControlFlowTok{in} \DecValTok{1}\OperatorTok{:}\NormalTok{m)\{}
\NormalTok{Ey=}\StringTok{ }\NormalTok{x}\OperatorTok{*}\NormalTok{beta.true[}\DecValTok{2}\NormalTok{]}\OperatorTok{*}\NormalTok{(i}\OperatorTok{<}\DecValTok{101}\NormalTok{)}\OperatorTok{+}\NormalTok{beta.true[}\DecValTok{1}\NormalTok{];}
\NormalTok{error.terms=}\StringTok{ }\KeywordTok{rnorm}\NormalTok{(n)}\OperatorTok{*}\DecValTok{2}\NormalTok{;  }
\NormalTok{Y[,i]=Ey}\OperatorTok{+}\NormalTok{error.terms;}
\NormalTok{\}}

\CommentTok{# Conduct the same test }
\NormalTok{pvalues=}\KeywordTok{numeric}\NormalTok{(m)}
\ControlFlowTok{for}\NormalTok{ (i }\ControlFlowTok{in} \DecValTok{1}\OperatorTok{:}\NormalTok{m)\{}
\NormalTok{  pval=}\KeywordTok{calculate.pvalue}\NormalTok{(}\DataTypeTok{covariate=}\NormalTok{x,}\DataTypeTok{outcome=}\NormalTok{Y[,i],}\DataTypeTok{type=}\StringTok{'z'}\NormalTok{);}
\NormalTok{  pvalues[i]<-pval[}\DecValTok{2}\NormalTok{];}
\NormalTok{\}}
\end{Highlighting}
\end{Shaded}

\begin{Shaded}
\begin{Highlighting}[]
\NormalTok{## Bonferroni correction }
\NormalTok{alpha=}\FloatTok{0.05}\NormalTok{;}
\NormalTok{alpha.Bonf=alpha}\OperatorTok{/}\NormalTok{m; }\CommentTok{# devided by the number of hypotheses}
\NormalTok{rej.flag=pvalues}\OperatorTok{<}\NormalTok{alpha.Bonf;}

\NormalTok{## True positive}
\KeywordTok{sum}\NormalTok{(rej.flag }\OperatorTok{&}\StringTok{ }\NormalTok{((}\DecValTok{1}\OperatorTok{:}\NormalTok{m)}\OperatorTok{<}\DecValTok{101}\NormalTok{))}
\NormalTok{## False positive }
\KeywordTok{sum}\NormalTok{(rej.flag }\OperatorTok{&}\StringTok{ }\NormalTok{((}\DecValTok{1}\OperatorTok{:}\NormalTok{m)}\OperatorTok{>}\DecValTok{100}\NormalTok{))}
\NormalTok{## True negative}
\KeywordTok{sum}\NormalTok{(}\OperatorTok{!}\NormalTok{rej.flag }\OperatorTok{&}\StringTok{ }\NormalTok{((}\DecValTok{1}\OperatorTok{:}\NormalTok{m)}\OperatorTok{>}\DecValTok{100}\NormalTok{))}

\NormalTok{## False negative}
\KeywordTok{sum}\NormalTok{(}\OperatorTok{!}\NormalTok{rej.flag }\OperatorTok{&}\StringTok{ }\NormalTok{((}\DecValTok{1}\OperatorTok{:}\NormalTok{m)}\OperatorTok{<}\DecValTok{101}\NormalTok{))}
\end{Highlighting}
\end{Shaded}

\begin{Shaded}
\begin{Highlighting}[]
\NormalTok{## Benjamini–Hochberg procedure for FDR control }

\NormalTok{pvalues.sorted<-}\KeywordTok{sort}\NormalTok{(pvalues,}\DataTypeTok{index.return=}\NormalTok{T)}
\NormalTok{alpha.BH<-}\StringTok{ }\NormalTok{(}\DecValTok{1}\OperatorTok{:}\NormalTok{m)}\OperatorTok{*}\NormalTok{alpha}\OperatorTok{/}\NormalTok{m;}
\NormalTok{k.hat<-}\KeywordTok{max}\NormalTok{(}\KeywordTok{which}\NormalTok{(pvalues.sorted}\OperatorTok{$}\NormalTok{x}\OperatorTok{<}\NormalTok{alpha.BH))}

\NormalTok{rej.flag.BH=}\KeywordTok{numeric}\NormalTok{(m);}
\NormalTok{rej.flag.BH[pvalues.sorted}\OperatorTok{$}\NormalTok{ix[}\DecValTok{1}\OperatorTok{:}\NormalTok{k.hat]]=}\DecValTok{1}\NormalTok{;}



\NormalTok{## True positive}
\KeywordTok{sum}\NormalTok{(rej.flag.BH }\OperatorTok{&}\StringTok{ }\NormalTok{((}\DecValTok{1}\OperatorTok{:}\NormalTok{m)}\OperatorTok{<}\DecValTok{101}\NormalTok{))}
\NormalTok{## False positive }
\KeywordTok{sum}\NormalTok{(rej.flag.BH }\OperatorTok{&}\StringTok{ }\NormalTok{((}\DecValTok{1}\OperatorTok{:}\NormalTok{m)}\OperatorTok{>}\DecValTok{100}\NormalTok{))}
\NormalTok{## True negative}
\KeywordTok{sum}\NormalTok{(}\OperatorTok{!}\NormalTok{rej.flag.BH }\OperatorTok{&}\StringTok{ }\NormalTok{((}\DecValTok{1}\OperatorTok{:}\NormalTok{m)}\OperatorTok{>}\DecValTok{100}\NormalTok{))}

\NormalTok{## False negative}
\KeywordTok{sum}\NormalTok{(}\OperatorTok{!}\NormalTok{rej.flag.BH }\OperatorTok{&}\StringTok{ }\NormalTok{((}\DecValTok{1}\OperatorTok{:}\NormalTok{m)}\OperatorTok{<}\DecValTok{101}\NormalTok{))}

\NormalTok{## False positive rate}

\KeywordTok{sum}\NormalTok{(rej.flag.BH }\OperatorTok{&}\StringTok{ }\NormalTok{((}\DecValTok{1}\OperatorTok{:}\NormalTok{m)}\OperatorTok{>}\DecValTok{100}\NormalTok{))}\OperatorTok{/}\KeywordTok{sum}\NormalTok{(rej.flag.BH)}

\CommentTok{# Run a simulation to verify this}
\end{Highlighting}
\end{Shaded}

\chapter{Model diagnostics}\label{ch:diagnostics}

Reading materials: Slides 92 - 100 in STA108\_LinearRegression\_S20.pdf.

Note: we will use existing functions in \texttt{R} for model diagnostics
in this chapter. However, in your midterm report, you are still required
to implement your own tools for model diagnostics.

\section{Residual plot}\label{residual-plot}

\begin{Shaded}
\begin{Highlighting}[]
\NormalTok{## We can extract residual from an `lm()` object.}

\NormalTok{fit.lm =}\StringTok{ }\KeywordTok{lm}\NormalTok{(sales}\OperatorTok{~}\NormalTok{TV}\OperatorTok{+}\DecValTok{1}\NormalTok{,}\DataTypeTok{data=}\NormalTok{dat.advertising); }\CommentTok{# Fit the linear regression}
\NormalTok{resid=}\StringTok{ }\NormalTok{fit.lm}\OperatorTok{$}\NormalTok{residuals;}

\KeywordTok{plot}\NormalTok{(resid}\OperatorTok{~}\NormalTok{dat.advertising}\OperatorTok{$}\NormalTok{TV,}\DataTypeTok{pch=}\DecValTok{16}\NormalTok{,}\DataTypeTok{col=}\StringTok{'red'}\NormalTok{,}\DataTypeTok{main=}\StringTok{'Residual plot'}\NormalTok{)}
\KeywordTok{abline}\NormalTok{(}\DataTypeTok{h=}\DecValTok{0}\NormalTok{,}\DataTypeTok{lwd=}\DecValTok{3}\NormalTok{)}
\end{Highlighting}
\end{Shaded}

Why don't we draw the residual plot as residuals v.s. the response? This
is because the two quantities are surely positively correlated, and thus
is hard to extract any useful information from the plot.

\begin{Shaded}
\begin{Highlighting}[]
\KeywordTok{plot}\NormalTok{(resid}\OperatorTok{~}\NormalTok{dat.advertising}\OperatorTok{$}\NormalTok{sales,}\DataTypeTok{pch=}\DecValTok{16}\NormalTok{,}\DataTypeTok{col=}\StringTok{'red'}\NormalTok{,}\DataTypeTok{main=}\StringTok{'Wrong plot'}\NormalTok{)}
\KeywordTok{abline}\NormalTok{(}\DataTypeTok{h=}\DecValTok{0}\NormalTok{,}\DataTypeTok{lwd=}\DecValTok{3}\NormalTok{)}
\end{Highlighting}
\end{Shaded}

How about drawing residuals against the fitted values? This is a good
choice to detect certain anomalies.

\begin{Shaded}
\begin{Highlighting}[]
\KeywordTok{par}\NormalTok{(}\DataTypeTok{mfrow=}\KeywordTok{c}\NormalTok{(}\DecValTok{1}\NormalTok{,}\DecValTok{2}\NormalTok{))}
\KeywordTok{plot}\NormalTok{(resid}\OperatorTok{~}\NormalTok{dat.advertising}\OperatorTok{$}\NormalTok{TV,}\DataTypeTok{pch=}\DecValTok{16}\NormalTok{,}\DataTypeTok{col=}\StringTok{'red'}\NormalTok{,}\DataTypeTok{main=}\StringTok{'Residual plot'}\NormalTok{)}
\KeywordTok{abline}\NormalTok{(}\DataTypeTok{h=}\DecValTok{0}\NormalTok{,}\DataTypeTok{lwd=}\DecValTok{3}\NormalTok{)}
\KeywordTok{plot}\NormalTok{(resid}\OperatorTok{~}\NormalTok{fit.lm}\OperatorTok{$}\NormalTok{fitted.values,}\DataTypeTok{pch=}\DecValTok{16}\NormalTok{,}\DataTypeTok{col=}\StringTok{'red'}\NormalTok{,}\DataTypeTok{main=}\StringTok{'Residual plot'}\NormalTok{,}\DataTypeTok{xlab=}\StringTok{'Fitted values'}\NormalTok{)}
\KeywordTok{abline}\NormalTok{(}\DataTypeTok{h=}\DecValTok{0}\NormalTok{,}\DataTypeTok{lwd=}\DecValTok{3}\NormalTok{)}
\end{Highlighting}
\end{Shaded}

\begin{Shaded}
\begin{Highlighting}[]
\KeywordTok{par}\NormalTok{(}\DataTypeTok{mfrow=}\KeywordTok{c}\NormalTok{(}\DecValTok{1}\NormalTok{,}\DecValTok{1}\NormalTok{))}
\end{Highlighting}
\end{Shaded}

\section{Remedies for non-linearity}\label{remedies-for-non-linearity}

From the residual plot, we see that there exist some non-linearity
between the residuals and the covariate, especially when the TV is
number is small. This could suggest that the relationship between TV and
sales is nonlinear. Typical form of nonlinearity takes form as
\(\exp(x)\), \(x^{1/2}\), \(\log(x)\), \(x^2\), etc. We may be able to
guess the nonlinearity from the residual plot, or use model selection to
pick the best nonlinear function, if there is not scientific knowledge
on the relationship. In this example, we will fit two regression
\(y\sim \log(x)\beta_1 + \beta_0\) and
\(y \sim x^{1/2} \beta_1 + \beta_0\).

\begin{Shaded}
\begin{Highlighting}[]
\NormalTok{fit.log =}\StringTok{ }\KeywordTok{lm}\NormalTok{(sales}\OperatorTok{~}\KeywordTok{log}\NormalTok{(TV)}\OperatorTok{+}\DecValTok{1}\NormalTok{,}\DataTypeTok{data=}\NormalTok{dat.advertising); }\CommentTok{# Fit the linear regression}
\NormalTok{dat.advertising}\OperatorTok{$}\NormalTok{sqrtTV=dat.advertising}\OperatorTok{$}\NormalTok{TV}\OperatorTok{^}\NormalTok{\{}\DecValTok{1}\OperatorTok{/}\DecValTok{2}\NormalTok{\};}
\NormalTok{fit.sqrt =}\StringTok{ }\KeywordTok{lm}\NormalTok{(sales}\OperatorTok{~}\NormalTok{sqrtTV}\OperatorTok{+}\DecValTok{1}\NormalTok{,}\DataTypeTok{data=}\NormalTok{dat.advertising); }\CommentTok{# Fit the linear regression}

\CommentTok{#par(mfrow=c(3,1))}
\KeywordTok{plot}\NormalTok{(fit.log}\OperatorTok{$}\NormalTok{residuals}\OperatorTok{~}\KeywordTok{log}\NormalTok{(dat.advertising}\OperatorTok{$}\NormalTok{TV),}\DataTypeTok{pch=}\DecValTok{16}\NormalTok{,}\DataTypeTok{col=}\StringTok{'red'}\NormalTok{,}\DataTypeTok{main=}\StringTok{'Residual plot (log TV)'}\NormalTok{,}\DataTypeTok{xlab=}\StringTok{'log(TV)'}\NormalTok{,}\DataTypeTok{ylab=}\StringTok{'Residuals'}\NormalTok{)}
\KeywordTok{abline}\NormalTok{(}\DataTypeTok{h=}\DecValTok{0}\NormalTok{,}\DataTypeTok{lwd=}\DecValTok{3}\NormalTok{)}
\end{Highlighting}
\end{Shaded}

\begin{Shaded}
\begin{Highlighting}[]
\KeywordTok{plot}\NormalTok{(fit.lm}\OperatorTok{$}\NormalTok{residuals}\OperatorTok{~}\NormalTok{dat.advertising}\OperatorTok{$}\NormalTok{TV,}\DataTypeTok{pch=}\DecValTok{16}\NormalTok{,}\DataTypeTok{col=}\StringTok{'red'}\NormalTok{,}\DataTypeTok{main=}\StringTok{'Residual plot (TV)'}\NormalTok{,}\DataTypeTok{xlab=}\StringTok{'TV'}\NormalTok{,}\DataTypeTok{ylab=}\StringTok{'Residuals'}\NormalTok{)}
\KeywordTok{abline}\NormalTok{(}\DataTypeTok{h=}\DecValTok{0}\NormalTok{,}\DataTypeTok{lwd=}\DecValTok{3}\NormalTok{)}
\end{Highlighting}
\end{Shaded}

\begin{Shaded}
\begin{Highlighting}[]
\KeywordTok{plot}\NormalTok{(fit.sqrt}\OperatorTok{$}\NormalTok{residuals}\OperatorTok{~}\NormalTok{dat.advertising}\OperatorTok{$}\NormalTok{sqrtTV,}\DataTypeTok{pch=}\DecValTok{16}\NormalTok{,}\DataTypeTok{col=}\StringTok{'red'}\NormalTok{,}\DataTypeTok{main=}\StringTok{'Residual plot (sq. rt. TV)'}\NormalTok{,}\DataTypeTok{xlab=}\StringTok{'sqrt(TV)'}\NormalTok{,}\DataTypeTok{ylab=}\StringTok{'Residuals'}\NormalTok{)}
\KeywordTok{abline}\NormalTok{(}\DataTypeTok{h=}\DecValTok{0}\NormalTok{,}\DataTypeTok{lwd=}\DecValTok{3}\NormalTok{)}
\end{Highlighting}
\end{Shaded}

\begin{Shaded}
\begin{Highlighting}[]
\CommentTok{#par(mfrow=c(1,1))}
\end{Highlighting}
\end{Shaded}

\section{Independence}\label{independence}

In the \texttt{advertising} data set, the residual plot does not tell us
whether certain data points are correlated or not.

\begin{Shaded}
\begin{Highlighting}[]
\KeywordTok{plot}\NormalTok{(fit.lm}\OperatorTok{$}\NormalTok{residuals}\OperatorTok{~}\NormalTok{dat.advertising}\OperatorTok{$}\NormalTok{TV,}\DataTypeTok{pch=}\DecValTok{16}\NormalTok{,}\DataTypeTok{col=}\StringTok{'red'}\NormalTok{,}\DataTypeTok{main=}\StringTok{'Residual plot'}\NormalTok{)}
\KeywordTok{abline}\NormalTok{(}\DataTypeTok{h=}\DecValTok{0}\NormalTok{,}\DataTypeTok{lwd=}\DecValTok{3}\NormalTok{)}
\end{Highlighting}
\end{Shaded}

We will consider a model where we know the errors are correlated, and
show that they may still be hard to recognized in practice. We consider
a model with autoregressive error, AR(1), which is widely used in
time-series data analysis (e.g., financial data). We consider a model
where, for \(i=1,2,\ldots, n\),
\[ y_i = x_i \beta_1 + \beta_0 + \epsilon_i,\] where
\(\epsilon_i = 3\epsilon_{i-1}/4 + z_i/4\) and
\(z_i\sim \mathcal{N}(0,1)\). Here \(i\) represents some unit of time
(e.g., months, years, days).

\begin{Shaded}
\begin{Highlighting}[]
\NormalTok{n=}\DecValTok{100}\NormalTok{;}
\KeywordTok{set.seed}\NormalTok{(}\DecValTok{1}\NormalTok{);}
\NormalTok{z=}\KeywordTok{rnorm}\NormalTok{(n);}
\NormalTok{epsilon=}\StringTok{ }\KeywordTok{numeric}\NormalTok{(n);}
\ControlFlowTok{for}\NormalTok{(i }\ControlFlowTok{in} \DecValTok{1}\OperatorTok{:}\NormalTok{n)\{}
\ControlFlowTok{if}\NormalTok{ (i}\OperatorTok{==}\DecValTok{1}\NormalTok{)\{}
\NormalTok{epsilon[i]=z[i];}

\NormalTok{\}}\ControlFlowTok{else}\NormalTok{\{}
\NormalTok{epsilon[i]=}\DecValTok{3}\OperatorTok{*}\NormalTok{epsilon[i}\OperatorTok{-}\DecValTok{1}\NormalTok{]}\OperatorTok{/}\DecValTok{4}\OperatorTok{+}\NormalTok{z[i]}\OperatorTok{/}\DecValTok{4}\NormalTok{;}
\NormalTok{\}}
\NormalTok{\}}
\NormalTok{x=}\KeywordTok{rnorm}\NormalTok{(n)}\OperatorTok{*}\DecValTok{10}\OperatorTok{+}\DecValTok{5}\NormalTok{;}
\NormalTok{y=x}\OperatorTok{*}\DecValTok{2}\OperatorTok{+}\DecValTok{1}\OperatorTok{+}\NormalTok{z;}

\NormalTok{ARfits =}\StringTok{ }\KeywordTok{lm}\NormalTok{(y}\OperatorTok{~}\NormalTok{x}\OperatorTok{+}\DecValTok{1}\NormalTok{);}


\KeywordTok{par}\NormalTok{(}\DataTypeTok{mfrow=}\KeywordTok{c}\NormalTok{(}\DecValTok{1}\NormalTok{,}\DecValTok{2}\NormalTok{))}
\KeywordTok{plot}\NormalTok{(ARfits}\OperatorTok{$}\NormalTok{residuals}\OperatorTok{~}\NormalTok{x,}\DataTypeTok{pch=}\DecValTok{16}\NormalTok{,}\DataTypeTok{col=}\StringTok{'red'}\NormalTok{,}\DataTypeTok{main=}\StringTok{'Residual plot 1'}\NormalTok{)}
\KeywordTok{abline}\NormalTok{(}\DataTypeTok{h=}\DecValTok{0}\NormalTok{,}\DataTypeTok{lwd=}\DecValTok{3}\NormalTok{)}

\KeywordTok{plot}\NormalTok{(ARfits}\OperatorTok{$}\NormalTok{residuals}\OperatorTok{~}\KeywordTok{c}\NormalTok{(}\DecValTok{1}\OperatorTok{:}\NormalTok{n),}\DataTypeTok{pch=}\DecValTok{16}\NormalTok{,}\DataTypeTok{col=}\StringTok{'red'}\NormalTok{,}\DataTypeTok{main=}\StringTok{'Residual plot 2'}\NormalTok{,}\DataTypeTok{xlab=}\StringTok{"Index i"}\NormalTok{)}
\KeywordTok{abline}\NormalTok{(}\DataTypeTok{h=}\DecValTok{0}\NormalTok{,}\DataTypeTok{lwd=}\DecValTok{3}\NormalTok{)}
\end{Highlighting}
\end{Shaded}

\begin{Shaded}
\begin{Highlighting}[]
\KeywordTok{par}\NormalTok{(}\DataTypeTok{mfrow=}\KeywordTok{c}\NormalTok{(}\DecValTok{1}\NormalTok{,}\DecValTok{1}\NormalTok{))}
\end{Highlighting}
\end{Shaded}

\section{Normality}\label{normality}

It is also hard to see normality based on the residual plot. We can use
a Quantile-Quantile plot to check if the errors are normally
distributed. In the Q-Q plot, we draw the quantiles of residuals against
the quantiles of the thoeretical quantiles from a normal distribution.
The \(100(i/n)\%\)th quantile of the residuals is defined as the \(i\)th
smallest residual.

Non-normal distributions in Q-Q plots.

\begin{Shaded}
\begin{Highlighting}[]
\NormalTok{n=}\DecValTok{500}\NormalTok{;}
\NormalTok{distributions=}\KeywordTok{matrix}\NormalTok{(}\DecValTok{0}\NormalTok{,n,}\DecValTok{4}\NormalTok{)}
\NormalTok{distributions[,}\DecValTok{1}\NormalTok{] =}\OperatorTok{-}\KeywordTok{exp}\NormalTok{(}\KeywordTok{rnorm}\NormalTok{(n));}
\NormalTok{distributions[,}\DecValTok{2}\NormalTok{] =}\KeywordTok{exp}\NormalTok{(}\KeywordTok{rnorm}\NormalTok{(n));}
\NormalTok{distributions[,}\DecValTok{3}\NormalTok{]=}\KeywordTok{rt}\NormalTok{(n,}\DataTypeTok{df=}\DecValTok{3}\NormalTok{);}
\NormalTok{distributions[,}\DecValTok{4}\NormalTok{]=}\KeywordTok{runif}\NormalTok{(n);}

\NormalTok{titles =}\StringTok{ }\KeywordTok{c}\NormalTok{(}\StringTok{'Left skewed'}\NormalTok{, }\StringTok{'Right skewed'}\NormalTok{, }\StringTok{'Heavy-tailed'}\NormalTok{, }\StringTok{'Light-tailed'}\NormalTok{ )}
\KeywordTok{par}\NormalTok{(}\DataTypeTok{mfrow=}\KeywordTok{c}\NormalTok{(}\DecValTok{2}\NormalTok{,}\DecValTok{2}\NormalTok{))}
\ControlFlowTok{for}\NormalTok{ ( i }\ControlFlowTok{in} \DecValTok{1}\OperatorTok{:}\DecValTok{4}\NormalTok{)\{}
\NormalTok{this_dist=}\KeywordTok{sort}\NormalTok{(distributions[,i]);}
\NormalTok{normal_mean =}\KeywordTok{mean}\NormalTok{(this_dist);normal_sd =}\StringTok{ }\KeywordTok{sd}\NormalTok{(this_dist);}
\NormalTok{this_dist=(this_dist }\OperatorTok{-}\StringTok{ }\NormalTok{normal_mean)}\OperatorTok{/}\NormalTok{normal_sd;}
\NormalTok{normal_quantiles =}\StringTok{ }\KeywordTok{qnorm}\NormalTok{( (}\DecValTok{1}\OperatorTok{:}\KeywordTok{length}\NormalTok{(this_dist))}\OperatorTok{/}\KeywordTok{length}\NormalTok{(this_dist));}

\KeywordTok{plot}\NormalTok{(this_dist}\OperatorTok{~}\NormalTok{normal_quantiles,}\DataTypeTok{pch=}\DecValTok{16}\NormalTok{,}\DataTypeTok{col=}\StringTok{'red'}\NormalTok{,}\DataTypeTok{main=}\NormalTok{titles[i],}\DataTypeTok{xlab=}\StringTok{'Normal quantiles'}\NormalTok{, }\DataTypeTok{ylab=}\StringTok{'Sample quantiles'}\NormalTok{,}\DataTypeTok{xlim=}\KeywordTok{c}\NormalTok{(}\OperatorTok{-}\DecValTok{4}\NormalTok{,}\DecValTok{4}\NormalTok{),}\DataTypeTok{ylim=}\KeywordTok{c}\NormalTok{(}\OperatorTok{-}\DecValTok{4}\NormalTok{,}\DecValTok{4}\NormalTok{))}
\KeywordTok{abline}\NormalTok{(}\DataTypeTok{a=}\DecValTok{0}\NormalTok{,}\DataTypeTok{b=}\DecValTok{1}\NormalTok{,}\DataTypeTok{lwd=}\DecValTok{3}\NormalTok{)}
\NormalTok{\}}
\end{Highlighting}
\end{Shaded}

\section{Homoscedasticity}\label{homoscedasticity}

We can see clearly that the residuals have wider spans when \texttt{TV}
is larger, which suggested increasing variance. We can stablize the
variance by transforming the response variable \(y^{1/2}\), \(\log(y)\),
etc. You can also use the Box-Cox transformation to find the most
appropriate function.

\begin{Shaded}
\begin{Highlighting}[]
\NormalTok{fit.ylog=}\KeywordTok{lm}\NormalTok{(}\KeywordTok{log}\NormalTok{(sales)}\OperatorTok{~}\NormalTok{TV}\OperatorTok{+}\DecValTok{1}\NormalTok{,}\DataTypeTok{data=}\NormalTok{dat.advertising)}
\NormalTok{fit.ysqrt=}\KeywordTok{lm}\NormalTok{(}\KeywordTok{sqrt}\NormalTok{(sales)}\OperatorTok{~}\NormalTok{TV}\OperatorTok{+}\DecValTok{1}\NormalTok{,}\DataTypeTok{data=}\NormalTok{dat.advertising)}
\CommentTok{#par(mfrow=c(3,1))}
\KeywordTok{plot}\NormalTok{(fit.lm}\OperatorTok{$}\NormalTok{residuals}\OperatorTok{~}\NormalTok{dat.advertising}\OperatorTok{$}\NormalTok{TV,}\DataTypeTok{pch=}\DecValTok{16}\NormalTok{,}\DataTypeTok{col=}\StringTok{'red'}\NormalTok{,}\DataTypeTok{main=}\StringTok{'Residual plot (original)'}\NormalTok{,}\DataTypeTok{xlab=}\StringTok{'TV'}\NormalTok{)}
\KeywordTok{abline}\NormalTok{(}\DataTypeTok{h=}\DecValTok{0}\NormalTok{,}\DataTypeTok{lwd=}\DecValTok{3}\NormalTok{)}
\end{Highlighting}
\end{Shaded}

\begin{Shaded}
\begin{Highlighting}[]
\KeywordTok{plot}\NormalTok{(fit.ylog}\OperatorTok{$}\NormalTok{residuals}\OperatorTok{~}\NormalTok{dat.advertising}\OperatorTok{$}\NormalTok{TV,}\DataTypeTok{pch=}\DecValTok{16}\NormalTok{,}\DataTypeTok{col=}\StringTok{'red'}\NormalTok{,}\DataTypeTok{main=}\StringTok{'Residual plot (log)'}\NormalTok{,}\DataTypeTok{ylab=}\StringTok{'residuals'}\NormalTok{,}\DataTypeTok{xlab=}\StringTok{'TV'}\NormalTok{)}
\KeywordTok{abline}\NormalTok{(}\DataTypeTok{h=}\DecValTok{0}\NormalTok{,}\DataTypeTok{lwd=}\DecValTok{3}\NormalTok{)}
\end{Highlighting}
\end{Shaded}

\begin{Shaded}
\begin{Highlighting}[]
\KeywordTok{plot}\NormalTok{(fit.ysqrt}\OperatorTok{$}\NormalTok{residuals}\OperatorTok{~}\NormalTok{dat.advertising}\OperatorTok{$}\NormalTok{TV,}\DataTypeTok{pch=}\DecValTok{16}\NormalTok{,}\DataTypeTok{col=}\StringTok{'red'}\NormalTok{,}\DataTypeTok{main=}\StringTok{'Residual plot (sqrt)'}\NormalTok{,}\DataTypeTok{ylab=}\StringTok{"residuals"}\NormalTok{,}\DataTypeTok{xlab=}\StringTok{'TV'}\NormalTok{)}
\KeywordTok{abline}\NormalTok{(}\DataTypeTok{h=}\DecValTok{0}\NormalTok{,}\DataTypeTok{lwd=}\DecValTok{3}\NormalTok{)}
\end{Highlighting}
\end{Shaded}

\section{Influential Observations and
Outliers}\label{influential-observations-and-outliers}

The influential observations are samples that has a large leverage. You
should search for a formal definition of leverage if you are interested.
In plain words, the inflential observations are data points that live
far away from others in terms of their values of the covariates.

Outliters are the observations whose responses are far away from
observations with similar covariates. We can see these from the usual
scatter plots. We will not cover formal testing or measures for
influential observations and outliers in this class.

\begin{Shaded}
\begin{Highlighting}[]
\NormalTok{influential_cov=dat.advertising}\OperatorTok{$}\NormalTok{TV;influential_cov[}\DecValTok{1}\NormalTok{]=}\DecValTok{500}\NormalTok{;}
\NormalTok{outliers_resp=fit.lm}\OperatorTok{$}\NormalTok{residuals;outliers_resp[}\DecValTok{1}\NormalTok{]=}\DecValTok{30}\NormalTok{;}

\KeywordTok{par}\NormalTok{(}\DataTypeTok{mfrow=}\KeywordTok{c}\NormalTok{(}\DecValTok{1}\NormalTok{,}\DecValTok{2}\NormalTok{))}
\KeywordTok{plot}\NormalTok{(fit.lm}\OperatorTok{$}\NormalTok{residuals}\OperatorTok{~}\NormalTok{influential_cov,}\DataTypeTok{pch=}\DecValTok{16}\NormalTok{,}\DataTypeTok{col=}\StringTok{'red'}\NormalTok{,}\DataTypeTok{main=}\StringTok{'Influential obs.'}\NormalTok{,}\DataTypeTok{xlab=}\StringTok{'TV'}\NormalTok{)}
\KeywordTok{abline}\NormalTok{(}\DataTypeTok{h=}\DecValTok{0}\NormalTok{,}\DataTypeTok{lwd=}\DecValTok{3}\NormalTok{)}
\KeywordTok{points}\NormalTok{(}\DataTypeTok{y=}\NormalTok{fit.lm}\OperatorTok{$}\NormalTok{residuals[}\DecValTok{1}\NormalTok{],}\DataTypeTok{x=}\NormalTok{influential_cov[}\DecValTok{1}\NormalTok{],}\DataTypeTok{col=}\StringTok{'green'}\NormalTok{,}\DataTypeTok{pch=}\DecValTok{16}\NormalTok{)}


\KeywordTok{plot}\NormalTok{(outliers_resp}\OperatorTok{~}\NormalTok{dat.advertising}\OperatorTok{$}\NormalTok{TV,}\DataTypeTok{pch=}\DecValTok{16}\NormalTok{,}\DataTypeTok{col=}\StringTok{'red'}\NormalTok{,}\DataTypeTok{main=}\StringTok{'Outlier'}\NormalTok{,}\DataTypeTok{xlab=}\StringTok{'TV'}\NormalTok{)}
\KeywordTok{abline}\NormalTok{(}\DataTypeTok{h=}\DecValTok{0}\NormalTok{,}\DataTypeTok{lwd=}\DecValTok{3}\NormalTok{)}
\KeywordTok{points}\NormalTok{(}\DataTypeTok{y=}\NormalTok{outliers_resp[}\DecValTok{1}\NormalTok{],}\DataTypeTok{x=}\NormalTok{dat.advertising}\OperatorTok{$}\NormalTok{TV[}\DecValTok{1}\NormalTok{],}\DataTypeTok{col=}\StringTok{'green'}\NormalTok{,}\DataTypeTok{pch=}\DecValTok{16}\NormalTok{)}
\end{Highlighting}
\end{Shaded}

\chapter{Multiple covariates}\label{ch:multiple}

\section{Examples}\label{examples}

Reading materials: Slides 101 - 110 in
STA108\_LinearRegression\_S20.pdf.

We will revisit the examples that have been studied in Chapter
\ref{ch:lmR}.

\subsection{Advertising data}\label{advertising-data-1}

We now consider all the covariates in the data set. The resulting model
is \[
    y_i =  \beta_0 + \sum_{j=1}^3 x_{i,j} \beta_j  +  \epsilon_i, i=1,\ldots, 200,
    \] where \(y_i\) is the \texttt{sales} (1 unit = 1000 dollars) for
the \(i\)th entry, \(x_{i,1}\) is the TV advertising budget for the
\(i\)th entry, \(x_{i,2}\) is the radio advertising budget for the
\(i\)th entry, \(x_{i,3}\) is the newspaper advertising budget for the
\(i\)th entry. In addition, we assume that the errors
\(\{\epsilon_i\}_{i=1}^{200}\) satisfy that
\(\epsilon_1,\ldots, \epsilon_200\) are independently and identically
distributed (i.i.d.), \(\mathbb{E}[\epsilon_i]= 0\) for
\(i=1,2,\ldots, 200\) and \(\mathrm{var}(\epsilon_i)=\sigma^2\) for
\(i=1,2,\ldots, 200\). Recall that we consider fixed design (i.e.,
\(x_i\) is not random) in this course for simplicity.

\begin{Shaded}
\begin{Highlighting}[]
\NormalTok{dat.advertising=}\KeywordTok{read.csv}\NormalTok{(}\StringTok{'./data/advertising.csv'}\NormalTok{);}
\CommentTok{# Fit a multiple linear regression}
\NormalTok{fit.advertising =}\StringTok{ }\KeywordTok{lm}\NormalTok{(sales}\OperatorTok{~}\NormalTok{TV}\OperatorTok{+}\NormalTok{radio}\OperatorTok{+}\NormalTok{newspaper}\OperatorTok{+}\DecValTok{1}\NormalTok{,}\DataTypeTok{data=}\NormalTok{dat.advertising); }
\CommentTok{# Summarize the fitted results}
\KeywordTok{summary}\NormalTok{(fit.advertising) }

\NormalTok{fit.advertising.slr =}\StringTok{ }\KeywordTok{lm}\NormalTok{(sales}\OperatorTok{~}\NormalTok{TV}\OperatorTok{+}\DecValTok{1}\NormalTok{,}\DataTypeTok{data=}\NormalTok{dat.advertising); }
\CommentTok{# Summarize the fitted results}
\KeywordTok{summary}\NormalTok{(fit.advertising.slr) }
\end{Highlighting}
\end{Shaded}

How would you interpret the fitted results? How do they compare with the
results from the simple linear regression in Chapter \ref{lmR}?

\subsection{Flu shot}\label{flu-shot-1}

We will include \texttt{age} and \texttt{female}(gender) in the
regression in addition to the treatment received. What can you conclude
from the fitted results?

\begin{Shaded}
\begin{Highlighting}[]
\NormalTok{dat.flu =}\StringTok{ }\KeywordTok{read.table}\NormalTok{(}\StringTok{"./data/flu240.txt"}\NormalTok{, }\DataTypeTok{header =} \OtherTok{TRUE}\NormalTok{)}
\CommentTok{# Fit a multiple linear regression}
\NormalTok{fit.flu=}\StringTok{ }\KeywordTok{lm}\NormalTok{(outcome}\OperatorTok{~}\NormalTok{treatment.received}\OperatorTok{+}\NormalTok{age}\OperatorTok{+}\NormalTok{female}\OperatorTok{+}\DecValTok{1}\NormalTok{,}\DataTypeTok{data=}\NormalTok{dat.flu); }
\CommentTok{# Summarize the fitted results}
\KeywordTok{summary}\NormalTok{(fit.flu) }
\end{Highlighting}
\end{Shaded}

In addition, we can fit a regression with treatment received as the
outcome, where treatment assigned, age and gender are covariates. What
can you conclude now?

\begin{Shaded}
\begin{Highlighting}[]
\NormalTok{fit.trt=}\StringTok{ }\KeywordTok{lm}\NormalTok{(treatment.received}\OperatorTok{~}\NormalTok{treatment.assigned}\OperatorTok{+}\NormalTok{age}\OperatorTok{+}\NormalTok{female}\OperatorTok{+}\DecValTok{1}\NormalTok{,}\DataTypeTok{data=}\NormalTok{dat.flu); }
\CommentTok{# Summarize the fitted results}
\KeywordTok{summary}\NormalTok{(fit.trt) }
\end{Highlighting}
\end{Shaded}

\subsection{Project STAR}\label{project-star-1}

Project STAR is a stratified randomized experiment, where randomization
happened within each school. Hence, we will add \texttt{schoolid2} as an
additional factor in the regression.

\begin{Shaded}
\begin{Highlighting}[]
\KeywordTok{library}\NormalTok{(AER)}
\KeywordTok{data}\NormalTok{(}\StringTok{"STAR"}\NormalTok{)}
\NormalTok{dat.STAR=STAR; }\CommentTok{# Just to be consistent}
\CommentTok{# Fit a simple linear regression}
\NormalTok{fit.STAR=}\StringTok{ }\KeywordTok{lm}\NormalTok{(math2}\OperatorTok{~}\KeywordTok{as.factor}\NormalTok{(star2)}\OperatorTok{+}\KeywordTok{as.factor}\NormalTok{(schoolid2)}\OperatorTok{+}\DecValTok{1}\NormalTok{,}\DataTypeTok{data=}\NormalTok{dat.STAR); }
\CommentTok{# Summarize the fitted results}
\KeywordTok{summary}\NormalTok{(fit.STAR)}\OperatorTok{$}\NormalTok{coef[}\DecValTok{1}\OperatorTok{:}\DecValTok{3}\NormalTok{,]}

\CommentTok{# We can no longer draw a fitted line here...}
\end{Highlighting}
\end{Shaded}

\section{Classification of variables}\label{classification-of-variables}

Reading materials: Slides 111 - 118 in
STA108\_LinearRegression\_S20.pdf.

We will use simulation to examine the roles of the variables.

\begin{Shaded}
\begin{Highlighting}[]
\KeywordTok{source}\NormalTok{(}\StringTok{'sources.r'}\NormalTok{)}

\NormalTok{simulate.one.instance.data<-}\ControlFlowTok{function}\NormalTok{()\{}
\NormalTok{n=}\DecValTok{100}\NormalTok{;}
\NormalTok{confounder =}\StringTok{ }\KeywordTok{rnorm}\NormalTok{(n)}\OperatorTok{*}\DecValTok{2}\NormalTok{;}
\NormalTok{precision =}\StringTok{ }\KeywordTok{runif}\NormalTok{(n);}
\NormalTok{instrument =}\StringTok{ }\KeywordTok{rbinom}\NormalTok{(n,}\DataTypeTok{size=}\DecValTok{1}\NormalTok{,}\DataTypeTok{prob=}\FloatTok{0.4}\NormalTok{);}
\NormalTok{trt=}\KeywordTok{rnorm}\NormalTok{(n) }\OperatorTok{+}\StringTok{ }\NormalTok{confounder}\OperatorTok{*}\FloatTok{0.4}\OperatorTok{-}\NormalTok{instrument;}
\NormalTok{y=trt}\OperatorTok{*}\DecValTok{1}\OperatorTok{-}\FloatTok{0.5}\OperatorTok{*}\NormalTok{confounder}\OperatorTok{+}\NormalTok{precision}\OperatorTok{*}\FloatTok{0.7}\OperatorTok{+}\FloatTok{0.5}\OperatorTok{*}\KeywordTok{rnorm}\NormalTok{(n);}
\NormalTok{out=}\KeywordTok{data.frame}\NormalTok{(}\DataTypeTok{y=}\NormalTok{y,}\DataTypeTok{trt=}\NormalTok{trt,}\DataTypeTok{confounder=}\NormalTok{confounder, }\DataTypeTok{precision=}\NormalTok{precision, }\DataTypeTok{instrument=}\NormalTok{instrument)}
\KeywordTok{return}\NormalTok{(out)}
\NormalTok{\}}
\end{Highlighting}
\end{Shaded}

\begin{Shaded}
\begin{Highlighting}[]
\NormalTok{## Effect if ignoring confounder:}

\NormalTok{simulate.one.instance<-}\ControlFlowTok{function}\NormalTok{()\{}
\NormalTok{  dat=}\KeywordTok{simulate.one.instance.data}\NormalTok{();}
\NormalTok{  beta.hat=}\KeywordTok{fit.linear.model}\NormalTok{(}\DataTypeTok{covariate=}\NormalTok{dat[,}\StringTok{'trt'}\NormalTok{],}\DataTypeTok{outcome=}\NormalTok{dat}\OperatorTok{$}\NormalTok{y)}
  
\NormalTok{  beta.hat.confounder=}\KeywordTok{fit.linear.model}\NormalTok{(}\DataTypeTok{covariate=}\KeywordTok{as.matrix}\NormalTok{(dat[,}\KeywordTok{c}\NormalTok{(}\StringTok{'trt'}\NormalTok{,}\StringTok{'confounder'}\NormalTok{)]),}\DataTypeTok{outcome=}\NormalTok{dat}\OperatorTok{$}\NormalTok{y)}
  \KeywordTok{return}\NormalTok{(}\KeywordTok{c}\NormalTok{(beta.hat[}\DecValTok{2}\NormalTok{],beta.hat.confounder[}\DecValTok{2}\NormalTok{]))}
\NormalTok{\}}
\NormalTok{N.sim=}\DecValTok{10000}\NormalTok{;}

\KeywordTok{set.seed}\NormalTok{(}\DecValTok{1}\NormalTok{)}

\NormalTok{sim.confounder=}\KeywordTok{replicate}\NormalTok{(N.sim, }\KeywordTok{simulate.one.instance}\NormalTok{());}
\KeywordTok{apply}\NormalTok{(sim.confounder,}\DecValTok{1}\NormalTok{,mean) }\CommentTok{# The true value is 1}
\KeywordTok{apply}\NormalTok{( (sim.confounder}\OperatorTok{-}\DecValTok{1}\NormalTok{),}\DecValTok{1}\NormalTok{,}\ControlFlowTok{function}\NormalTok{(x)\{ }\KeywordTok{sum}\NormalTok{(x}\OperatorTok{^}\DecValTok{2}\NormalTok{) \}) }\CommentTok{# The mean squared error}
\end{Highlighting}
\end{Shaded}

\begin{Shaded}
\begin{Highlighting}[]
\NormalTok{## Effect if ignoring precision variable:}

\NormalTok{simulate.one.instance<-}\ControlFlowTok{function}\NormalTok{()\{}
\NormalTok{  dat=}\KeywordTok{simulate.one.instance.data}\NormalTok{();}
\NormalTok{  beta.hat=}\KeywordTok{fit.linear.model}\NormalTok{(}\DataTypeTok{covariate=}\KeywordTok{as.matrix}\NormalTok{(dat[,}\KeywordTok{c}\NormalTok{(}\StringTok{'trt'}\NormalTok{,}\StringTok{'confounder'}\NormalTok{)]),}\DataTypeTok{outcome=}\NormalTok{dat}\OperatorTok{$}\NormalTok{y)}
  
\NormalTok{  beta.hat.precision=}\KeywordTok{fit.linear.model}\NormalTok{(}\DataTypeTok{covariate=}\KeywordTok{as.matrix}\NormalTok{(dat[,}\KeywordTok{c}\NormalTok{(}\StringTok{'trt'}\NormalTok{,}\StringTok{'confounder'}\NormalTok{,}\StringTok{'precision'}\NormalTok{)]),}\DataTypeTok{outcome=}\NormalTok{dat}\OperatorTok{$}\NormalTok{y)}
  \KeywordTok{return}\NormalTok{(}\KeywordTok{c}\NormalTok{(beta.hat[}\DecValTok{2}\NormalTok{],beta.hat.precision[}\DecValTok{2}\NormalTok{]))}
\NormalTok{\}}
\NormalTok{N.sim=}\DecValTok{10000}\NormalTok{;}

\KeywordTok{set.seed}\NormalTok{(}\DecValTok{1}\NormalTok{)}

\NormalTok{sim.precision=}\KeywordTok{replicate}\NormalTok{(N.sim, }\KeywordTok{simulate.one.instance}\NormalTok{());}
\KeywordTok{apply}\NormalTok{(sim.precision,}\DecValTok{1}\NormalTok{,mean) }\CommentTok{# The true value is 1}
\KeywordTok{apply}\NormalTok{( (sim.precision}\OperatorTok{-}\DecValTok{1}\NormalTok{),}\DecValTok{1}\NormalTok{,}\ControlFlowTok{function}\NormalTok{(x)\{ }\KeywordTok{sum}\NormalTok{(x}\OperatorTok{^}\DecValTok{2}\NormalTok{) \}) }\CommentTok{# The mean squared error}
\end{Highlighting}
\end{Shaded}

\begin{Shaded}
\begin{Highlighting}[]
\NormalTok{## The power of instruments}
\NormalTok{###  If the confounder is unobserved, we can use the two stage least squares method}

\NormalTok{simulate.one.instance<-}\ControlFlowTok{function}\NormalTok{()\{}
\NormalTok{  dat=}\KeywordTok{simulate.one.instance.data}\NormalTok{();}
\NormalTok{  beta.hat.naive=}\KeywordTok{fit.linear.model}\NormalTok{(}\DataTypeTok{covariate=}\KeywordTok{as.matrix}\NormalTok{(dat[,}\StringTok{'trt'}\NormalTok{]),}\DataTypeTok{outcome=}\NormalTok{dat}\OperatorTok{$}\NormalTok{y)}
  
\NormalTok{  yiv=}\KeywordTok{fit.linear.model}\NormalTok{(}\DataTypeTok{covariate=}\KeywordTok{as.matrix}\NormalTok{(dat[,}\KeywordTok{c}\NormalTok{(}\StringTok{'instrument'}\NormalTok{)]),}\DataTypeTok{outcome=}\NormalTok{dat}\OperatorTok{$}\NormalTok{y)}
  
\NormalTok{  trtiv=}\KeywordTok{fit.linear.model}\NormalTok{(}\DataTypeTok{covariate=}\KeywordTok{as.matrix}\NormalTok{(dat[,}\KeywordTok{c}\NormalTok{(}\StringTok{'instrument'}\NormalTok{)]),}\DataTypeTok{outcome=}\NormalTok{dat}\OperatorTok{$}\NormalTok{trt)}
  
  
  \KeywordTok{return}\NormalTok{(}\KeywordTok{c}\NormalTok{(beta.hat.naive[}\DecValTok{2}\NormalTok{],yiv[}\DecValTok{2}\NormalTok{]}\OperatorTok{/}\NormalTok{trtiv[}\DecValTok{2}\NormalTok{]))}
\NormalTok{\}}
\NormalTok{N.sim=}\DecValTok{10000}\NormalTok{;}

\KeywordTok{set.seed}\NormalTok{(}\DecValTok{1}\NormalTok{)}

\NormalTok{sim.iv=}\KeywordTok{replicate}\NormalTok{(N.sim, }\KeywordTok{simulate.one.instance}\NormalTok{());}
\KeywordTok{apply}\NormalTok{(sim.iv,}\DecValTok{1}\NormalTok{,mean) }\CommentTok{# The true value is 1}
\KeywordTok{apply}\NormalTok{( (sim.iv}\OperatorTok{-}\DecValTok{1}\NormalTok{),}\DecValTok{1}\NormalTok{,}\ControlFlowTok{function}\NormalTok{(x)\{ }\KeywordTok{sum}\NormalTok{(x}\OperatorTok{^}\DecValTok{2}\NormalTok{) \}) }\CommentTok{# The mean squared error}
\end{Highlighting}
\end{Shaded}

\section{Least squares estimation}\label{least-squares-estimation}

Reading materials: Slides 119 - 141 in
STA108\_LinearRegression\_S20.pdf.

\begin{Shaded}
\begin{Highlighting}[]
\CommentTok{# Consider another data set with a bit more covariates }
\KeywordTok{library}\NormalTok{(AER)}
\KeywordTok{data}\NormalTok{(}\StringTok{"Fatalities"}\NormalTok{)}
\NormalTok{dat.fatalities =}\StringTok{ }\NormalTok{Fatalities}
\CommentTok{# This dataset contains 34 variables}
\CommentTok{# It is actually a longitudinal/panel data}
\CommentTok{# We will ignore the spatial temporal structure in this class}

\NormalTok{y=dat.fatalities}\OperatorTok{$}\NormalTok{fatal; }\CommentTok{#Number of vehicle fatalities.}
\NormalTok{X=}\KeywordTok{as.matrix}\NormalTok{(dat.fatalities[,}\KeywordTok{c}\NormalTok{(}\StringTok{'spirits'}\NormalTok{,}\StringTok{'unemp'}\NormalTok{,}\StringTok{'income'}\NormalTok{,}\StringTok{'beertax'}\NormalTok{,}\StringTok{'baptist'}\NormalTok{,}\StringTok{'mormon'}\NormalTok{,}\StringTok{'drinkage'}\NormalTok{,}\StringTok{'dry'}\NormalTok{,}\StringTok{'youngdrivers'}\NormalTok{,}\StringTok{'miles'}\NormalTok{,}\StringTok{'gsp'}\NormalTok{)]);}


\CommentTok{# Rewrite the functions in Chapter 2}
\NormalTok{linear.model<-}\ControlFlowTok{function}\NormalTok{(beta,covariate)\{}
  \CommentTok{# beta: a 2-vector, where the first entry is the intercept}
\NormalTok{  beta=}\KeywordTok{as.matrix}\NormalTok{(beta,}\DataTypeTok{nrow=}\KeywordTok{length}\NormalTok{(beta))}
\NormalTok{  yout=covariate}\OperatorTok\NormalTok{beta[}\OperatorTok{-}\DecValTok{1}\NormalTok{]}\OperatorTok{+}\NormalTok{beta[}\DecValTok{1}\NormalTok{];}
  \KeywordTok{return}\NormalTok{(yout);}
  \CommentTok{# Note: this function only works with simple linear regression model}
  \CommentTok{# How would you generalize it?}
\NormalTok{\}}
\NormalTok{sum.of.squares<-}\ControlFlowTok{function}\NormalTok{(beta,covariate,outcome)\{}
\NormalTok{  yout=}\KeywordTok{linear.model}\NormalTok{(}\DataTypeTok{beta=}\NormalTok{beta,}\DataTypeTok{covariate=}\NormalTok{covariate);}
\NormalTok{  res=outcome}\OperatorTok{-}\NormalTok{yout;}
\NormalTok{  sos=}\StringTok{ }\KeywordTok{sum}\NormalTok{(res}\OperatorTok{^}\DecValTok{2}\NormalTok{);}
  \KeywordTok{return}\NormalTok{(sos)}
\NormalTok{\}}
\NormalTok{fit.linear.model<-}\ControlFlowTok{function}\NormalTok{(covariate,outcome)\{}
\NormalTok{  X=}\KeywordTok{cbind}\NormalTok{(}\DecValTok{1}\NormalTok{,covariate);}
  
\NormalTok{  beta.fit=}\KeywordTok{solve}\NormalTok{( }\KeywordTok{t}\NormalTok{(X)}\OperatorTok\NormalTok{X )}\OperatorTok\KeywordTok{t}\NormalTok{(X)}\OperatorTok\NormalTok{outcome;}
  \KeywordTok{return}\NormalTok{(beta.fit)}
\NormalTok{\}}

\NormalTok{beta.hat=}\KeywordTok{fit.linear.model}\NormalTok{(}\DataTypeTok{covariate=}\NormalTok{X,}\DataTypeTok{outcome=}\NormalTok{y)}

\CommentTok{# Compare with lm()}
\NormalTok{fit.lm=}\KeywordTok{lm}\NormalTok{(y}\OperatorTok{~}\NormalTok{X}\OperatorTok{+}\DecValTok{1}\NormalTok{);}
\NormalTok{beta.hat}
\end{Highlighting}
\end{Shaded}

\begin{Shaded}
\begin{Highlighting}[]
\CommentTok{# Hat matrix }
\NormalTok{calculate.hat<-}\ControlFlowTok{function}\NormalTok{(covariate)\{}
\NormalTok{  X=}\KeywordTok{cbind}\NormalTok{(}\DecValTok{1}\NormalTok{,covariate);}
\NormalTok{  P=X}\OperatorTok\KeywordTok{solve}\NormalTok{(}\KeywordTok{t}\NormalTok{(X)}\OperatorTok\NormalTok{X)}\OperatorTok\KeywordTok{t}\NormalTok{(X);}
  \KeywordTok{return}\NormalTok{(P)}
\NormalTok{\}}
\NormalTok{P=}\KeywordTok{calculate.hat}\NormalTok{(}\DataTypeTok{covariate=}\NormalTok{X);}

\CommentTok{# Verify the properties of the hat matrix:}
\NormalTok{## Residuals }
\NormalTok{resid=}\StringTok{ }\NormalTok{y}\OperatorTok{-}\KeywordTok{linear.model}\NormalTok{(}\DataTypeTok{beta=}\NormalTok{beta.hat,}\DataTypeTok{covariate=}\NormalTok{X);}
\NormalTok{resid.hat =}\StringTok{ }\NormalTok{(}\KeywordTok{diag}\NormalTok{(}\KeywordTok{length}\NormalTok{(y))}\OperatorTok{-}\NormalTok{P)}\OperatorTok\NormalTok{y;}
\KeywordTok{sum}\NormalTok{((resid}\OperatorTok{-}\NormalTok{resid.hat)}\OperatorTok{^}\DecValTok{2}\NormalTok{)}

\NormalTok{## Idempotent }
\KeywordTok{max}\NormalTok{( }\KeywordTok{abs}\NormalTok{(P}\OperatorTok\NormalTok{P}\OperatorTok{-}\NormalTok{P))}

\NormalTok{## Orthogonality }
\KeywordTok{max}\NormalTok{(}\KeywordTok{abs}\NormalTok{((}\KeywordTok{diag}\NormalTok{(}\KeywordTok{length}\NormalTok{(y))}\OperatorTok{-}\NormalTok{P)}\OperatorTok\NormalTok{X))}
\end{Highlighting}
\end{Shaded}

\begin{Shaded}
\begin{Highlighting}[]
\CommentTok{# Sum of squares with the hat matrix:}
\NormalTok{n=}\KeywordTok{length}\NormalTok{(y);J=}\KeywordTok{matrix}\NormalTok{(}\DecValTok{1}\NormalTok{,n,n);I=}\KeywordTok{diag}\NormalTok{(n)}
\NormalTok{total.sum.of.squares=}\StringTok{ }\KeywordTok{t}\NormalTok{(y)}\OperatorTok\NormalTok{(I}\OperatorTok{-}\NormalTok{J}\OperatorTok{/}\NormalTok{n)}\OperatorTok\NormalTok{y}
\NormalTok{explained.sum.of.squares=}\KeywordTok{t}\NormalTok{(y)}\OperatorTok\NormalTok{(P}\OperatorTok{-}\NormalTok{J}\OperatorTok{/}\NormalTok{n)}\OperatorTok\NormalTok{y}
\NormalTok{residual.sum.of.squares=}\KeywordTok{t}\NormalTok{(y)}\OperatorTok\NormalTok{(I}\OperatorTok{-}\NormalTok{P)}\OperatorTok\NormalTok{y}

\CommentTok{# Check if it is true}
\KeywordTok{sum.of.squares}\NormalTok{(beta.hat,}\DataTypeTok{covariate =}\NormalTok{ X,}\DataTypeTok{outcome =}\NormalTok{ y)}
\end{Highlighting}
\end{Shaded}

\begin{Shaded}
\begin{Highlighting}[]
\CommentTok{# R^2}
\NormalTok{R.sq =}\StringTok{ }\NormalTok{explained.sum.of.squares}\OperatorTok{/}\NormalTok{total.sum.of.squares;}

\CommentTok{# Through more variables into X}
\CommentTok{# These variables are meaningless}
\KeywordTok{set.seed}\NormalTok{(}\DecValTok{1}\NormalTok{)}
\NormalTok{Z=}\KeywordTok{matrix}\NormalTok{(}\KeywordTok{rnorm}\NormalTok{(}\DecValTok{5}\OperatorTok{*}\NormalTok{n),}\DataTypeTok{nrow=}\NormalTok{n,}\DataTypeTok{ncol=}\DecValTok{5}\NormalTok{);}
\NormalTok{X.Z=}\KeywordTok{cbind}\NormalTok{(X,Z);}

\NormalTok{P.Z=}\KeywordTok{calculate.hat}\NormalTok{(}\DataTypeTok{covariate=}\NormalTok{X.Z);}

\NormalTok{R.sq.Z=}\StringTok{ }\KeywordTok{t}\NormalTok{(y)}\OperatorTok\NormalTok{(P.Z}\OperatorTok{-}\NormalTok{J}\OperatorTok{/}\NormalTok{n)}\OperatorTok\NormalTok{y}\OperatorTok{/}\KeywordTok{t}\NormalTok{(y)}\OperatorTok\NormalTok{(I}\OperatorTok{-}\NormalTok{J}\OperatorTok{/}\NormalTok{n)}\OperatorTok\NormalTok{y;}

\NormalTok{R.sq.Z}
\NormalTok{R.sq}

\CommentTok{# Hence the adjusted R^2:}
\NormalTok{R.sq.adj=}\DecValTok{1}\OperatorTok{-}\NormalTok{(residual.sum.of.squares}\OperatorTok{/}\NormalTok{(n}\OperatorTok{-}\KeywordTok{dim}\NormalTok{(X)[}\DecValTok{2}\NormalTok{]}\OperatorTok{-}\DecValTok{1}\NormalTok{) )}\OperatorTok{/}\NormalTok{(total.sum.of.squares}\OperatorTok{/}\NormalTok{(n}\OperatorTok{-}\DecValTok{1}\NormalTok{));}

\NormalTok{R.sq.Z.adj=}\DecValTok{1}\OperatorTok{-}\NormalTok{(}\KeywordTok{t}\NormalTok{(y)}\OperatorTok\NormalTok{(I}\OperatorTok{-}\NormalTok{P.Z)}\OperatorTok\NormalTok{y}\OperatorTok{/}\NormalTok{(n}\OperatorTok{-}\KeywordTok{dim}\NormalTok{(X.Z)[}\DecValTok{2}\NormalTok{]}\OperatorTok{-}\DecValTok{1}\NormalTok{) )}\OperatorTok{/}\NormalTok{(}\KeywordTok{t}\NormalTok{(y)}\OperatorTok\NormalTok{(I}\OperatorTok{-}\NormalTok{J}\OperatorTok{/}\NormalTok{n)}\OperatorTok\NormalTok{y}\OperatorTok{/}\NormalTok{(n}\OperatorTok{-}\DecValTok{1}\NormalTok{));}
\end{Highlighting}
\end{Shaded}

\section{Underfitting and
overfitting}\label{underfitting-and-overfitting}

Reading materials: Slides 142 - 152 in
STA108\_LinearRegression\_S20.pdf.

\begin{Shaded}
\begin{Highlighting}[]
\NormalTok{## Underfitting}
\NormalTok{simulate.one.instance<-}\ControlFlowTok{function}\NormalTok{(beta.true,X,Z)\{}
\NormalTok{  eta=}\KeywordTok{c}\NormalTok{(}\DecValTok{3}\NormalTok{,}\DecValTok{4}\NormalTok{);}
\NormalTok{ n=}\StringTok{ }\KeywordTok{dim}\NormalTok{(X)[}\DecValTok{1}\NormalTok{];}
\NormalTok{  noise =}\StringTok{ }\KeywordTok{rnorm}\NormalTok{(n)}\OperatorTok{/}\DecValTok{2}\NormalTok{;}
\NormalTok{  y.XZ=X}\OperatorTok\NormalTok{beta.true}\OperatorTok{+}\NormalTok{Z}\OperatorTok\NormalTok{eta }\OperatorTok{+}\StringTok{ }\NormalTok{noise;}
  
\NormalTok{  beta.full =}\StringTok{ }\KeywordTok{fit.linear.model}\NormalTok{(}\DataTypeTok{covariate=}\KeywordTok{cbind}\NormalTok{(X,Z),}\DataTypeTok{outcome=}\NormalTok{y.XZ)}
\NormalTok{  beta.under=}\StringTok{ }\KeywordTok{fit.linear.model}\NormalTok{(}\DataTypeTok{covariate=}\NormalTok{X,}\DataTypeTok{outcome=}\NormalTok{y.XZ)}
  
  \KeywordTok{return}\NormalTok{(}\KeywordTok{cbind}\NormalTok{(beta.full[}\DecValTok{2}\OperatorTok{:}\DecValTok{3}\NormalTok{]}\OperatorTok{-}\NormalTok{beta.true,beta.under[}\DecValTok{2}\OperatorTok{:}\DecValTok{3}\NormalTok{]}\OperatorTok{-}\NormalTok{beta.true))}
\NormalTok{\}}

\NormalTok{beta.true=}\KeywordTok{c}\NormalTok{(}\DecValTok{1}\NormalTok{,}\DecValTok{2}\NormalTok{)}
\NormalTok{n=}\DecValTok{100}\NormalTok{;}
\NormalTok{X=}\KeywordTok{matrix}\NormalTok{(}\KeywordTok{rnorm}\NormalTok{(}\DecValTok{2}\OperatorTok{*}\NormalTok{n),}\DataTypeTok{ncol=}\DecValTok{2}\NormalTok{)}
\NormalTok{Z=}\KeywordTok{matrix}\NormalTok{(}\KeywordTok{rnorm}\NormalTok{(}\DecValTok{2}\OperatorTok{*}\NormalTok{n),}\DataTypeTok{ncol=}\DecValTok{2}\NormalTok{)}
  
\KeywordTok{set.seed}\NormalTok{(}\DecValTok{1}\NormalTok{);}
\NormalTok{N.sim=}\DecValTok{10000}\NormalTok{;}
\NormalTok{under.sim=}\KeywordTok{replicate}\NormalTok{(N.sim,}\KeywordTok{simulate.one.instance}\NormalTok{(beta.true,X,Z))}
\CommentTok{# Fitting the true model gives}

\KeywordTok{apply}\NormalTok{(under.sim[}\DecValTok{1}\NormalTok{,,],}\DecValTok{1}\NormalTok{,mean) }\CommentTok{# Non-zero bias}

\CommentTok{# Does it contradict the claim about precision variable?}
\CommentTok{# Try another simulation:}
\NormalTok{simulate.one.instance.pre<-}\ControlFlowTok{function}\NormalTok{(beta.true)\{}
\NormalTok{  eta=}\KeywordTok{c}\NormalTok{(}\DecValTok{3}\NormalTok{,}\DecValTok{4}\NormalTok{);}
\NormalTok{n=}\DecValTok{100}\NormalTok{;}
\NormalTok{X=}\KeywordTok{matrix}\NormalTok{(}\KeywordTok{rnorm}\NormalTok{(}\DecValTok{2}\OperatorTok{*}\NormalTok{n),}\DataTypeTok{ncol=}\DecValTok{2}\NormalTok{)}
\NormalTok{Z=}\KeywordTok{matrix}\NormalTok{(}\KeywordTok{rnorm}\NormalTok{(}\DecValTok{2}\OperatorTok{*}\NormalTok{n),}\DataTypeTok{ncol=}\DecValTok{2}\NormalTok{)}
\NormalTok{  noise =}\StringTok{ }\KeywordTok{rnorm}\NormalTok{(n)}\OperatorTok{/}\DecValTok{2}\NormalTok{;}
\NormalTok{  y.XZ=X}\OperatorTok\NormalTok{beta.true}\OperatorTok{+}\NormalTok{Z}\OperatorTok\NormalTok{eta }\OperatorTok{+}\StringTok{ }\NormalTok{noise;}
  
\NormalTok{  beta.full =}\StringTok{ }\KeywordTok{fit.linear.model}\NormalTok{(}\DataTypeTok{covariate=}\KeywordTok{cbind}\NormalTok{(X,Z),}\DataTypeTok{outcome=}\NormalTok{y.XZ)}
\NormalTok{  beta.under=}\StringTok{ }\KeywordTok{fit.linear.model}\NormalTok{(}\DataTypeTok{covariate=}\NormalTok{X,}\DataTypeTok{outcome=}\NormalTok{y.XZ)}
  
  \KeywordTok{return}\NormalTok{(}\KeywordTok{cbind}\NormalTok{(beta.full[}\DecValTok{2}\OperatorTok{:}\DecValTok{3}\NormalTok{]}\OperatorTok{-}\NormalTok{beta.true,beta.under[}\DecValTok{2}\OperatorTok{:}\DecValTok{3}\NormalTok{]}\OperatorTok{-}\NormalTok{beta.true))}
\NormalTok{\}}

\NormalTok{beta.true=}\KeywordTok{c}\NormalTok{(}\DecValTok{1}\NormalTok{,}\DecValTok{2}\NormalTok{)}
  
\KeywordTok{set.seed}\NormalTok{(}\DecValTok{1}\NormalTok{);}
\NormalTok{N.sim=}\DecValTok{10000}\NormalTok{;}
\NormalTok{pre.sim=}\KeywordTok{replicate}\NormalTok{(N.sim,}\KeywordTok{simulate.one.instance.pre}\NormalTok{(beta.true))}

\KeywordTok{apply}\NormalTok{(pre.sim[}\DecValTok{1}\NormalTok{,,],}\DecValTok{1}\NormalTok{,mean) }\CommentTok{# almost zero bias}
\end{Highlighting}
\end{Shaded}

\begin{Shaded}
\begin{Highlighting}[]
\NormalTok{### Overfitting}
\CommentTok{# Just modify the code slightly..}
\NormalTok{simulate.one.instance<-}\ControlFlowTok{function}\NormalTok{(beta.true,X,Z)\{}
\NormalTok{  eta=}\KeywordTok{c}\NormalTok{(}\DecValTok{3}\NormalTok{,}\DecValTok{4}\NormalTok{);}
\NormalTok{ n=}\StringTok{ }\KeywordTok{dim}\NormalTok{(X)[}\DecValTok{1}\NormalTok{];}
\NormalTok{  noise =}\StringTok{ }\KeywordTok{rnorm}\NormalTok{(n)}\OperatorTok{/}\DecValTok{2}\NormalTok{;}
\NormalTok{  y.X=X}\OperatorTok\NormalTok{beta.true}\OperatorTok{+}\StringTok{ }\NormalTok{noise;}
  
\NormalTok{  beta.over=}\StringTok{ }\KeywordTok{fit.linear.model}\NormalTok{(}\DataTypeTok{covariate=}\KeywordTok{cbind}\NormalTok{(X,Z),}\DataTypeTok{outcome=}\NormalTok{y.X)}
\NormalTok{  beta.full=}\StringTok{ }\KeywordTok{fit.linear.model}\NormalTok{(}\DataTypeTok{covariate=}\NormalTok{X,}\DataTypeTok{outcome=}\NormalTok{y.X)}
  
  \KeywordTok{return}\NormalTok{(}\KeywordTok{cbind}\NormalTok{(beta.full[}\DecValTok{2}\OperatorTok{:}\DecValTok{3}\NormalTok{]}\OperatorTok{-}\NormalTok{beta.true,beta.over[}\DecValTok{2}\OperatorTok{:}\DecValTok{3}\NormalTok{]}\OperatorTok{-}\NormalTok{beta.true))}
\NormalTok{\}}

\NormalTok{beta.true=}\KeywordTok{c}\NormalTok{(}\DecValTok{1}\NormalTok{,}\DecValTok{2}\NormalTok{)}
\NormalTok{n=}\DecValTok{100}\NormalTok{;}
\NormalTok{X=}\KeywordTok{matrix}\NormalTok{(}\KeywordTok{rnorm}\NormalTok{(}\DecValTok{2}\OperatorTok{*}\NormalTok{n),}\DataTypeTok{ncol=}\DecValTok{2}\NormalTok{)}
\NormalTok{Z=}\KeywordTok{matrix}\NormalTok{(}\KeywordTok{rnorm}\NormalTok{(}\DecValTok{2}\OperatorTok{*}\NormalTok{n),}\DataTypeTok{ncol=}\DecValTok{2}\NormalTok{)}
  
\KeywordTok{set.seed}\NormalTok{(}\DecValTok{1}\NormalTok{);}
\NormalTok{N.sim=}\DecValTok{10000}\NormalTok{;}
\NormalTok{over.sim=}\KeywordTok{replicate}\NormalTok{(N.sim,}\KeywordTok{simulate.one.instance}\NormalTok{(beta.true,X,Z))}
\CommentTok{# Fitting the true model gives}

\KeywordTok{apply}\NormalTok{(over.sim[}\DecValTok{1}\NormalTok{,,],}\DecValTok{1}\NormalTok{,mean) }\CommentTok{# No bias}
\end{Highlighting}
\end{Shaded}

\section{Sampling distribution and
inference}\label{sampling-distribution-and-inference}

Reading materials: Slides 153 - 165 in
STA108\_LinearRegression\_S20.pdf.

We will skip the part for multivariate normal distribution. You will
learn more about this in STA 135: Multivariate Data Analysis.

\subsection{ANOVA}\label{anova}

\begin{Shaded}
\begin{Highlighting}[]
\KeywordTok{attach}\NormalTok{(dat.advertising)}
\NormalTok{full.model=}\KeywordTok{lm}\NormalTok{(sales}\OperatorTok{~}\NormalTok{TV}\OperatorTok{+}\NormalTok{radio}\OperatorTok{+}\NormalTok{newspaper);}
\NormalTok{reduced.model=}\KeywordTok{lm}\NormalTok{(sales}\OperatorTok{~}\NormalTok{TV);}
\KeywordTok{anova}\NormalTok{(reduced.model,full.model)}

\NormalTok{RSS.Reduced<-}\KeywordTok{sum}\NormalTok{((}\KeywordTok{summary}\NormalTok{(reduced.model)}\OperatorTok{$}\NormalTok{residuals)}\OperatorTok{^}\DecValTok{2}\NormalTok{);}
\NormalTok{RSS.Full<-}\KeywordTok{sum}\NormalTok{((}\KeywordTok{summary}\NormalTok{(full.model)}\OperatorTok{$}\NormalTok{residuals)}\OperatorTok{^}\DecValTok{2}\NormalTok{);}
\NormalTok{df.Reduced <-}\StringTok{ }\KeywordTok{summary}\NormalTok{(reduced.model)}\OperatorTok{$}\NormalTok{df[}\DecValTok{2}\NormalTok{];}
\NormalTok{df.Full <-}\StringTok{ }\KeywordTok{summary}\NormalTok{(full.model)}\OperatorTok{$}\NormalTok{df[}\DecValTok{2}\NormalTok{];}

\NormalTok{tF<-((RSS.Reduced}\OperatorTok{-}\NormalTok{RSS.Full)}\OperatorTok{/}\NormalTok{(df.Reduced}\OperatorTok{-}\NormalTok{df.Full))}\OperatorTok{/}\NormalTok{(RSS.Full}\OperatorTok{/}\NormalTok{df.Full);}
\NormalTok{tF}
\end{Highlighting}
\end{Shaded}

\subsection{Wald test}\label{wald-test}

Suppose that we want to test the null hypothesis
\[H_0: \beta_1-2\beta_2 = 0 \ {\rm v.s.}\ H_a:  \beta_1-2\beta_2 \neq 0,\]
where \(\beta_1\) is the regression coefficient for \texttt{TV} and
\(\beta_2\) is the regression coefficient for \texttt{radio}.

\begin{Shaded}
\begin{Highlighting}[]
\NormalTok{full.model=}\KeywordTok{lm}\NormalTok{(sales}\OperatorTok{~}\NormalTok{TV}\OperatorTok{+}\NormalTok{radio}\OperatorTok{+}\NormalTok{newspaper);}
\NormalTok{covariance.LSE=}\KeywordTok{vcov}\NormalTok{(full.model); }\CommentTok{# Calculate the covariance for the least squares estimators}
\NormalTok{coef.LSE=}\KeywordTok{coef}\NormalTok{(full.model);}
\NormalTok{covariance.test =}\StringTok{ }\NormalTok{covariance.LSE[}\DecValTok{2}\OperatorTok{:}\DecValTok{3}\NormalTok{,}\DecValTok{2}\OperatorTok{:}\DecValTok{3}\NormalTok{]; }\CommentTok{# Extract the submatrix of covariance for beta1 and beta2}
\NormalTok{coef.test=coef.LSE[}\DecValTok{2}\OperatorTok{:}\DecValTok{3}\NormalTok{];}
\NormalTok{R=}\KeywordTok{matrix}\NormalTok{(}\DecValTok{0}\NormalTok{,}\DecValTok{1}\NormalTok{,}\DecValTok{2}\NormalTok{);}
\NormalTok{R[}\DecValTok{1}\NormalTok{,}\DecValTok{1}\NormalTok{]=}\DecValTok{1}\NormalTok{;R[}\DecValTok{1}\NormalTok{,}\DecValTok{2}\NormalTok{]=}\OperatorTok{-}\DecValTok{2}\NormalTok{;}

\NormalTok{test.stat=}\KeywordTok{t}\NormalTok{(R}\OperatorTok\NormalTok{coef.test)}\OperatorTok\KeywordTok{solve}\NormalTok{(R}\OperatorTok\NormalTok{covariance.test}\OperatorTok\KeywordTok{t}\NormalTok{(R))}\OperatorTok\NormalTok{(R}\OperatorTok\NormalTok{coef.test);}
\NormalTok{p.value=}\DecValTok{1}\OperatorTok{-}\KeywordTok{pchisq}\NormalTok{(test.stat,}\DataTypeTok{df=}\DecValTok{1}\NormalTok{);}
\end{Highlighting}
\end{Shaded}

We can verify that the Wald test yields similar results with the F-test
using \texttt{anova}.
\[H_0: \beta_1=\beta_2 = 0 \ {\rm v.s.}\ H_a:  \beta_1\neq 0, \beta_2 \neq 0.\]

\begin{Shaded}
\begin{Highlighting}[]
\NormalTok{full.model=}\KeywordTok{lm}\NormalTok{(sales}\OperatorTok{~}\NormalTok{TV}\OperatorTok{+}\NormalTok{radio}\OperatorTok{+}\NormalTok{newspaper);}
\NormalTok{covariance.LSE=}\KeywordTok{vcov}\NormalTok{(full.model); }\CommentTok{# Calculate the covariance for the least squares estimators}
\NormalTok{coef.LSE=}\KeywordTok{coef}\NormalTok{(full.model);}
\NormalTok{covariance.test =}\StringTok{ }\NormalTok{covariance.LSE[}\DecValTok{2}\OperatorTok{:}\DecValTok{3}\NormalTok{,}\DecValTok{2}\OperatorTok{:}\DecValTok{3}\NormalTok{]; }\CommentTok{# Extract the submatrix of covariance for beta1 and beta2}
\NormalTok{coef.test=coef.LSE[}\DecValTok{2}\OperatorTok{:}\DecValTok{3}\NormalTok{];}
\NormalTok{R=}\KeywordTok{matrix}\NormalTok{(}\DecValTok{0}\NormalTok{,}\DecValTok{2}\NormalTok{,}\DecValTok{2}\NormalTok{);}
\NormalTok{R[}\DecValTok{1}\NormalTok{,}\DecValTok{1}\NormalTok{]=}\DecValTok{1}\NormalTok{;R[}\DecValTok{2}\NormalTok{,}\DecValTok{2}\NormalTok{]=}\DecValTok{1}\NormalTok{;}

\NormalTok{test.stat=}\KeywordTok{t}\NormalTok{(R}\OperatorTok\NormalTok{coef.test)}\OperatorTok\KeywordTok{solve}\NormalTok{(R}\OperatorTok\NormalTok{covariance.test}\OperatorTok\KeywordTok{t}\NormalTok{(R))}\OperatorTok\NormalTok{(R}\OperatorTok\NormalTok{coef.test);}
\NormalTok{p_value=}\DecValTok{1}\OperatorTok{-}\KeywordTok{pchisq}\NormalTok{(test.stat,}\DataTypeTok{df=}\DecValTok{2}\NormalTok{);}


\NormalTok{reduced_model=}\KeywordTok{lm}\NormalTok{(sales}\OperatorTok{~}\NormalTok{newspaper);}
\KeywordTok{anova}\NormalTok{(reduced.model,full.model)}
\end{Highlighting}
\end{Shaded}

\chapter{Model selection}\label{ch:selection}

\begin{Shaded}
\begin{Highlighting}[]
\KeywordTok{library}\NormalTok{(AER)}
\KeywordTok{data}\NormalTok{(}\StringTok{"Fatalities"}\NormalTok{)}
\NormalTok{dat.fatalities =}\StringTok{ }\NormalTok{Fatalities[,}\KeywordTok{c}\NormalTok{(}\StringTok{'fatal'}\NormalTok{,}\StringTok{'spirits'}\NormalTok{,}\StringTok{'unemp'}\NormalTok{,}\StringTok{'income'}\NormalTok{,}\StringTok{'beertax'}\NormalTok{,}\StringTok{'baptist'}\NormalTok{,}\StringTok{'mormon'}\NormalTok{,}\StringTok{'drinkage'}\NormalTok{,}\StringTok{'dry'}\NormalTok{,}\StringTok{'youngdrivers'}\NormalTok{,}\StringTok{'miles'}\NormalTok{,}\StringTok{'gsp'}\NormalTok{)]}
\end{Highlighting}
\end{Shaded}

\section{Criteria}\label{criteria}

Reading materials: Slides 166 - 180 in
STA108\_LinearRegression\_S20.pdf.

\begin{Shaded}
\begin{Highlighting}[]
\CommentTok{# Information criteria}


\NormalTok{fit.full=}\KeywordTok{lm}\NormalTok{(fatal}\OperatorTok{~}\NormalTok{spirits}\OperatorTok{+}\NormalTok{unemp}\OperatorTok{+}\NormalTok{income}\OperatorTok{+}\NormalTok{beertax}\OperatorTok{+}\NormalTok{baptist}\OperatorTok{+}\NormalTok{mormon}\OperatorTok{+}\NormalTok{drinkage}\OperatorTok{+}\NormalTok{dry}\OperatorTok{+}\NormalTok{youngdrivers}\OperatorTok{+}\NormalTok{miles}\OperatorTok{+}\NormalTok{gsp,}\DataTypeTok{data=}\NormalTok{dat.fatalities);}
\NormalTok{fit.six=}\KeywordTok{lm}\NormalTok{(fatal}\OperatorTok{~}\NormalTok{spirits}\OperatorTok{+}\NormalTok{unemp}\OperatorTok{+}\NormalTok{income}\OperatorTok{+}\NormalTok{beertax}\OperatorTok{+}\NormalTok{baptist}\OperatorTok{+}\NormalTok{mormon,}\DataTypeTok{data=}\NormalTok{dat.fatalities)}

\KeywordTok{BIC}\NormalTok{(fit.full)}
\KeywordTok{BIC}\NormalTok{(fit.six)}


\KeywordTok{AIC}\NormalTok{(fit.full)}
\KeywordTok{AIC}\NormalTok{(fit.six) }
\end{Highlighting}
\end{Shaded}

\begin{Shaded}
\begin{Highlighting}[]
\CommentTok{# cross-validation:}
\CommentTok{# We will use a package for this:}
\KeywordTok{library}\NormalTok{(boot)}


\NormalTok{fit.full.g=}\KeywordTok{glm}\NormalTok{(fatal}\OperatorTok{~}\NormalTok{spirits}\OperatorTok{+}\NormalTok{unemp}\OperatorTok{+}\NormalTok{income}\OperatorTok{+}\NormalTok{beertax}\OperatorTok{+}\NormalTok{baptist}\OperatorTok{+}\NormalTok{mormon}\OperatorTok{+}\NormalTok{drinkage}\OperatorTok{+}\NormalTok{dry}\OperatorTok{+}\NormalTok{youngdrivers}\OperatorTok{+}\NormalTok{miles}\OperatorTok{+}\NormalTok{gsp,}\DataTypeTok{data=}\NormalTok{dat.fatalities);}
\NormalTok{fit.six.g=}\KeywordTok{glm}\NormalTok{(fatal}\OperatorTok{~}\NormalTok{spirits}\OperatorTok{+}\NormalTok{unemp}\OperatorTok{+}\NormalTok{income}\OperatorTok{+}\NormalTok{beertax}\OperatorTok{+}\NormalTok{baptist}\OperatorTok{+}\NormalTok{mormon,}\DataTypeTok{data=}\NormalTok{dat.fatalities)}


\KeywordTok{set.seed}\NormalTok{(}\DecValTok{1}\NormalTok{)}
\NormalTok{cv.full.}\DecValTok{10}\NormalTok{=}\KeywordTok{cv.glm}\NormalTok{(dat.fatalities,fit.full.g,}\DataTypeTok{K=}\DecValTok{10}\NormalTok{)}
\NormalTok{cv.full.}\DecValTok{10}\OperatorTok{$}\NormalTok{delta}

\NormalTok{cv.six.}\DecValTok{10}\NormalTok{=}\KeywordTok{cv.glm}\NormalTok{(dat.fatalities,fit.six.g,}\DataTypeTok{K=}\DecValTok{10}\NormalTok{)}
\NormalTok{cv.six.}\DecValTok{10}\OperatorTok{$}\NormalTok{delta}

\CommentTok{# From the helpfile, delta is:}

\CommentTok{# A vector of length two. The first component is the raw cross-validation estimate of prediction error. The second component is the adjusted cross-validation estimate. The adjustment is designed to compensate for the bias introduced by not using leave-one-out cross-validation.}
\end{Highlighting}
\end{Shaded}

\section{Selection procedure}\label{selection-procedure}

Reading materials: Slides 181 - 190 in
STA108\_LinearRegression\_S20.pdf.

\begin{Shaded}
\begin{Highlighting}[]
\CommentTok{# Model selection procedure }

\CommentTok{# Forward selection & AIC}
\KeywordTok{step}\NormalTok{(}\KeywordTok{lm}\NormalTok{(fatal}\OperatorTok{~}\NormalTok{spirits}\OperatorTok{+}\NormalTok{unemp}\OperatorTok{+}\NormalTok{income}\OperatorTok{+}\NormalTok{beertax}\OperatorTok{+}\NormalTok{baptist}\OperatorTok{+}\NormalTok{mormon}\OperatorTok{+}\NormalTok{drinkage}\OperatorTok{+}\NormalTok{dry}\OperatorTok{+}\NormalTok{youngdrivers}\OperatorTok{+}\NormalTok{miles}\OperatorTok{+}\NormalTok{gsp,}\DataTypeTok{data=}\NormalTok{dat.fatalities),}\DataTypeTok{direction=}\StringTok{"forward"}\NormalTok{,}\DataTypeTok{k=}\DecValTok{2}\NormalTok{) }


\CommentTok{# Backward selection & AIC}
\KeywordTok{step}\NormalTok{(}\KeywordTok{lm}\NormalTok{(fatal}\OperatorTok{~}\NormalTok{spirits}\OperatorTok{+}\NormalTok{unemp}\OperatorTok{+}\NormalTok{income}\OperatorTok{+}\NormalTok{beertax}\OperatorTok{+}\NormalTok{baptist}\OperatorTok{+}\NormalTok{mormon}\OperatorTok{+}\NormalTok{drinkage}\OperatorTok{+}\NormalTok{dry}\OperatorTok{+}\NormalTok{youngdrivers}\OperatorTok{+}\NormalTok{miles}\OperatorTok{+}\NormalTok{gsp,}\DataTypeTok{data=}\NormalTok{dat.fatalities),}\DataTypeTok{direction=}\StringTok{"backward"}\NormalTok{, }\DataTypeTok{k=}\DecValTok{2}\NormalTok{)}



\CommentTok{# Backward selection & BIC}
\KeywordTok{step}\NormalTok{(}\KeywordTok{lm}\NormalTok{(fatal}\OperatorTok{~}\NormalTok{spirits}\OperatorTok{+}\NormalTok{unemp}\OperatorTok{+}\NormalTok{income}\OperatorTok{+}\NormalTok{beertax}\OperatorTok{+}\NormalTok{baptist}\OperatorTok{+}\NormalTok{mormon}\OperatorTok{+}\NormalTok{drinkage}\OperatorTok{+}\NormalTok{dry}\OperatorTok{+}\NormalTok{youngdrivers}\OperatorTok{+}\NormalTok{miles}\OperatorTok{+}\NormalTok{gsp,}\DataTypeTok{data=}\NormalTok{dat.fatalities),}\DataTypeTok{direction=}\StringTok{"backward"}\NormalTok{, }\DataTypeTok{k=}\KeywordTok{log}\NormalTok{(}\KeywordTok{dim}\NormalTok{(dat.fatalities)[}\DecValTok{1}\NormalTok{]))}
\end{Highlighting}
\end{Shaded}

\begin{Shaded}
\begin{Highlighting}[]
\CommentTok{# Penalized regression: lasso}
\KeywordTok{library}\NormalTok{(glmnet)}

\NormalTok{y=dat.fatalities}\OperatorTok{$}\NormalTok{fatal; }\CommentTok{#Number of vehicle fatalities.}
\NormalTok{X=}\KeywordTok{as.matrix}\NormalTok{(dat.fatalities[,}\KeywordTok{c}\NormalTok{(}\StringTok{'spirits'}\NormalTok{,}\StringTok{'unemp'}\NormalTok{,}\StringTok{'income'}\NormalTok{,}\StringTok{'beertax'}\NormalTok{,}\StringTok{'baptist'}\NormalTok{,}\StringTok{'mormon'}\NormalTok{,}\StringTok{'drinkage'}\NormalTok{,}\StringTok{'dry'}\NormalTok{,}\StringTok{'youngdrivers'}\NormalTok{,}\StringTok{'miles'}\NormalTok{,}\StringTok{'gsp'}\NormalTok{)]);}


\NormalTok{fit.cv.glmnet=}\KeywordTok{cv.glmnet}\NormalTok{(}\DataTypeTok{y=}\NormalTok{y,}\DataTypeTok{x=}\NormalTok{X,}\DataTypeTok{nfold=}\DecValTok{10}\NormalTok{,}\DataTypeTok{family=}\StringTok{'gaussian'}\NormalTok{);}
        
\NormalTok{fit.cv.glmnet}\OperatorTok{$}\NormalTok{lambda.min }\CommentTok{# selected tuning parameter}

\NormalTok{fit.glmnet=}\KeywordTok{glmnet}\NormalTok{(}\DataTypeTok{x=}\NormalTok{X,}\DataTypeTok{y=}\NormalTok{y,}\DataTypeTok{family=}\StringTok{'gaussian'}\NormalTok{,}\DataTypeTok{alpha=}\DecValTok{1}\NormalTok{,}\DataTypeTok{lambda=}\NormalTok{fit.cv.glmnet}\OperatorTok{$}\NormalTok{lambda.min)}

\NormalTok{fit.glmnet}\OperatorTok{$}\NormalTok{beta }\CommentTok{# One of the fitted coefficient is zero }
\end{Highlighting}
\end{Shaded}

\appendix


\chapter{R basics}\label{ch:RBasics}

Reading materials: Sections 4, 6, and 8 in
\href{https://r4ds.had.co.nz/index.html}{R for data science} by Garrett
Grolemund and Hadley Wickham (optional).

This section reviews some basic functions in \texttt{R} that will be
useful in this class.

\section{\texorpdfstring{Basic objects in
\texttt{R}}{Basic objects in R}}\label{basic-objects-in-r}

\begin{Shaded}
\begin{Highlighting}[]
\NormalTok{## Generating a vector with seq()}

\KeywordTok{seq}\NormalTok{(}\DataTypeTok{from =} \FloatTok{3.1}\NormalTok{, }\DataTypeTok{to =} \FloatTok{8.9}\NormalTok{, }\DataTypeTok{by =} \FloatTok{0.8}\NormalTok{)}
\KeywordTok{seq}\NormalTok{(}\DataTypeTok{from =} \FloatTok{3.1}\NormalTok{, }\DataTypeTok{to =} \FloatTok{8.9}\NormalTok{, }\DataTypeTok{length.out =} \DecValTok{10}\NormalTok{)}
\CommentTok{# seq(from = 3.1, to = 8.9, by=1, length.out = 10)}

\NormalTok{### checking for equality of two vectors}
\NormalTok{a=}\DecValTok{1}\OperatorTok{:}\DecValTok{3}
\NormalTok{b=}\KeywordTok{c}\NormalTok{(}\DecValTok{1}\NormalTok{,}\DecValTok{2}\NormalTok{,}\DecValTok{3}\NormalTok{) }\CommentTok{#b<-c(1,2,3)}
\NormalTok{a }\OperatorTok{==}\StringTok{ }\NormalTok{b}

\NormalTok{### How do we check if *all* entries are the same?}
\KeywordTok{all}\NormalTok{(a}\OperatorTok{==}\NormalTok{b)}

\NormalTok{### How about the identical function?}
\KeywordTok{identical}\NormalTok{(a,b)}
\KeywordTok{typeof}\NormalTok{(a)}
\KeywordTok{typeof}\NormalTok{(b)}

\NormalTok{### Logical and character vectors}
\NormalTok{u =}\StringTok{ }\KeywordTok{c}\NormalTok{(T,F,T,F,F,T)}
\KeywordTok{typeof}\NormalTok{(u)}

\NormalTok{### Define character vectors}
\NormalTok{w=}\KeywordTok{c}\NormalTok{(}\StringTok{'a'}\NormalTok{, }\StringTok{'b'}\NormalTok{,}\StringTok{'c'}\NormalTok{,}\StringTok{'e'}\NormalTok{);}

\NormalTok{### What happens when merging character vectors with numbers?}
\NormalTok{y=}\KeywordTok{c}\NormalTok{(}\DecValTok{1}\OperatorTok{:}\DecValTok{3}\NormalTok{,w)}
\KeywordTok{typeof}\NormalTok{(y[}\DecValTok{1}\NormalTok{])}
\NormalTok{y=}\KeywordTok{c}\NormalTok{(pi,w)}

\NormalTok{### What if we force them to be numeric?}
\CommentTok{#?as.numeric}
\KeywordTok{as.numeric}\NormalTok{(y[}\DecValTok{1}\NormalTok{])}
\KeywordTok{as.numeric}\NormalTok{(y)}

\NormalTok{### Searching for elements of a vector}
\NormalTok{### Using > == < & |, and which}
\NormalTok{x =}\StringTok{ }\KeywordTok{c}\NormalTok{(}\DecValTok{1}\NormalTok{,}\DecValTok{3}\NormalTok{,}\DecValTok{4}\NormalTok{,}\FloatTok{6.7}\NormalTok{,}\OperatorTok{-}\DecValTok{4}\NormalTok{,}\OtherTok{NA}\NormalTok{,}\OperatorTok{-}\FloatTok{2.9}\NormalTok{)}

\NormalTok{x}
\NormalTok{x}\OperatorTok{==}\DecValTok{1}
\NormalTok{x}\OperatorTok{>}\StringTok{ }\DecValTok{1}
\NormalTok{x}\OperatorTok{<}\StringTok{ }\DecValTok{1}
\NormalTok{x }\OperatorTok{>=}\StringTok{ }\DecValTok{1}
\NormalTok{x}\OperatorTok{<=}\DecValTok{1}
\NormalTok{x}\OperatorTok{<}\DecValTok{0}
\NormalTok{x[}\DecValTok{5}\NormalTok{]}

\NormalTok{x[}\KeywordTok{which}\NormalTok{(x}\OperatorTok{<}\DecValTok{0}\NormalTok{)]}
\NormalTok{x[x}\OperatorTok{<}\DecValTok{0}\NormalTok{]}


\NormalTok{### Filtering of vectors}
\CommentTok{# 0< x < 2}
\NormalTok{x[(}\DecValTok{0} \OperatorTok{<}\StringTok{ }\NormalTok{x) }\OperatorTok{&}\StringTok{ }\NormalTok{(x }\OperatorTok{<}\StringTok{ }\DecValTok{2}\NormalTok{)]}
\KeywordTok{which}\NormalTok{((}\DecValTok{0} \OperatorTok{<}\StringTok{ }\NormalTok{x) }\OperatorTok{&}\StringTok{ }\NormalTok{(x }\OperatorTok{<}\StringTok{ }\DecValTok{2}\NormalTok{))}

\NormalTok{x[ }\OperatorTok{!}\NormalTok{(x}\OperatorTok{<}\DecValTok{0}\NormalTok{)]}
\OperatorTok{!}\NormalTok{(x}\OperatorTok{<}\DecValTok{0}\NormalTok{)}
\NormalTok{x[}\OperatorTok{-}\DecValTok{5}\NormalTok{]}
\end{Highlighting}
\end{Shaded}

\section{Summary statistics}\label{summary-statistics}

\begin{Shaded}
\begin{Highlighting}[]
\NormalTok{x =}\StringTok{ }\KeywordTok{c}\NormalTok{(}\DecValTok{1}\NormalTok{,}\DecValTok{3}\NormalTok{,}\DecValTok{4}\NormalTok{,}\FloatTok{6.7}\NormalTok{,}\OperatorTok{-}\DecValTok{4}\NormalTok{,}\OtherTok{NA}\NormalTok{,}\OperatorTok{-}\FloatTok{2.9}\NormalTok{)}
\NormalTok{x}

\KeywordTok{length}\NormalTok{(x) }\CommentTok{# you have seen this already}
\KeywordTok{sum}\NormalTok{(x) }\CommentTok{# summation of all elements of x}
\KeywordTok{sd}\NormalTok{(x) }\CommentTok{# standard deviation}
\NormalTok{### Why are they all NA?}

\NormalTok{### working with NA (missing values), NULL, and NaN}
\NormalTok{y=}\OtherTok{NULL}\NormalTok{;}
\DecValTok{0}\OperatorTok{/}\DecValTok{0} \CommentTok{# NaN}

\DecValTok{1}\OperatorTok{/}\DecValTok{0}
\OperatorTok{-}\DecValTok{1}\OperatorTok{/}\DecValTok{0}
\NormalTok{### How would you find the NAs in x?}
\NormalTok{x}\OperatorTok{==}\OtherTok{NA}
\KeywordTok{which}\NormalTok{(}\KeywordTok{is.na}\NormalTok{(x))}
\KeywordTok{is.null}\NormalTok{(y)}
\CommentTok{#is.nan()}
\CommentTok{#is.infinite()}


\KeywordTok{sum}\NormalTok{(x,}\DataTypeTok{na.rm=}\OtherTok{TRUE}\NormalTok{)}
\CommentTok{# ?sum}
\KeywordTok{args}\NormalTok{(sum) }\CommentTok{# check the default argument(s) for sum()}
\KeywordTok{example}\NormalTok{(sum)}
\NormalTok{### Summarize  matrices using apply()}
\end{Highlighting}
\end{Shaded}

\section{Data structures}\label{data-structures}

\subsection{String}\label{string}

\begin{Shaded}
\begin{Highlighting}[]
\NormalTok{wrd =}\StringTok{ }\KeywordTok{c}\NormalTok{(}\StringTok{"a"}\NormalTok{,}\StringTok{"new"}\NormalTok{,}\StringTok{"sentence"}\NormalTok{)}
\NormalTok{u =}\StringTok{ }\KeywordTok{c}\NormalTok{(}\StringTok{"_"}\NormalTok{,}\StringTok{"!"}\NormalTok{,}\StringTok{"?"}\NormalTok{)}
\NormalTok{wrd}
\KeywordTok{typeof}\NormalTok{(wrd)}

\KeywordTok{print}\NormalTok{(}\StringTok{"this is a sentence"}\NormalTok{)}
\KeywordTok{length}\NormalTok{(}\StringTok{"this is a sentence"}\NormalTok{)}
\KeywordTok{length}\NormalTok{(wrd)}

\NormalTok{### Use paste() function to manipulate strings}
\KeywordTok{paste}\NormalTok{(wrd,u,}\DataTypeTok{sep=}\StringTok{"/"}\NormalTok{)}

\NormalTok{### Use strsplit() to split strings}
\NormalTok{split.str=}\KeywordTok{strsplit}\NormalTok{(}\StringTok{"this is a sentence"}\NormalTok{, }\DataTypeTok{split=}\StringTok{" "}\NormalTok{)}

\KeywordTok{typeof}\NormalTok{(split.str)}
\KeywordTok{length}\NormalTok{(split.str)}
\NormalTok{split.str[[}\DecValTok{1}\NormalTok{]]}
\end{Highlighting}
\end{Shaded}

\subsection{Data frame}\label{data-frame}

\begin{Shaded}
\begin{Highlighting}[]
\NormalTok{### manipulation of data frame with base R}
\NormalTok{### No dplyr}
\NormalTok{### creating and accessing data frame}
\NormalTok{name =}\KeywordTok{c}\NormalTok{(}\StringTok{"Bob"}\NormalTok{,}\StringTok{"Jane"}\NormalTok{)}
\NormalTok{age =}\StringTok{ }\KeywordTok{c}\NormalTok{(}\DecValTok{19}\NormalTok{,}\DecValTok{21}\NormalTok{)}
\NormalTok{GPA =}\StringTok{ }\KeywordTok{c}\NormalTok{(}\FloatTok{3.2}\NormalTok{,}\FloatTok{3.6}\NormalTok{)}
\NormalTok{students =}\StringTok{ }\KeywordTok{data.frame}\NormalTok{(name,age,GPA,}\DataTypeTok{stringsAsFactors=}\NormalTok{F)}
\NormalTok{students}
\KeywordTok{str}\NormalTok{(students)}


\NormalTok{### Add new entry into the data frame}
\NormalTok{students =}\StringTok{ }\KeywordTok{rbind}\NormalTok{(students,}\KeywordTok{c}\NormalTok{(}\StringTok{"Ashley"}\NormalTok{,}\DecValTok{20}\NormalTok{,}\FloatTok{3.4}\NormalTok{))}

\NormalTok{### Add new variable:}
\NormalTok{students =}\StringTok{ }\KeywordTok{cbind}\NormalTok{(students,}\DataTypeTok{resident=}\KeywordTok{c}\NormalTok{(T,F,T))}

\NormalTok{students}
\NormalTok{##-------------------------------------##}
\end{Highlighting}
\end{Shaded}

\section{List}\label{list}

\begin{Shaded}
\begin{Highlighting}[]
\NormalTok{## List ####}
\NormalTok{bob=}\KeywordTok{list}\NormalTok{(}\DataTypeTok{name=}\StringTok{"bob"}\NormalTok{,}\DataTypeTok{age=}\DecValTok{19}\NormalTok{,}\DataTypeTok{school=}\StringTok{"UCD"}\NormalTok{,}\DataTypeTok{GPA=}\FloatTok{3.2}\NormalTok{,}\DataTypeTok{resident=}\NormalTok{T)}

\NormalTok{### Use lapply() with list}
\NormalTok{newlist =}\StringTok{ }\KeywordTok{list}\NormalTok{(}\DataTypeTok{field_one=} \KeywordTok{runif}\NormalTok{(}\DecValTok{10}\NormalTok{), }\DataTypeTok{field_two=}\KeywordTok{runif}\NormalTok{(}\DecValTok{20}\NormalTok{),}\DataTypeTok{field_three=}\KeywordTok{runif}\NormalTok{(}\DecValTok{50}\NormalTok{) );}
\KeywordTok{lapply}\NormalTok{(newlist,sum)}
\NormalTok{##-------------------------------------##}
\end{Highlighting}
\end{Shaded}

\section{\texorpdfstring{Functions in
\texttt{R}}{Functions in R}}\label{functions-in-r}

\begin{Shaded}
\begin{Highlighting}[]
\NormalTok{### Evaluating a function}
\NormalTok{x =}\StringTok{ }\KeywordTok{seq}\NormalTok{(}\DecValTok{0}\NormalTok{,}\DecValTok{1}\NormalTok{,}\FloatTok{0.05}\NormalTok{)}
\NormalTok{x}
\NormalTok{y =}\StringTok{ }\KeywordTok{sin}\NormalTok{(}\DecValTok{2}\OperatorTok{*}\NormalTok{pi}\OperatorTok{*}\NormalTok{x)}\OperatorTok{*}\KeywordTok{exp}\NormalTok{(}\OperatorTok{-}\DecValTok{4}\OperatorTok{*}\NormalTok{x}\OperatorTok{^}\DecValTok{2}\NormalTok{)}

\NormalTok{### Writing a function}
\NormalTok{sinexp =}\StringTok{ }\ControlFlowTok{function}\NormalTok{(z)\{}
\NormalTok{  y=}\KeywordTok{sin}\NormalTok{(}\DecValTok{2}\OperatorTok{*}\NormalTok{pi}\OperatorTok{*}\NormalTok{z)}\OperatorTok{*}\KeywordTok{exp}\NormalTok{(}\OperatorTok{-}\DecValTok{4}\OperatorTok{*}\NormalTok{z}\OperatorTok{^}\DecValTok{2}\NormalTok{);}
  \KeywordTok{return}\NormalTok{(y)}
\NormalTok{\}}

\KeywordTok{sinexp}\NormalTok{(x)}
\NormalTok{w=}\KeywordTok{sinexp}\NormalTok{(x)}

\NormalTok{y}\OperatorTok{==}\NormalTok{w}
\KeywordTok{all}\NormalTok{(y}\OperatorTok{==}\NormalTok{w)}
\KeywordTok{identical}\NormalTok{(y,w)}
\end{Highlighting}
\end{Shaded}

\section{Miscellaneous}\label{miscellaneous}

\begin{Shaded}
\begin{Highlighting}[]
\NormalTok{### Checking the state of work session}
\KeywordTok{objects}\NormalTok{()  }\CommentTok{# all the objects in the current R session}
\KeywordTok{getwd}\NormalTok{()  }\CommentTok{# print the current working directory}
\KeywordTok{remove}\NormalTok{()}
\CommentTok{#setwd()}

\CommentTok{#?persp}
\KeywordTok{example}\NormalTok{(persp)}
\end{Highlighting}
\end{Shaded}

\begin{Shaded}
\begin{Highlighting}[]
\CommentTok{#data()}
\end{Highlighting}
\end{Shaded}

\chapter{Linear algebra}\label{ch:algebra}

Reading materials: STA108\_LinearAlgebra\_S20.pdf.

\section{Vector}\label{vector}

\begin{Shaded}
\begin{Highlighting}[]
\NormalTok{  x =}\StringTok{ }\KeywordTok{c}\NormalTok{(}\DecValTok{1}\NormalTok{,}\DecValTok{2}\NormalTok{,}\DecValTok{4}\NormalTok{,}\FloatTok{2.5}\NormalTok{,}\OperatorTok{-}\FloatTok{1.7}\NormalTok{,}\DecValTok{1}\NormalTok{) }\CommentTok{# creates a vector (you can use "<-" instead of "=")}

  \KeywordTok{length}\NormalTok{(x) }\CommentTok{# check the length of x}
  \KeywordTok{class}\NormalTok{(x) }\CommentTok{# class of the object; you can also check typeof()}
\NormalTok{  ### Try z=list(); class(z);}

\NormalTok{  x[ }\DecValTok{3}\NormalTok{ ] }\CommentTok{# extract one element in x..}
\NormalTok{  x[}\KeywordTok{c}\NormalTok{(}\DecValTok{1}\NormalTok{,}\DecValTok{2}\NormalTok{,}\DecValTok{3}\NormalTok{)]}
\NormalTok{  x[}\DecValTok{2}\OperatorTok{:}\DecValTok{4}\NormalTok{]}

\NormalTok{  z<-}\KeywordTok{numeric}\NormalTok{(}\DecValTok{10}\NormalTok{)}

\NormalTok{  ### Transpose:}
\NormalTok{  x.t=}\KeywordTok{t}\NormalTok{(x);}
  \KeywordTok{length}\NormalTok{(x.t)}
\NormalTok{  x.t}
\NormalTok{  x}
  
\NormalTok{  y =}\StringTok{ }\NormalTok{x}\OperatorTok{^}\DecValTok{2}
\NormalTok{  ### Q: what is the length of y?}
  \KeywordTok{length}\NormalTok{(y)}

\NormalTok{  ### The "default" operation in R is elementwise:}
\NormalTok{  x}\OperatorTok{/}\NormalTok{y }\CommentTok{# elementwise division}
\NormalTok{  y}\OperatorTok{-}\NormalTok{x}\OperatorTok{*}\NormalTok{x }\CommentTok{# all elementwise..}

\NormalTok{  ### There are other operators..}
\NormalTok{  x }\OperatorTok\StringTok{ }\NormalTok{y }\CommentTok{# outer product of x and y}

\NormalTok{  x }\OperatorTok\StringTok{ }\NormalTok{y }\CommentTok{# Matrix product. Q: what is R doing here?}

  \KeywordTok{sum}\NormalTok{(x}\OperatorTok{*}\NormalTok{y)}
\end{Highlighting}
\end{Shaded}

\section{Matrix}\label{matrix}

\begin{Shaded}
\begin{Highlighting}[]
\NormalTok{M =}\StringTok{ }\KeywordTok{matrix}\NormalTok{(}\DecValTok{0}\NormalTok{,}\DecValTok{2}\NormalTok{,}\DecValTok{3}\NormalTok{) }\CommentTok{# matrix with 2 rows and 3 columns with all entries 0}
\CommentTok{# ?matrix}


\NormalTok{Xmat1 =}\StringTok{ }\KeywordTok{matrix}\NormalTok{(x,}\DecValTok{2}\NormalTok{,}\DecValTok{3}\NormalTok{) }\CommentTok{# 2 x 3 matrix formed by elements of vector x,}
\NormalTok{### ordered in *column-wise* manner}
\NormalTok{Xmat2 =}\StringTok{ }\KeywordTok{matrix}\NormalTok{(x,}\DecValTok{2}\NormalTok{,}\DecValTok{3}\NormalTok{,}\DataTypeTok{byrow=}\NormalTok{T) }\CommentTok{# ordered row-wise}


\NormalTok{M<-}\StringTok{ }\NormalTok{Xmat1;}
\KeywordTok{dimnames}\NormalTok{(M)}
\KeywordTok{rownames}\NormalTok{(M) =}\StringTok{ }\KeywordTok{c}\NormalTok{(}\StringTok{"first"}\NormalTok{,}\StringTok{"second"}\NormalTok{)}
\KeywordTok{colnames}\NormalTok{(M) =}\StringTok{ }\KeywordTok{c}\NormalTok{(}\StringTok{"a"}\NormalTok{,}\StringTok{"b"}\NormalTok{,}\StringTok{"c"}\NormalTok{)}

\KeywordTok{dimnames}\NormalTok{(M)}
\KeywordTok{typeof}\NormalTok{(}\KeywordTok{dimnames}\NormalTok{(M))}

\NormalTok{### access elements through names of rows/columns}
\NormalTok{M[,}\StringTok{"a"}\NormalTok{]}
\NormalTok{M[,}\DecValTok{1}\NormalTok{]}

\NormalTok{M[,}\KeywordTok{c}\NormalTok{(}\StringTok{"a"}\NormalTok{,}\StringTok{"c"}\NormalTok{)]}
\NormalTok{M[,}\KeywordTok{c}\NormalTok{(}\DecValTok{1}\NormalTok{,}\DecValTok{3}\NormalTok{)]}

\NormalTok{M }\OperatorTok\StringTok{ }\KeywordTok{c}\NormalTok{(}\DecValTok{4}\NormalTok{,}\OperatorTok{-}\DecValTok{3}\NormalTok{,}\DecValTok{1}\NormalTok{)  }\CommentTok{# multiply matrix M to vector c(4,-3,1)}
\NormalTok{M }\OperatorTok{+}\StringTok{ }\KeywordTok{c}\NormalTok{(}\DecValTok{1}\NormalTok{,}\OperatorTok{-}\DecValTok{2}\NormalTok{) }\CommentTok{# add the column vector c(1,-2) to each column of M}

\NormalTok{M2 =}\StringTok{ }\KeywordTok{cbind}\NormalTok{(M,}\KeywordTok{c}\NormalTok{(}\DecValTok{4}\NormalTok{,}\DecValTok{5}\NormalTok{))  }\CommentTok{# create matrix M2 by appending column c(4,5) to M}
\NormalTok{### How to append row to M2?}
\NormalTok{new_row =}\StringTok{ }\KeywordTok{c}\NormalTok{(}\DecValTok{1}\NormalTok{,}\DecValTok{3}\NormalTok{,}\DecValTok{5}\NormalTok{,}\DecValTok{7}\NormalTok{)}
\CommentTok{#?cbind}
\CommentTok{#?rbind}

\NormalTok{### What if we call length() on a matrix?}
\KeywordTok{length}\NormalTok{(M)}
\KeywordTok{dim}\NormalTok{(M)}




\NormalTok{### Transpose of the matrix}
\NormalTok{M.t=}\KeywordTok{t}\NormalTok{(M);}
\KeywordTok{identical}\NormalTok{(}\KeywordTok{t}\NormalTok{(M.t),M)}

\NormalTok{### Diagonal and identity matrix}

\NormalTok{## Identity matrix}
\KeywordTok{diag}\NormalTok{(}\DecValTok{5}\NormalTok{)}
\NormalTok{## Diagonal matrix}
\KeywordTok{diag}\NormalTok{(}\KeywordTok{c}\NormalTok{(}\DecValTok{1}\NormalTok{,}\DecValTok{2}\NormalTok{,}\DecValTok{3}\NormalTok{,}\DecValTok{4}\NormalTok{,}\DecValTok{5}\NormalTok{))}
\NormalTok{## Symmatric matrix}
\NormalTok{A=}\StringTok{ }\KeywordTok{matrix}\NormalTok{(}\KeywordTok{rnorm}\NormalTok{(}\DecValTok{9}\NormalTok{),}\DecValTok{3}\NormalTok{,}\DecValTok{3}\NormalTok{);}
\NormalTok{A=A}\OperatorTok{+}\KeywordTok{t}\NormalTok{(A); }\CommentTok{# To make sure it is symmetric }
\KeywordTok{identical}\NormalTok{(A,}\KeywordTok{t}\NormalTok{(A))}


\NormalTok{### Trace of a matrix:}
\KeywordTok{sum}\NormalTok{(}\KeywordTok{diag}\NormalTok{(A))}
\NormalTok{## Define a tr() function to match the common convention}
\NormalTok{tr<-}\ControlFlowTok{function}\NormalTok{(x)\{}
\NormalTok{  out=}\KeywordTok{sum}\NormalTok{(}\KeywordTok{diag}\NormalTok{(x));}
  \KeywordTok{return}\NormalTok{(out)}
\NormalTok{\}}
\KeywordTok{tr}\NormalTok{(A)}

\NormalTok{## Verify some properties of the trace }
\NormalTok{B=}\StringTok{ }\KeywordTok{matrix}\NormalTok{(}\KeywordTok{rnorm}\NormalTok{(}\DecValTok{9}\NormalTok{),}\DecValTok{3}\NormalTok{,}\DecValTok{3}\NormalTok{);}
\KeywordTok{tr}\NormalTok{(A)}
\KeywordTok{tr}\NormalTok{(}\KeywordTok{t}\NormalTok{(A))}

\KeywordTok{tr}\NormalTok{(A}\OperatorTok{+}\NormalTok{B)}
\KeywordTok{tr}\NormalTok{(A)}\OperatorTok{+}\KeywordTok{tr}\NormalTok{(B)}

\KeywordTok{tr}\NormalTok{(A}\OperatorTok\NormalTok{B)}
\KeywordTok{tr}\NormalTok{(B}\OperatorTok\NormalTok{A) }\CommentTok{# if the dimensions check out..}

\KeywordTok{tr}\NormalTok{(A)}
\KeywordTok{sum}\NormalTok{(}\KeywordTok{eigen}\NormalTok{(A)}\OperatorTok{$}\NormalTok{values)}

\NormalTok{### Matrix-vector product:}

\NormalTok{Xmat1 =}\StringTok{ }\KeywordTok{matrix}\NormalTok{(}\KeywordTok{rnorm}\NormalTok{(}\DecValTok{6}\NormalTok{),}\DecValTok{2}\NormalTok{,}\DecValTok{3}\NormalTok{) }
\NormalTok{xvec1=}\KeywordTok{rnorm}\NormalTok{(}\DecValTok{3}\NormalTok{)}
\NormalTok{Xmat1}\OperatorTok\NormalTok{xvec1}

\NormalTok{### elementwise product of Xmat1 and Xmat2}
\NormalTok{Xmat1}\OperatorTok{*}\NormalTok{Xmat2}



\NormalTok{### Matrix multiplication}
\CommentTok{# Xmat1%*%Xmat2}
\KeywordTok{dim}\NormalTok{(Xmat1)}
\KeywordTok{dim}\NormalTok{(Xmat2)}
\NormalTok{Xmat1}\OperatorTok\KeywordTok{t}\NormalTok{(Xmat2)}
\NormalTok{### Verify another property of transpose:}
\KeywordTok{t}\NormalTok{( Xmat1}\OperatorTok\KeywordTok{t}\NormalTok{(Xmat2) ) }
\NormalTok{Xmat2}\OperatorTok\KeywordTok{t}\NormalTok{(Xmat1)}
\CommentTok{# solve(Xmat1) # what does the error message say?}
\end{Highlighting}
\end{Shaded}

\begin{Shaded}
\begin{Highlighting}[]
\NormalTok{### Inverse of a marix}
\CommentTok{# set.seed(1)}
\CommentTok{# A= matrix(rnorm(9),3,3);}
\CommentTok{# set.seed(): specify the random seed for generating random numbers }
\CommentTok{# Because a computer (if not a quantum one) can only generate pseudorandomness }
\CommentTok{# set.seed(1)}
\CommentTok{# B=matrix(rnorm(9),3,3);}
\CommentTok{# identical(A,B)}

\KeywordTok{set.seed}\NormalTok{(}\DecValTok{1}\NormalTok{)}
\NormalTok{A=}\StringTok{ }\KeywordTok{matrix}\NormalTok{(}\KeywordTok{rnorm}\NormalTok{(}\DecValTok{9}\NormalTok{),}\DecValTok{3}\NormalTok{,}\DecValTok{3}\NormalTok{);}
\NormalTok{A.inv=}\KeywordTok{solve}\NormalTok{(A);}

\NormalTok{A.inv}\OperatorTok\NormalTok{A }
\NormalTok{A}\OperatorTok\NormalTok{A.inv}


\NormalTok{B=}\KeywordTok{matrix}\NormalTok{(}\KeywordTok{rnorm}\NormalTok{(}\DecValTok{9}\NormalTok{),}\DecValTok{3}\NormalTok{,}\DecValTok{3}\NormalTok{);}
\KeywordTok{identical}\NormalTok{(A,B)}
\NormalTok{B.inv=}\KeywordTok{solve}\NormalTok{(B);}

\NormalTok{AB.inv=}\KeywordTok{solve}\NormalTok{(A}\OperatorTok\NormalTok{B)}

\NormalTok{B.inv}\OperatorTok\NormalTok{A.inv}
\NormalTok{AB.inv}

\KeywordTok{max}\NormalTok{(}\KeywordTok{abs}\NormalTok{(B.inv}\OperatorTok\NormalTok{A.inv}\OperatorTok{-}\NormalTok{AB.inv))}

\CommentTok{# Why?}
\CommentTok{# This is because }
\CommentTok{# (A%*%B)%*%AB.int= I if AB.inv is the inverse of AB=A%*%B}
\CommentTok{# and that }
\CommentTok{# A%*%B %*%B.inv%*%A.inv=A%*%(B%*%B.inv)%*%A.inv =A%*%A.inv=I}

\NormalTok{### Projection matrix}

\KeywordTok{set.seed}\NormalTok{(}\DecValTok{12}\NormalTok{)}
\NormalTok{X =}\StringTok{ }\KeywordTok{matrix}\NormalTok{(}\KeywordTok{rnorm}\NormalTok{(}\DecValTok{6}\NormalTok{),}\DecValTok{3}\NormalTok{,}\DecValTok{2}\NormalTok{) }
\NormalTok{P=X}\OperatorTok\KeywordTok{solve}\NormalTok{(}\KeywordTok{t}\NormalTok{(X)}\OperatorTok\NormalTok{X)}\OperatorTok\KeywordTok{t}\NormalTok{(X)}

\NormalTok{P}\OperatorTok\NormalTok{X}
\NormalTok{X}
\NormalTok{(}\KeywordTok{diag}\NormalTok{(}\DecValTok{3}\NormalTok{)}\OperatorTok{-}\NormalTok{P)}\OperatorTok\NormalTok{X}

\NormalTok{### Rank of a matrix:}
\KeywordTok{qr}\NormalTok{(X)}\OperatorTok{$}\NormalTok{rank }\CommentTok{# Check out the qr() function yourself!}
\CommentTok{# The rank of X is 2 when the dimension of X is 3 by 2}
\CommentTok{# This means that the matrix X is full rank }


\NormalTok{## How about x.transpose times X?}
\NormalTok{XtX=}\KeywordTok{t}\NormalTok{(X)}\OperatorTok\NormalTok{X}
\KeywordTok{dim}\NormalTok{(XtX)}
\KeywordTok{qr}\NormalTok{(XtX)}\OperatorTok{$}\NormalTok{rank}
\CommentTok{# This is a full rank matrix}

\NormalTok{## How about X times X.transpose?}
\NormalTok{XXt=X}\OperatorTok\KeywordTok{t}\NormalTok{(X)}
\KeywordTok{dim}\NormalTok{(XXt)}
\KeywordTok{qr}\NormalTok{(XXt)}\OperatorTok{$}\NormalTok{rank}
\CommentTok{# This is NOT a full rank matrix }

\CommentTok{# Rank and inverse}
\KeywordTok{solve}\NormalTok{(XtX)}
\CommentTok{# solve(XXt)}

\NormalTok{### Determinants}
\KeywordTok{det}\NormalTok{(A)}
\DecValTok{1}\OperatorTok{/}\KeywordTok{det}\NormalTok{(}\KeywordTok{solve}\NormalTok{(A)) }
\KeywordTok{det}\NormalTok{(A}\OperatorTok\NormalTok{B)}
\KeywordTok{det}\NormalTok{(A)}\OperatorTok\KeywordTok{det}\NormalTok{(B)}
\KeywordTok{det}\NormalTok{(}\DecValTok{3}\OperatorTok{*}\NormalTok{A)}

\NormalTok{A.symm =}\StringTok{ }\NormalTok{(A}\OperatorTok{+}\KeywordTok{t}\NormalTok{(A))}\OperatorTok{/}\DecValTok{2}\NormalTok{;}
\KeywordTok{det}\NormalTok{(A.symm)}
\NormalTok{A.symm.eigenvalues=}\KeywordTok{eigen}\NormalTok{(A.symm)}\OperatorTok{$}\NormalTok{values}
\KeywordTok{prod}\NormalTok{(A.symm.eigenvalues)}
\CommentTok{# Trace: summation of eigenvalues}
\CommentTok{# Determinant: product of eigenvalues (for symmetric matrix only!)}


\NormalTok{### Eigenvalue and eigenvector}
\NormalTok{A.symm.eigenvectors=}\KeywordTok{eigen}\NormalTok{(A.symm)}\OperatorTok{$}\NormalTok{vectors}

\NormalTok{A.symm}\OperatorTok\NormalTok{A.symm.eigenvectors[,}\DecValTok{1}\NormalTok{]}
\NormalTok{A.symm.eigenvalues[}\DecValTok{1}\NormalTok{]}\OperatorTok{*}\NormalTok{A.symm.eigenvectors[,}\DecValTok{1}\NormalTok{]}
\NormalTok{A.symm.eigenvalues}
\KeywordTok{qr}\NormalTok{(A.symm)}\OperatorTok{$}\NormalTok{rank}

\KeywordTok{qr}\NormalTok{(XXt)}\OperatorTok{$}\NormalTok{rank}
\KeywordTok{eigen}\NormalTok{(XXt) }\CommentTok{# the last eigenvalue equals zero, with some numerical error }



\NormalTok{### Matrix definiteness}
\KeywordTok{eigen}\NormalTok{(XtX)}
\KeywordTok{eigen}\NormalTok{(P)}


\NormalTok{### Eigenvalue decomposition }
\NormalTok{A.symm.eigenvectors}\OperatorTok\KeywordTok{diag}\NormalTok{(A.symm.eigenvalues)}\OperatorTok\KeywordTok{t}\NormalTok{(A.symm.eigenvectors)}
\NormalTok{A.symm}


\NormalTok{### Singular value decomposition}
\KeywordTok{svd}\NormalTok{(X)  }\CommentTok{# singular value decomposition of X}
\end{Highlighting}
\end{Shaded}

\section{Other operations on vectors and
matrices}\label{other-operations-on-vectors-and-matrices}

\begin{Shaded}
\begin{Highlighting}[]
\NormalTok{### How does R handle mathematically invalid operations?}
\NormalTok{### Example 1:}
\NormalTok{ x =}\StringTok{ }\KeywordTok{c}\NormalTok{(}\DecValTok{1}\NormalTok{,}\DecValTok{2}\NormalTok{,}\DecValTok{4}\NormalTok{,}\FloatTok{2.5}\NormalTok{,}\OperatorTok{-}\FloatTok{1.7}\NormalTok{,}\DecValTok{1}\NormalTok{) }
\NormalTok{w =}\DecValTok{1}\OperatorTok{:}\DecValTok{3}  \CommentTok{# short-handed for c(1,2,3)}
\NormalTok{### Q: What is the length of w?}

\NormalTok{x }\OperatorTok{+}\StringTok{ }\NormalTok{w  }\CommentTok{# no error displayed, but not quite a mathematically valid operation}
\NormalTok{### Q: what did R just do???}
\KeywordTok{c}\NormalTok{(x[}\DecValTok{1}\OperatorTok{:}\DecValTok{3}\NormalTok{]}\OperatorTok{+}\NormalTok{w, x[}\DecValTok{4}\OperatorTok{:}\DecValTok{6}\NormalTok{]}\OperatorTok{+}\NormalTok{w)}

\NormalTok{### What if...}
\CommentTok{# Create a vector of length 4 (from 1 to 4)}
\NormalTok{n<-}\DecValTok{1}\OperatorTok{:}\DecValTok{4}\NormalTok{;}
\NormalTok{x}\OperatorTok{+}\NormalTok{n}

\NormalTok{### Example 2:}

\NormalTok{z =}\StringTok{ }\KeywordTok{sin}\NormalTok{(x) }\OperatorTok{+}\StringTok{ }\NormalTok{x}\OperatorTok{^}\NormalTok{\{}\FloatTok{1.5}\NormalTok{\}}\OperatorTok{*}\StringTok{ }\KeywordTok{exp}\NormalTok{(}\OperatorTok{-}\NormalTok{x}\OperatorTok{^}\DecValTok{3}\NormalTok{)}
\NormalTok{### What is (-1.7)^\{1.5\}???}
\NormalTok{### How did R handle this?}
\NormalTok{x[}\DecValTok{5}\NormalTok{]}
\NormalTok{z[}\DecValTok{5}\NormalTok{]}
\end{Highlighting}
\end{Shaded}

\subsection{Array}\label{array}

\begin{Shaded}
\begin{Highlighting}[]
\NormalTok{### Arrays, a.k.a. tensors}
\NormalTok{### A matrix is a two-way array}

\CommentTok{# ?array # check for R documentation on array}

\NormalTok{A =}\StringTok{ }\KeywordTok{array}\NormalTok{(}\KeywordTok{rnorm}\NormalTok{(}\DecValTok{2}\OperatorTok{*}\DecValTok{3}\OperatorTok{*}\DecValTok{4}\NormalTok{),}\DataTypeTok{dim=}\KeywordTok{c}\NormalTok{(}\DecValTok{2}\NormalTok{,}\DecValTok{3}\NormalTok{,}\DecValTok{4}\NormalTok{))  }\CommentTok{# creates an array with entries that are i.i.d. N(0,1)}
\CommentTok{#?rnorm}
\KeywordTok{dim}\NormalTok{(A) }\CommentTok{# checks the dimension of A}

\NormalTok{A[,}\DecValTok{1}\NormalTok{,] }\CommentTok{# returns the 2 x 4 matrix formed by restricting the second dimension index to 1}

\NormalTok{### What is the dimension of A[,1,]?}

\NormalTok{A[}\DecValTok{2}\NormalTok{,,] }\CommentTok{# returns the 3 x 4 matrix formed by restricting the first dimension index to 2}

\NormalTok{A[}\DecValTok{2}\NormalTok{,}\DecValTok{1}\NormalTok{,}\DecValTok{2}\NormalTok{]}
\NormalTok{A[A}\OperatorTok{<}\DecValTok{0}\NormalTok{]}
\end{Highlighting}
\end{Shaded}

\subsection{Datasets as matrices}\label{datasets-as-matrices}

\begin{Shaded}
\begin{Highlighting}[]
\KeywordTok{library}\NormalTok{(AER) }\CommentTok{# install.packages('AER')}
\KeywordTok{data}\NormalTok{(}\StringTok{"Fatalities"}\NormalTok{)}
\CommentTok{#?Fatalities}

\KeywordTok{data}\NormalTok{(}\StringTok{"CreditCard"}\NormalTok{)}
\NormalTok{dat.CC=CreditCard}
\CommentTok{#?CreditCard}
\KeywordTok{dim}\NormalTok{(dat.CC)}
\KeywordTok{head}\NormalTok{(dat.CC)}

\NormalTok{Y.CC=dat.CC[,}\DecValTok{1}\NormalTok{]}\OperatorTok{==}\StringTok{'yes'}\NormalTok{;}
\NormalTok{X.CC=}\KeywordTok{as.matrix}\NormalTok{(dat.CC[,}\KeywordTok{c}\NormalTok{(}\StringTok{'reports'}\NormalTok{,}\StringTok{'age'}\NormalTok{,}\StringTok{'income'}\NormalTok{,}\StringTok{'share'}\NormalTok{,}\StringTok{'expenditure'}\NormalTok{,}\StringTok{'dependents'}\NormalTok{,}\StringTok{'months'}\NormalTok{,}\StringTok{'majorcards'}\NormalTok{,}\StringTok{'active'}\NormalTok{)])}

\CommentTok{# Check the rank of the covariates:                 }
\KeywordTok{qr}\NormalTok{(X.CC)}\OperatorTok{$}\NormalTok{rank                }
\KeywordTok{dim}\NormalTok{(X.CC) }\CommentTok{# full rank, i.e., no multicolinearity}


\CommentTok{# Linear regression:}
\NormalTok{fit.CC=}\KeywordTok{lm}\NormalTok{(Y.CC}\OperatorTok{~}\NormalTok{X.CC}\OperatorTok{-}\DecValTok{1}\NormalTok{);}
\KeywordTok{summary}\NormalTok{(fit.CC)}

\CommentTok{# Fitted coefficients:}
\NormalTok{beta.hat=}\KeywordTok{solve}\NormalTok{(}\KeywordTok{t}\NormalTok{(X.CC)}\OperatorTok\NormalTok{X.CC)}\OperatorTok\KeywordTok{t}\NormalTok{(X.CC)}\OperatorTok\NormalTok{Y.CC;}
\NormalTok{fit.CC}\OperatorTok{$}\NormalTok{coefficients}\OperatorTok{-}\NormalTok{beta.hat}

\CommentTok{# Hat matrix: }
\NormalTok{P.CC=X.CC}\OperatorTok\KeywordTok{solve}\NormalTok{(}\KeywordTok{t}\NormalTok{(X.CC)}\OperatorTok\NormalTok{X.CC)}\OperatorTok\KeywordTok{t}\NormalTok{(X.CC)}
\CommentTok{# Fitted Y}
\NormalTok{Yhat.CC=P.CC}\OperatorTok\NormalTok{Y.CC }
\CommentTok{# Compare with the fits from lm()}
\KeywordTok{max}\NormalTok{(}\KeywordTok{abs}\NormalTok{(fit.CC}\OperatorTok{$}\NormalTok{fitted.values}\OperatorTok{-}\NormalTok{Yhat.CC))}
\end{Highlighting}
\end{Shaded}

\chapter{Simulation and visualization}\label{ch:sim}

Reading materials: Chapters 3, 5, and 7 in
\href{https://r4ds.had.co.nz/index.html}{R for data science} by Garrett
Grolemund and Hadley Wickham (optional).

\section{Simulation and visualization:
univariate}\label{simulation-and-visualization-univariate}

\begin{Shaded}
\begin{Highlighting}[]
\NormalTok{## Random seed}
\KeywordTok{set.seed}\NormalTok{(}\DecValTok{1}\NormalTok{) }\CommentTok{# set random number generator seed for reproducibility}
\KeywordTok{runif}\NormalTok{(}\DecValTok{1}\NormalTok{)}
\KeywordTok{runif}\NormalTok{(}\DecValTok{1}\NormalTok{) }

\KeywordTok{set.seed}\NormalTok{(}\DecValTok{2}\NormalTok{)}
\KeywordTok{runif}\NormalTok{(}\DecValTok{1}\NormalTok{)}

\KeywordTok{set.seed}\NormalTok{(}\DecValTok{1}\NormalTok{)}
\KeywordTok{runif}\NormalTok{(}\DecValTok{1}\NormalTok{)}
\KeywordTok{runif}\NormalTok{(}\DecValTok{1}\NormalTok{)}
\end{Highlighting}
\end{Shaded}

\begin{Shaded}
\begin{Highlighting}[]
\NormalTok{## Uniform distribution}
\NormalTok{### Draw ten uniform random variables:}
\KeywordTok{runif}\NormalTok{(}\DecValTok{10}\NormalTok{)}
\CommentTok{# use ?runif to see what other options are available }

\CommentTok{# To help understand this function, we can draw the density plot for this distribution}
\NormalTok{x.grid=}\KeywordTok{seq}\NormalTok{(}\DataTypeTok{from=}\DecValTok{0}\NormalTok{,}\DataTypeTok{to=}\DecValTok{1}\NormalTok{,}\DataTypeTok{length.out=}\DecValTok{100}\NormalTok{)}
\NormalTok{unif.pdf=}\KeywordTok{dunif}\NormalTok{(x.grid,}\DataTypeTok{min=}\DecValTok{0}\NormalTok{,}\DataTypeTok{max=}\DecValTok{1}\NormalTok{);}
\KeywordTok{plot}\NormalTok{(}\DataTypeTok{y=}\NormalTok{unif.pdf,}\DataTypeTok{x=}\NormalTok{x.grid,}\DataTypeTok{type=}\StringTok{'l'}\NormalTok{)}
\end{Highlighting}
\end{Shaded}

\begin{Shaded}
\begin{Highlighting}[]
\CommentTok{# Compare this with a histogram from a set of random variables from runif:}

\NormalTok{unif.rv.}\DecValTok{100}\NormalTok{=}\KeywordTok{runif}\NormalTok{(}\DecValTok{100}\NormalTok{);}
\KeywordTok{hist}\NormalTok{(unif.rv.}\DecValTok{100}\NormalTok{,}\DataTypeTok{freq=}\OtherTok{FALSE}\NormalTok{,}\DataTypeTok{main=}\StringTok{'Uniform'}\NormalTok{,}\DataTypeTok{xlab=}\StringTok{'X'}\NormalTok{)}
\KeywordTok{lines}\NormalTok{(}\DataTypeTok{y=}\NormalTok{unif.pdf,}\DataTypeTok{x=}\NormalTok{x.grid,}\DataTypeTok{col=}\StringTok{'blue'}\NormalTok{,}\DataTypeTok{lwd=}\DecValTok{3}\NormalTok{)}
\CommentTok{# Further compare it with a smooth density estimation based on the samples:}
\NormalTok{unif.pdf.est=}\KeywordTok{density}\NormalTok{(unif.rv.}\DecValTok{100}\NormalTok{);}
\KeywordTok{lines}\NormalTok{(unif.pdf.est,}\DataTypeTok{col=}\StringTok{'red'}\NormalTok{,}\DataTypeTok{lwd=}\DecValTok{3}\NormalTok{)}
\end{Highlighting}
\end{Shaded}

\begin{Shaded}
\begin{Highlighting}[]
\CommentTok{# Wrap up the above plotting code as a function }

\NormalTok{plot.density.empirical<-}\ControlFlowTok{function}\NormalTok{(rvs,pdf.true,pdf.grid,main)\{}
\KeywordTok{hist}\NormalTok{(rvs,}\DataTypeTok{freq=}\OtherTok{FALSE}\NormalTok{,}\DataTypeTok{main=}\NormalTok{main,}\DataTypeTok{xlab=}\StringTok{'X'}\NormalTok{)}
\KeywordTok{lines}\NormalTok{(}\DataTypeTok{y=}\NormalTok{pdf.true,}\DataTypeTok{x=}\NormalTok{pdf.grid,}\DataTypeTok{col=}\StringTok{'blue'}\NormalTok{,}\DataTypeTok{lwd=}\DecValTok{3}\NormalTok{)}
\CommentTok{# Further compare it with a smooth density estimation based on the samples:}
\NormalTok{pdf.est=}\KeywordTok{density}\NormalTok{(rvs);}
\KeywordTok{lines}\NormalTok{(pdf.est,}\DataTypeTok{col=}\StringTok{'red'}\NormalTok{,}\DataTypeTok{lwd=}\DecValTok{3}\NormalTok{)}
\NormalTok{\}}


\KeywordTok{plot.density.empirical}\NormalTok{(unif.rv.}\DecValTok{100}\NormalTok{,unif.pdf,x.grid,}\DataTypeTok{main=}\StringTok{'Uniform'}\NormalTok{)}
\end{Highlighting}
\end{Shaded}

\begin{Shaded}
\begin{Highlighting}[]
\NormalTok{### Normal}
\NormalTok{### Draw ten normal random variables with mean 0 and variance 2}
\KeywordTok{rnorm}\NormalTok{(}\DecValTok{10}\NormalTok{,}\DataTypeTok{mean=}\DecValTok{0}\NormalTok{,}\DataTypeTok{sd=}\KeywordTok{sqrt}\NormalTok{(}\DecValTok{2}\NormalTok{))}

\NormalTok{### There are many other distributions in R}
\NormalTok{### try ?rchisq, ?rf, ?rt, ?rbeta, ?rpois, ...}
\end{Highlighting}
\end{Shaded}

\begin{Shaded}
\begin{Highlighting}[]
\NormalTok{## Other distributions}
\NormalTok{### Poisson  ?rpois}

\NormalTok{### Exponential ?rexp}

\NormalTok{### t ?rt}


\NormalTok{### Chisq ?rchisq}


\NormalTok{### F ?rf}
\end{Highlighting}
\end{Shaded}

\begin{Shaded}
\begin{Highlighting}[]
\NormalTok{## Sampling from a user-specified vector }

\NormalTok{### Draw samples from any vectors using sample()}
\NormalTok{wrd =}\StringTok{ }\KeywordTok{c}\NormalTok{(}\StringTok{"yet"}\NormalTok{, }\StringTok{"a"}\NormalTok{,}\StringTok{"new"}\NormalTok{,}\StringTok{"sentence"}\NormalTok{)}
\KeywordTok{sample}\NormalTok{(wrd,}\DataTypeTok{size=}\DecValTok{2}\NormalTok{)}
\CommentTok{# use ?sample to see other options}
\end{Highlighting}
\end{Shaded}

\begin{Shaded}
\begin{Highlighting}[]
\NormalTok{### Visulize the simulated data}

\NormalTok{y<-}\KeywordTok{rnorm}\NormalTok{(}\DecValTok{100}\NormalTok{);}
\CommentTok{# We start with the very basic histogram }
\KeywordTok{hist}\NormalTok{(y)}
\end{Highlighting}
\end{Shaded}

\begin{Shaded}
\begin{Highlighting}[]
\CommentTok{# Use ?hist to modify the plot }

\CommentTok{# And the boxplot:}
\KeywordTok{boxplot}\NormalTok{(y)}
\end{Highlighting}
\end{Shaded}

\begin{Shaded}
\begin{Highlighting}[]
\CommentTok{# We can also draw visualize the data using ggplot2}
\KeywordTok{library}\NormalTok{(ggplot2)}
\NormalTok{dat=}\KeywordTok{data.frame}\NormalTok{(}\DataTypeTok{y=}\NormalTok{y)}
\KeywordTok{ggplot}\NormalTok{(dat, }\KeywordTok{aes}\NormalTok{(}\DataTypeTok{x=}\DecValTok{1}\NormalTok{,}\DataTypeTok{y=}\NormalTok{y)) }\OperatorTok{+}\StringTok{ }
\StringTok{  }\KeywordTok{geom_violin}\NormalTok{(}\DataTypeTok{trim=}\OtherTok{FALSE}\NormalTok{)}
\end{Highlighting}
\end{Shaded}

\begin{Shaded}
\begin{Highlighting}[]
\CommentTok{# Read more about the violin plots in the post here:}
\CommentTok{# http://www.sthda.com/english/wiki/ggplot2-violin-plot-quick-start-guide-r-software-and-data-visualization}
\end{Highlighting}
\end{Shaded}

\section{Simulation and visualization:
multivariate}\label{simulation-and-visualization-multivariate}

\begin{Shaded}
\begin{Highlighting}[]
\CommentTok{# We can generate many independent univariate random variables using code in the previous section }

\CommentTok{# Here we will generate dependent random variables, for instance, }
\NormalTok{n=}\DecValTok{50}\NormalTok{;}\CommentTok{# sample size}
\NormalTok{x1=}\KeywordTok{rnorm}\NormalTok{(n);}
\NormalTok{x2=x1}\OperatorTok{*}\DecValTok{2}\OperatorTok{+}\KeywordTok{runif}\NormalTok{(n)}\OperatorTok{*}\DecValTok{2}\NormalTok{;}
\NormalTok{x3=x1}\OperatorTok{*}\NormalTok{x2}\OperatorTok{*}\KeywordTok{rpois}\NormalTok{(n,}\DataTypeTok{lambda=}\DecValTok{3}\NormalTok{);}

\CommentTok{# You can also generate a random variable using its probability density function using importance sampling}
\CommentTok{# We will be fine with the simple data generating method in this class }

\CommentTok{# We put the three random variables into one data.frame for ease of plotting }
\NormalTok{dat.multi<-}\KeywordTok{data.frame}\NormalTok{(}\DataTypeTok{x1=}\NormalTok{x1,}\DataTypeTok{x2=}\NormalTok{x2,}\DataTypeTok{x3=}\NormalTok{x3)}
\end{Highlighting}
\end{Shaded}

\begin{Shaded}
\begin{Highlighting}[]
\NormalTok{### Visualization}

\NormalTok{### Pairwise scatter plot}
\KeywordTok{pairs}\NormalTok{(dat.multi,}\DataTypeTok{pch=}\StringTok{'.'}\NormalTok{,}\DataTypeTok{cex=}\DecValTok{4}\NormalTok{)}
\end{Highlighting}
\end{Shaded}

\begin{Shaded}
\begin{Highlighting}[]
\NormalTok{### Visualization in ggplot2}
\KeywordTok{library}\NormalTok{(GGally)}
\KeywordTok{ggpairs}\NormalTok{(dat.multi)}
\end{Highlighting}
\end{Shaded}

\begin{Shaded}
\begin{Highlighting}[]
\NormalTok{### For the pair x1 and x3}

\CommentTok{# A more informative scatterplot, using base R }
\KeywordTok{plot}\NormalTok{(}\DataTypeTok{x=}\NormalTok{dat.multi}\OperatorTok{$}\NormalTok{x1,}\DataTypeTok{y=}\NormalTok{dat.multi}\OperatorTok{$}\NormalTok{x3,}\DataTypeTok{data=}\NormalTok{dat.multi,}\DataTypeTok{pch=}\DecValTok{16}\NormalTok{,}\DataTypeTok{cex=}\DecValTok{2}\NormalTok{)}
\KeywordTok{lines}\NormalTok{(}\KeywordTok{lowess}\NormalTok{(dat.multi}\OperatorTok{$}\NormalTok{x1, dat.multi}\OperatorTok{$}\NormalTok{x3), }\DataTypeTok{col=}\DecValTok{2}\NormalTok{,}\DataTypeTok{lwd=}\DecValTok{3}\NormalTok{)}
\end{Highlighting}
\end{Shaded}

\begin{Shaded}
\begin{Highlighting}[]
\CommentTok{# Or using ggplot2}
\KeywordTok{ggplot}\NormalTok{(}\DataTypeTok{data =}\NormalTok{ dat.multi) }\OperatorTok{+}\StringTok{ }
\StringTok{  }\KeywordTok{geom_point}\NormalTok{(}\DataTypeTok{mapping =} \KeywordTok{aes}\NormalTok{(}\DataTypeTok{x =}\NormalTok{ x1, }\DataTypeTok{y =}\NormalTok{ x3)) }\OperatorTok{+}
\StringTok{  }\KeywordTok{geom_smooth}\NormalTok{(}\DataTypeTok{mapping =} \KeywordTok{aes}\NormalTok{(}\DataTypeTok{x =}\NormalTok{ x1, }\DataTypeTok{y =}\NormalTok{ x3))}
\end{Highlighting}
\end{Shaded}

\bibliography{book.bib,packages.bib}


\end{document}
